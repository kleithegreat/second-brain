\documentclass{article}
\usepackage{amsmath,amssymb,amsthm}
\usepackage{fancyhdr}
\usepackage{enumerate}
\usepackage{graphicx}
\usepackage{caption}
\usepackage{float}
\usepackage{xcolor}
\usepackage{listings}
\usepackage{ulem}

\pagestyle{fancy}
\fancyhf{}
\lhead{4th Homework - CSCE 312 503}
\rhead{Kevin Lei}
\renewcommand{\headrulewidth}{0.4pt}
\renewcommand{\arraystretch}{1.2}


\begin{document}

\section*{Bryant Chapter 4}

\section*{4.13 - correct}
\begin{tabular}{l l l}
    Stage & Generic & Specific \\
    & \texttt{irmovq} V, rB & \texttt{irmovq \$128, \%rsp} \\
    \hline
    Fetch & icode:ifun $\leftarrow$ $M_1$[PC] & icode:ifun $\leftarrow$ $M_1$[0x016] = 0x3:0x0 \\
    & rA:rB $\leftarrow$ $M_1$[PC+1] & rA:rB $\leftarrow$ $M_1$[0x017] = 0xF:0x4 \\
    & valC $\leftarrow$ $M_8$[PC+2] & valC $\leftarrow$ $M_8$[0x018] = 0x80 \\
    & valP $\leftarrow$ PC + 10 & valP $\leftarrow$ 0x016 + 0xA = 0x020 \\
    Decode & & \\
    Execute & valE $\leftarrow$ 0 + valC & valE $\leftarrow$ 0x0 + 0x80 = 0x80 \\
    Memory & & \\
    Write back & R[rB] $\leftarrow$ valE & R[\%rsp] $\leftarrow$ 0x80 \\
    PC update & PC $\leftarrow$ valP & PC $\leftarrow$ valP = 0x020 \\
\end{tabular}

\section*{4.14 - correct}
\begin{tabular}{l l l}
    Stage & Generic & Specific \\
    & \texttt{popq} rA & \texttt{popq \%rax} \\
    \hline
    Fetch & icode:ifun $\leftarrow$ $M_1$[PC] & icode:ifun $\leftarrow$ $M_1$[0x02C] = 0xB:0x0 \\
    & rA:rB $\leftarrow$ $M_1$[PC+1] & rA:rB $\leftarrow$ $M_1$[0x02D] = 0x0:0xF \\
    & valP $\leftarrow$ PC + 2 & valP $\leftarrow$ 0x02C + 0x2 = 0x02E \\
    Decode & valA $\leftarrow$ R[\%rsp] & valA $\leftarrow$ R[\%rsp] = 120 \\
    & valB $\leftarrow$ R[\%rsp] & valB $\leftarrow$ R[\%rsp] = 120 \\
    Execute & valE $\leftarrow$ valB + 8 & valE $\leftarrow$ 120 + 8 = 128 \\
    Memory & valM $\leftarrow$ $M_8$[valA] & valM $\leftarrow$ $M_8$[120] = 9 \\
    Write back & R[\%rsp] $\leftarrow$ valE & R[\%rsp] $\leftarrow$ 128 \\
    & R[rA] $\leftarrow$ valM & R[\%rax] $\leftarrow$ 9 \\
    PC update & PC $\leftarrow$ valP & PC $\leftarrow$ 0x02E \\
\end{tabular}

\section*{4.43 - correct}
\begin{tabular}{l c c c c c}
    Cause & Name & Instruction frequency & Condition frequency & Bubbles & Product \\
    \hline
    Load/use  & $lp$ & 0.25 & 0.20 & 1 & 0.05 \\
    Mispredict & $mp$ & 0.20 & 0.35 & 2 & 0.14 \\
    Return & $rp$ & 0.02 & 1.00 & 3 & 0.06 \\
    Total penalty & & & & & 0.25 \\
\end{tabular}

\vspace{1em}\noindent
Thus, the CPI is $1 + 0.25 = 1.25$.

\section*{Bryant Chapter 6}

\section*{6.12 - correct}
Set index: 3 bits since 8 sets $\rightarrow \log_2 8 = 3$ bits \\
Block offset: 2 bits since 4-byte block size $\rightarrow \log_2 4 = 2$ bits \\
Tag: 13 - (3 + 2) = 8 bits \\
\begin{tabular}{|c|c|c|c|c|c|c|c|c|c|c|c|c|}
    \hline
    CT & CT & CT & CT & CT & CT & CT & CT & CI & CI & CI & CO & CO \\
    \hline
    12 & 11 & 10 & 9 & 8 & 7 & 6 & 5 & 4 & 3 & 2 & 1 & 0 \\
    \hline
\end{tabular}

\section*{6.17 - correct}

\subsection*{A.}
Given the following properties:
\begin{itemize}
    \item \texttt{sizeof(int) = 4}
    \item \texttt{src} is at address 0
    \item \texttt{dst} is at address 16
    \item One L1 cache - direct mapped, write-through, write-allocate, 8-byte blocks
    \item Cache has 16 bytes total, is initially empty
    \item Accesses to \texttt{src} and \texttt{dst} are the only sources of hits and misses
\end{itemize}

\noindent
Set index: Direct mapped $\rightarrow E = 1$, so $\log_2 \frac{16}{8} = 1$ bit \\
\noindent
Block offset: 8-byte blocks $\rightarrow \log_2 8 = 3$ bits \\
\noindent
Memory addresses will have the form of: \\

\begin{center}
    \begin{tabular}{|c|c|c|c|c|c|}
        \hline
        \ldots & CT & CI & CO & CO & CO \\
        \hline
        \ldots & 4 & 3 & 2 & 1 & 0 \\
        \hline
    \end{tabular}
\end{center}

\noindent Thus,
\begin{center}
    \begin{tabular}{|l|c|c|c|c|c|}
        \hline
        Array element & Decimal address & Binary address & Tag & Index & Offset \\
        \hline
        \texttt{src[0][0]} & 0 & \ldots 00000 & 0 & 0 & 000 \\
        \texttt{src[0][1]} & 4 & \ldots 00100 & 0 & 0 & 100 \\
        \texttt{src[1][0]} & 8 & \ldots 01000 & 0 & 1 & 000 \\
        \texttt{src[1][1]} & 12 & \ldots 01100 & 0 & 1 & 100 \\
        \texttt{dst[0][0]} & 16 & \ldots 10000 & 1 & 0 & 000 \\
        \texttt{dst[0][1]} & 20 & \ldots 10100 & 1 & 0 & 100 \\
        \texttt{dst[1][0]} & 24 & \ldots 11000 & 1 & 1 & 000 \\
        \texttt{dst[1][1]} & 28 & \ldots 11100 & 1 & 1 & 100 \\
        \hline
    \end{tabular}
\end{center}

\noindent So the cache will have this kind of history:
\begin{center}
    \begin{tabular}{|c|c|}
        \hline
        Set & Tag + Array access \\
        \hline
        0 & 1: dst[0][1] \\
        & \sout{0: src[0][1]} \\
        & \sout{1: dst[0][0]} \\
        & \sout{0: src[0][0]} \\
        \hline
        1 & 1: dst[1][1] \\
        & \sout{0: src[1][1]} \textbf{hit} \\
        & \sout{0: src[1][0]} \\
        & \sout{1: dst[1][0]} \\
        \hline
    \end{tabular}
\end{center}

\begin{center}
    \begin{tabular}{l c c c l c c}
        & \texttt{dst} array & & \quad & & \texttt{src} array & \\
        \hline
        & Col. 0 & Col. 1 & & & Col. 0 & Col. 1 \\
        \hline
        Row 0 & m & m & & Row 0 & m & m \\
        Row 1 & m & m & & Row 1 & m & h \\
    \end{tabular}
\end{center}

\subsection*{B.}
Now, with a cache size of 32 bytes, the cache will have 4 sets. \\
Set index: 2 bits since $\log_2 \frac{32}{8} = 2$ bits \\
Block offset: Still 3 bits since 8-byte blocks $\rightarrow \log_2 8 = 3$ bits \\
Thus, memory addresses will have the form of: \\
\begin{center}
    \begin{tabular}{|c|c|c|c|c|c|c|}
        \hline
        \ldots & CT & CI & CI & CO & CO & CO \\
        \hline
        \ldots & 6 & 5 & 4 & 3 & 2 & 1 \\
        \hline
    \end{tabular}
\end{center}

\noindent Therefore,
\begin{center}
    \begin{tabular}{|l|c|c|c|c|c|}
        \hline
        Array element & Decimal address & Binary address & Tag & Index & Offset \\
        \hline
        \texttt{src[0][0]} & 0 & \ldots 000000 & 0 & 00 & 000 \\
        \texttt{src[0][1]} & 4 & \ldots 000100 & 0 & 00 & 100 \\
        \texttt{src[1][0]} & 8 & \ldots 001000 & 0 & 01 & 000 \\
        \texttt{src[1][1]} & 12 & \ldots 001100 & 0 & 01 & 100 \\
        \texttt{dst[0][0]} & 16 & \ldots 010000 & 0 & 10 & 000 \\
        \texttt{dst[0][1]} & 20 & \ldots 010100 & 0 & 10 & 100 \\
        \texttt{dst[1][0]} & 24 & \ldots 011000 & 0 & 11 & 000 \\
        \texttt{dst[1][1]} & 28 & \ldots 011100 & 0 & 11 & 100 \\
        \hline
    \end{tabular}
\end{center}

\noindent So the cache will have this kind of history:
\begin{center}
    \begin{tabular}{|c|c|}
        \hline
        Set & Tag + Array access \\
        \hline
        0 & 0: src[0][1] \textbf{hit} \\
        & 0: src[0][0] \\
        \hline
        1 & 0: src[1][1] \textbf{hit} \\

        & 0: src[1][0] \\
        \hline
        2 & 0: dst[0][1] \textbf{hit} \\

        & 0: dst[0][0] \\
        \hline
        3 & 0: dst[1][1] \textbf{hit} \\

        & 0: dst[1][0] \\
        \hline
    \end{tabular}
\end{center}

\begin{center}
    \begin{tabular}{l c c c l c c}
        & \texttt{dst} array & & \quad & & \texttt{src} array & \\
        \hline
        & Col. 0 & Col. 1 & & & Col. 0 & Col. 1 \\
        \hline
        Row 0 & m & h & & Row 0 & m & h \\
        Row 1 & m & h & & Row 1 & m & h \\
    \end{tabular}
\end{center}

\section*{6.18 - correct}

\subsection*{A.}
Each array element is read twice, so $32 \times 32 \times 2 = 2048$ read operations.

\subsection*{B.}
Each block consists of two structs, so each access will alternate in hit/miss.
Therefore, there should be 1024 cache misses.

\subsection*{C.}
The miss rate is 50\%

\end{document}