\documentclass{article}
\usepackage{amsmath,amssymb,amsthm}
\usepackage{fancyhdr}
\usepackage{enumerate}
\usepackage{graphicx}
\usepackage{caption}
\usepackage{float}
\usepackage{fontspec}
\usepackage{xcolor}
\usepackage{listings}

\setmonofont{JetBrainsMonoNerdFont-Regular.ttf}[Path=./]

% Catppuccin Macchiato color scheme
\definecolor{ctp-macchiato-rosewater}{HTML}{f4dbd6}
\definecolor{ctp-macchiato-flamingo}{HTML}{f0c6c6}
\definecolor{ctp-macchiato-pink}{HTML}{f5bde6}
\definecolor{ctp-macchiato-mauve}{HTML}{c6a0f6}
\definecolor{ctp-macchiato-red}{HTML}{ed8796}
\definecolor{ctp-macchiato-maroon}{HTML}{ee99a0}
\definecolor{ctp-macchiato-peach}{HTML}{f5a97f}
\definecolor{ctp-macchiato-yellow}{HTML}{eed49f}
\definecolor{ctp-macchiato-green}{HTML}{a6da95}
\definecolor{ctp-macchiato-teal}{HTML}{8bd5ca}
\definecolor{ctp-macchiato-sky}{HTML}{91d7e3}
\definecolor{ctp-macchiato-sapphire}{HTML}{7dc4e4}
\definecolor{ctp-macchiato-blue}{HTML}{8aadf4}
\definecolor{ctp-macchiato-lavender}{HTML}{b7bdf8}
\definecolor{ctp-macchiato-text}{HTML}{cad3f5}
\definecolor{ctp-macchiato-subtext1}{HTML}{b8c0e0}
\definecolor{ctp-macchiato-subtext0}{HTML}{a5adcb}
\definecolor{ctp-macchiato-overlay2}{HTML}{939ab7}
\definecolor{ctp-macchiato-overlay1}{HTML}{8087a2}
\definecolor{ctp-macchiato-overlay0}{HTML}{6e738d}
\definecolor{ctp-macchiato-surface2}{HTML}{5b6078}
\definecolor{ctp-macchiato-surface1}{HTML}{494d64}
\definecolor{ctp-macchiato-surface0}{HTML}{363a4f}
\definecolor{ctp-macchiato-base}{HTML}{24273a}
\definecolor{ctp-macchiato-mantle}{HTML}{1e2030}
\definecolor{ctp-macchiato-crust}{HTML}{181926}

\lstset{
    basicstyle=\ttfamily\color{ctp-macchiato-text},
    backgroundcolor=\color{ctp-macchiato-base},
    commentstyle=\color{ctp-macchiato-surface1},
    keywordstyle=\color{ctp-macchiato-mauve},
    stringstyle=\color{ctp-macchiato-green},
    identifierstyle=\color{ctp-macchiato-blue},
    frame=single,
    rulecolor=\color{ctp-macchiato-base},
    breaklines=true,
    breakatwhitespace=true,
    breakautoindent=true,
    showspaces=false,
    showstringspaces=false,
    showtabs=false,
    tabsize=4,
    captionpos=b,
    belowskip=1em,
    belowcaptionskip=2em
}

\lstdefinelanguage{myassembly}{
    basicstyle=\ttfamily\color{ctp-macchiato-text},
    backgroundcolor=\color{ctp-macchiato-base},
    commentstyle=\color{ctp-macchiato-surface1},
    keywordstyle=\color{ctp-macchiato-mauve},
    stringstyle=\color{ctp-macchiato-green},
    identifierstyle=\color{ctp-macchiato-blue},
    frame=single,
    rulecolor=\color{ctp-macchiato-base},
    breaklines=true,
    breakatwhitespace=true,
    breakautoindent=true,
    showspaces=false,
    showstringspaces=false,
    showtabs=false,
    tabsize=4,
    captionpos=b,
    belowskip=1em,
    belowcaptionskip=2em,
    % escapeinside={%}{)},
    morecomment=[l]{;},
    morestring=[b]",
    morekeywords={mov, add, sub, mul, div, cmp, jmp, je, jne, jg, jge, jl, jle, push, pop, call, ret, lea, xor, or, and, not, test},
    sensitive=false,
    upquote=true,
}

\pagestyle{fancy}
\fancyhf{}
\lhead{3rd Homework - CSCE 312 503}
\rhead{Kevin Lei}
\renewcommand{\headrulewidth}{0.4pt}
\renewcommand{\arraystretch}{1.2}

\begin{document}

\section*{Bryant Chapter 3}

\section*{3.5}

\begin{lstlisting}[language=C]
void decode1(long *xp, long *yp, long *zp) {
    // xp in %rdi, yp in %rsi, zp in %rdx
    long x = *xp;
    long y = *yp;
    long z = *zp;

    *yp = x;
    *zp = y;
    *xp = z;

    return;
}
\end{lstlisting}

\section*{3.6}

\texttt{\%rbx} holds $p$ and \texttt{\%rdx} holds $q$.

\vspace{1em}    

\noindent \begin{tabular}{|l|l|}
    \hline
    \textbf{Instruction} & \textbf{Result} \\
    \hline
    \texttt{leaq 9(\%rdx), \%rax} & \texttt{\%rax} $\leftarrow \; q + 9$ \\
    \texttt{leaq (\%rdx, \%rbx), \%rax} & \texttt{\%rax} $\leftarrow \; p + q$ \\
    \texttt{leaq (\%rdx, \%rbx, 3), \%rax} & \texttt{\%rax} $\leftarrow \; 3p + q$ \\
    \texttt{leaq 2(\%rbx, \%rbx, 7), \%rax} & \texttt{\%rax} $\leftarrow \; 7p + p + 2$ \\
    \texttt{leaq 0xE(,\%rdx, 3), \%rax} & \texttt{\%rax} $\leftarrow \; 3q + 14$ \\
    \texttt{leaq 6(\%rbx, \%rdx, 7), \%rax} & \texttt{\%rax} $\leftarrow \; 7q + p + 6$ \\
    \hline
\end{tabular}

\section*{3.7}

\begin{lstlisting}[language=myassembly]
; short scale3(short x, short y, short z)
; x in %rdi, y in %rsi, z in %rdx
scale3:
    leaq (%rsi, %rsi, 9), %rbx       ; rbx = 9y + y = 10y
    leaq (%rbx, %rdx), %rbx          ; rbx = z + rbx = z + 10y
    leaq (%rbx, %rdi, %rsi), %rbx    ; rbx = xy + rbx = xy + z + 10y
    ret
\end{lstlisting}

\begin{lstlisting}[language=C]
short scale3(short x, short y, short z) {
    short t = x * y + z + 10 * y;
    return t;
}
\end{lstlisting}

\section*{3.8}
Initial values:

\vspace{1em}

\noindent \begin{tabular}{|l l|l l|}
    \hline
    \textbf{Address} & \textbf{Value} & \textbf{Register} & \textbf{Value} \\
    \hline
    \texttt{0x100} & 0xFF & \texttt{\%rax} & 0x100 \\
    \texttt{0x108} & 0xAB & \texttt{\%rcx} & 0x1 \\
    \texttt{0x110} & 0x13 & \texttt{\%rdx} & 0x3 \\
    \texttt{0x118} & 0x11 & & \\
    \hline
\end{tabular}

\vspace{1em}

\noindent Effects of the instructions:

\vspace{1em}

\noindent \begin{tabular}{|l l l|}
    \hline
    \textbf{Instruction} & \textbf{Destination} & \textbf{Value} \\
    \hline
    \texttt{addq \%rcx, (\%rax)} & $M[0x100]$ & 0x1 \\
    \texttt{subq \%rdx, 8(\%rax)} & $M[0x108]$ & 0xA8 \\
    \texttt{imulq \$16, (\%rax, \%rdx, 8)} & $M[0x118]$ & 0x110 \\
    \texttt{incq 16(\%rax)} & $M[0x110]$ & 0x14 \\
    \texttt{decq \%rcx} & \texttt{\%rcx} & 0x0 \\
    \texttt{subq \%rdx, \%rax} & \texttt{\%rax} & 0xFD \\ 
    \hline
\end{tabular}

\section*{3.18}

\begin{lstlisting}[language=C]
short test(short x, short y, short z) {
    short val = y + z - x;
    if (z > 5) {
        if (y > 2)
            val = x / z;
        else
            val = x / y;
    } else if (z < 3)
        val = z / y;
    return val;
}
\end{lstlisting}

\section*{3.20}

\begin{lstlisting}[language=myassembly]
; x in %rdi
arith:
    leaq 15(%rdi), %rbx    ; %rbx <- x + 15
    testq %rdi, %rdi       ; test if x is zero
    cmovns %rdi, %rbx      ; if x > 0, %rbx <- x
    sarq $4, %rbx          ; shift %rbx right by 4 bits
    ret
\end{lstlisting}

\noindent The \texttt{OP} operation is division. We use the bias of 15 to round the result of division to the nearest integer, since we are dividing by 16.

\section*{3.24}

\begin{lstlisting}[language=C]
short loop_while(short a, short b) {
    short result = 0;
    while (a > b) {
        result = result + a * b;
        a = a - 1;
    }
    return result;
}
\end{lstlisting}

\begin{lstlisting}[language=myassembly]
; short loop_while(short a, short b)
; a in %rdi, b in %rsi
loop_while:
    movl $0, %eax
    jmp .L2
.L3:
    leaq (,%rsi,%rdi), %rdx
    addq %rdx, %rax
    subq $1, %rdi
.L2:
    cmpq %rsi, %rdi
    jg .L3
    rep; ret
\end{lstlisting}

\newpage
\section*{3.25}

\begin{lstlisting}[language=C]
long loop_while2(long a, long b) {
long result = b;
while (b > 0) {
    result = result * a;
    b = b - 1;
}
return result;
}
\end{lstlisting}

\begin{lstlisting}[language=myassembly]
; a in %rdi, b in %rsi
loop_while2:
    testq %rsi, %rsi
    jle .L8
    movq %rsi, %rax
.L7:
    imulq %rdi, %rax
    subq $1, %rsi
    testq %rsi, %rsi
    jg .L7
    rep; ret
.L8:
    movq %rsi, %rax
    ret
\end{lstlisting}

\section*{3.32}

\begin{table}[H]
\centering
\resizebox{\textwidth}{!}{%
\begin{tabular}{|c|c|l|c|c|c|c|c|l|}
\hline
\textbf{Label} & \textbf{PC} & \textbf{Instruction} & \textbf{\%rdi} & \textbf{\%rsi} & \textbf{\%rax} & \textbf{\%rsp} & \textbf{* \%rsp} & \textbf{Description} \\ \hline
M1 & \texttt{0x400560} & \texttt{callq} & 10 & -- & -- & \texttt{0x7fffffffe820} & -- & Call \texttt{first(10)} \\ \hline
F1 & \texttt{0x400548} & \texttt{lea} & 10 & -- & -- & \texttt{0x7fffffffe818} & \texttt{0x400565} & Load \texttt{x+1} into \%rdi \\ \hline
F2 & \texttt{0x40054c} & \texttt{sub} & 10 & 11 & -- & \texttt{0x7fffffffe818} & \texttt{0x400565} & Subtract 1 from \%rdi \\ \hline
F3 & \texttt{0x400550} & \texttt{callq} & 9 & 11 & -- & \texttt{0x7fffffffe818} & \texttt{0x400565} & Call \texttt{last(x-1, x+1)} \\ \hline
L1 & \texttt{0x400540} & \texttt{mov} & 9 & 11 & -- & \texttt{0x7fffffffe810} & \texttt{0x400555} & Move \texttt{x} to \%rax \\ \hline
L2 & \texttt{0x400543} & \texttt{imul} & 9 & 11 & 9 & \texttt{0x7fffffffe810} & \texttt{0x400555} & Multiply \texttt{x} by 11 \\ \hline
L3 & \texttt{0x400547} & \texttt{ret} & 9 & 11 & 99 & \texttt{0x7fffffffe810} & \texttt{0x400555} & Return 99 from \texttt{last} \\ \hline
F4 & \texttt{0x400555} & \texttt{repz retq} & 9 & 11 & 99 & \texttt{0x7fffffffe818} & \texttt{0x400565} & Return 99 from \texttt{first} \\ \hline
M2 & \texttt{0x400565} & \texttt{mov} & 10 & 11 & 99 & \texttt{0x7fffffffe820} & -- & Move 99 to \%rdx \\ \hline
\end{tabular}%
}
\end{table}

\newpage
\section*{3.35}

\begin{lstlisting}[language=C]
long rfun(unsigned long x) {
    if (x == 0)
        return 0;
    unsigned long nx = x >> 2;
    long rv rfun(nx);
    return x + rv;
}
\end{lstlisting}

\begin{lstlisting}[language=myassembly]
; long rfun(unsigned long x)
; x in %rdi
rfun:
    pushq %rbx          ; save %rbx
    movq %rdi, %rbx     ; use callee saved register %rbx to store x
    movl $0, %eax       ; set return value to 0
    testq %rdi, %rdi    ; if x == 0
    je .L2              ; go to .L2 if x == 0
    shrq $2, %rdi       ; shift x right by 2 bits
    call rfun           ; call rfun with x >> 2
    addq %rbx, %rax     ; add x to return value
.L2:
    popq %rbx           ; restore %rbx
    ret
\end{lstlisting}

\noindent The \texttt{\%rbx} register is used to store the value of \texttt{x} in the current function call.

\section*{3.37}

\begin{table}[H]
\centering
\resizebox{\textwidth}{!}{%
\begin{tabular}{|l|l|l|l|}
\hline
\textbf{Expression} & \textbf{Type} & \textbf{Value} & \textbf{Assembly code} \\ \hline
\texttt{P[1]} & short & $M[x_p + 2]$ & \texttt{movw 2(\%rdx), \%ax} \\ \hline
\texttt{P + 3 + i} & short* & $x_p + 6 + 2i$ & \texttt{leaq 6(\%rdx, \%rcx, 2), \%rax} \\ \hline
\texttt{P[i * 6 - 5]} & short & $M[x_p + 12i - 10]$ & \texttt{movw -10(\%rdx, \%rcx, 12), \%ax} \\ \hline
\texttt{P[2]} & short & $M[x_p + 4]$ & \texttt{movw 4(\%rdx), \%ax} \\ \hline
\texttt{\&P[i + 2]} & short* & $x_p + 2i + 4$ & \texttt{leaq 4(\%rdx, \%rcx, 2), \%rax} \\ \hline
\end{tabular}%
}
\end{table}

\newpage
\section*{3.38}

\begin{lstlisting}[language=myassembly]
; long sum_element(long i, long j)
; i in %rdi, j in %rsi

sum_element:
    leaq 0(,%rdi,8), %rdx         ; %rdx <- 8i
    subq %rdi, %rdx               ; %rdx <- 7i
    addq %rsi, %rdx               ; %rdx <- 7i + j
    leaq (%rsi, %rsi, 4), %rax    ; %rax <- 5j
    addq %rax, %rdi               ; %rdi <- 5j + i
    movq Q(,%rdi,8), %rax         ; %rax <- M[x_q + 8(5j + i)]
    addq P(,%rdx,8), %rax         ; %rax <- M[x_p + 8(7i + j)] + M[x_q + 8(5j + i)]
    ret
\end{lstlisting}

\noindent The index for \texttt{P} is $7i + j$ and the index for \texttt{Q} is $5j + i$. Therefore, \texttt{M} = $6$ and \texttt{N} = $8$.

\section*{3.41}

\begin{lstlisting}[language=C]
void st_init(struct test *st) {
    st->s.y = st->s.x;
    st->p = &(st->s.y);
    st->next = st;
}
\end{lstlisting}

\vspace{1em}
\noindent\textbf{Offsets:}
\begin{itemize}
    \item \texttt{p}: 8 bytes
    \item \texttt{s.x}: 2 bytes
    \item \texttt{s.y}: 2 bytes
    \item \texttt{next}: 8 bytes
\end{itemize}

\end{document}
