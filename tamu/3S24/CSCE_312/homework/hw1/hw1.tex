\documentclass{article}
\usepackage{amsmath,amssymb,amsthm}
\usepackage{fancyhdr}
\usepackage{enumerate}

\pagestyle{fancy}
\fancyhf{}
\lhead{1st Homework - CSCE 312 503}
\rhead{Kevin Lei}
\renewcommand{\headrulewidth}{0.4pt}
\renewcommand{\arraystretch}{1.2}

\begin{document}

\section*{Question 1}
The way we solve this is using Amdahl's Law.
The formula for Amdahl's Law is: 
$$ S = \frac{1}{(1 - \alpha) + \frac{\alpha}{k}} $$
Where $S$ is the speedup, $\alpha$ is the fraction of the program that can be parallelized, and $k$ is the perfomance uplift.
\\ \\
In this situation, our $\alpha$ is 0.5, since half of the processing time is spent on floating-point instructions.
Since the enhancement has allowed the machine to run floating-point instructions five times faster, our $k$ is 5.
Putting all this into our equation, we get
\begin{align*}
S &= \frac{1}{(1 - 0.5) + \frac{0.5}{5}} \\
&= \frac{1}{0.5 + 0.1} \\
&= \frac{1}{0.6} \\
&\approx 1.67 \; .
\end{align*}
Thus, our overall speedup is 1.67.

\section*{Question 2}

\subsection*{a)}
Here we need to figure out which one of the optimizations has the greater effect on the performance.
If we only make the divide operation 3 times faster, we have an overall speedup of
\begin{align*}
    S &= \frac{1}{(1 - 0.2) + \frac{0.2}{3}} \\
    &= \frac{1}{0.8 + \frac{1}{15}} \\
    &\approx 1.15 \; \text{times.}
\end{align*}
If we only make the multiple operation 8 times faster, we have an overall speedup of
\begin{align*}
    S &= \frac{1}{(1 - 0.6) + \frac{0.6}{8}} \\
    &= \frac{1}{0.4 + \frac{3}{40}} \\
    &\approx 2.11 \; \text{times.}
\end{align*}
Therefore, we cannot meet management's goal of achieving a 5 times performance uplift by only making either the divide or multiply operation faster, 
as the maximum speedup we can achieve is 2.11 times with only the multiply operation being 8 times faster.

\subsection*{b)}
Amdahl's law works by diving the old time by the new time to get the speedup, which makes intuitive sense.
If the old time is 1, the new time can be found by removing the parts that are sped up and adding the speed adjusted parts back in.
Thus we can generalize Amdahl's law to multiple speedups:
$$ S = \frac{1}{(1 - \alpha - \beta) + \frac{\alpha}{k_1} + \frac{\beta}{k_2}} $$
Where $\alpha$ and $\beta$ are the fractions of the program that can be parallelized, and $k_1$ and $k_2$ are their respective perfomance uplifts.
Now we can use this to find the overall speedup if we make both improvements.
\begin{align*}
    S &= \frac{1}{(1 - 0.2 - 0.6) + \frac{0.2}{3} + \frac{0.6}{8}} \\
    &= \frac{1}{0.2 + \frac{1}{15} + \frac{3}{40}} \\
    &\approx 2.93 \; .
\end{align*}
Now, our speedup relative to the old machine is 2.93 times.

\newpage
\section*{Textbook Problems:}
\textit{Digital Design, 2nd Ed, by Frank Vahid, Wiley publication, 2010}

\subsection*{1.8}

\subsection*{1.15}

\subsection*{1.18}

\subsection*{1.22}

\subsection*{1.32}

\subsection*{2.12}

\subsection*{2.18}

\subsection*{2.24}

\subsection*{2.28}

\subsection*{2.33}

\subsection*{2.39}

\subsection*{2.52}

\subsection*{2.58}

\subsection*{2.66}

\subsection*{2.70}

\subsection*{2.71}

\subsection*{2.75}

\subsection*{3.2}

\subsection*{3.4}

\subsection*{3.10}

\subsection*{3.12}

\subsection*{3.21}

\subsection*{3.30}

\subsection*{3.40}

\subsection*{3.43}

\end{document}
