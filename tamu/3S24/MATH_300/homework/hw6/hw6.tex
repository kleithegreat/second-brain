\documentclass{article}
\usepackage{amsmath,amssymb,amsthm}
\usepackage{fancyhdr}
\usepackage{enumerate}

\pagestyle{fancy}
\fancyhf{}
\lhead{6th Homework - MATH 300 904}
\rhead{Kevin Lei}
\renewcommand{\headrulewidth}{0.4pt}
\renewcommand{\arraystretch}{1.2}

\begin{document}

\section*{Question 1}

\begin{enumerate}[(a)]
    \item $\{\frac{\pi}{2}, \frac{3 \pi}{2}, \frac{5 \pi}{2}, \frac{7 \pi}{2}, \ldots\}$
    \item $\{\ldots, -13, -8, -3, 2, 7, 12, 17, \ldots\}$
\end{enumerate}

\section*{Question 2}

\begin{enumerate}[(a)]
    \item False
    \item True
    \item True
    \item True
    \item True
    \item True
    \item False
\end{enumerate}

\section*{Question 3}
\begin{proof}
    Let $A$ and $B$ be sets.
    We want to prove that $A = B$, which means that $A \subseteq B$ and $B \subseteq A$.
    First, assume that $x$ is an element of $A$.
    Then, by the definition of $A$, we can write $x$ as $x = 4k + 1$ for some integer $k$.
    We can then do the following manipulations:
    \begin{align*}
        x &= 4k + 1 \\
        &= 4k + 1 - 8 + 8 \\
        &= 4k + 8 - 7 \\
        &= 4(k + 2) - 7.
    \end{align*}
    Since $k + 2$ is an integer, $x$ is by definition an element of $B$.
    Therefore, $A \subseteq B$.
    Similarly, assume that $x$ is an element of $B$.
    Then, by the definition of $B$, we can write $x$ as $x = 4j - 7$ for some integer $j$.
    We can then do the following manipulations:
    \begin{align*}
        x &= 4j - 7 \\
        &= 4j - 7 + 8 - 8 \\
        &= 4j - 8 + 1 \\
        &= 4(j - 2) + 1.
    \end{align*}
    Since $j - 2$ is an integer, $x$ is by definition an element of $A$.
    Therefore, $B \subseteq A$.
    Since $A \subseteq B$ and $B \subseteq A$, we have proven that $A = B$.
\end{proof}

\section*{Question 4}
\begin{enumerate}[(a)]
    \item
    \begin{align*}(A \cup \overline{B}) \cap C &= (\{a, b, \{2\}\} \cup \{1, \{2\}, a \}) \cap \{1, \{2\}, c\} \\
        &= \{1, \{2\}, a, b\} \cap \{1, \{2\}, c\} \\
        &= \{1, 2\}
    \end{align*}
    \item
    \begin{align*}
        A \cup (\overline{B} \cap C) &= \{a, b, \{2\}\} \cup (\{1, \{2\}, a\} \cap \{1, \{2\}, c\}) \\
        &= \{a, b, \{2\}\} \cup \{1, \{2\}\} \\
        &= \{1, \{2\}, a, b\}
    \end{align*}
\end{enumerate}

\section*{Question 5}
\renewcommand*{\proofname}{Disproof}
\begin{enumerate}[(a)]
    \item
        \begin{proof}
            Let $A = \{1, 2, 3\}$, $B = \{2, 3, 4\}$, and $C = \{2, 3, 5\}$.
            Then, $A \cap B = \{2, 3\} = A \cap C$.
            However, $B \neq C$, as $4 \in B$ but $4 \notin C$, and $5 \in C$ but $5 \notin B$.
            Therefore, the statement is false.
        \end{proof}
    \item
        \begin{proof}
            Let $A = \{1, 2, 3\}$, $B = \{2, 3, 4\}$, and $C = \{1, 2, 3, 4\}$.
            Then, $A \setminus B = \{1\} = A \setminus C$.
            However, $B \neq C$, as $4 \in B$ but $4 \notin C$.
            Therefore, the statement is false.
        \end{proof}
\end{enumerate}

\section*{Question 6}
\renewcommand*{\proofname}{Proof}
\begin{proof}
    Let $A$ and $B$ be sets. We want to show that $A \subseteq B$ if and only if $\overline{B} \subseteq \overline{A}$.
    First, $A \subseteq B$ is equivalent to saying $x \in A \rightarrow x \in B$.
    This is logically equivalent to its contrapositive, that being $x \notin B \rightarrow x \notin A$, which is the definition of $\overline{B} \subseteq \overline{A}$.
    Since we have this sequence of logically equivalent statements, we have shown that $A \subseteq B \leftrightarrow \overline{B} \subseteq \overline{A}$.
\end{proof}

\section*{Question 7}
\begin{enumerate}[(a)]
    \item
        \begin{proof}
            Let $A$ and $B$ be sets.
            We want to show that $\overline{A \cap B} = \overline{A} \cup \overline{B}$.
            Assuming that $x \in \overline{A \cap B}$, we have $x \notin A \cap B$.
            By the definition of the set intersection, this means that $x \notin A$ or $x \notin B$.
            By the definition of the set complement, this means that $x \in \overline{A}$ or $x \in \overline{B}$.
            This can be rewritten using the set union as $x \in \overline{A} \cup \overline{B}$.
            Since these are all bidirectional logical equivalences, we have shown that $\overline{A \cap B} = \overline{A} \cup \overline{B}$.
        \end{proof}
    \item
        \begin{proof}
            Let $A$ be a set.
            We want to show that $A \cap \emptyset = \emptyset$.
            Seeking a contradiction, assume that $A \cap \emptyset \neq \emptyset$.
            Suppose that $x \in A \cap \emptyset$.
            By definition of the set intersection, we have $x \in A$ and $x \in \emptyset$.
            However, the empty set has no elements, so $x$ cannot be in $\emptyset$.
            Thus, we have reached a contradiction, and our assumption that $A \cap \emptyset \neq \emptyset$ must be false.
            Therefore, $A \cap \emptyset = \emptyset$.
        \end{proof}
\end{enumerate}

\end{document}