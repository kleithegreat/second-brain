\documentclass{article}
\usepackage{amsmath,amssymb,amsthm}
\usepackage{fancyhdr}
\usepackage{enumerate}

\pagestyle{fancy}
\fancyhf{}
\lhead{6th Homework - MATH 300 904}
\rhead{Kevin Lei}
\renewcommand{\headrulewidth}{0.4pt}
\renewcommand{\arraystretch}{1.2}

\begin{document}

\section*{Question 1}

\begin{enumerate}[(a)]
    \item $\{\frac{\pi}{2}, \frac{3 \pi}{2}, \frac{5 \pi}{2}, \frac{7 \pi}{2}, \ldots\}$
    \item $\{\ldots, -13, -8, -3, 2, 7, 12, 17, \ldots\}$
\end{enumerate}

\section*{Question 2}

\begin{enumerate}[(a)]
    \item False
    \item True
    \item True
    \item True
    \item True
    \item True
    \item False
\end{enumerate}

\section*{Question 3}
\begin{proof}
    Let $A$ and $B$ be sets.
    We want to prove that $A = B$, which means that $A \subseteq B$ and $B \subseteq A$.
    First, assume that $x$ is an element of $A$.
    Then, by the definition of $A$, we can write $x$ as $x = 4k + 1$ for some integer $k$.
    We can then do the following manipulations:
    \begin{align*}
        x &= 4k + 1 \\
        &= 4k + 1 - 8 + 8 \\
        &= 4k + 8 - 7 \\
        &= 4(k + 2) - 7.
    \end{align*}
    Since $k + 2$ is an integer, $x$ is by definition an element of $B$.
    Therefore, $A \subseteq B$.
    Similarly, assume that $x$ is an element of $B$.
    Then, by the definition of $B$, we can write $x$ as $x = 4j - 7$ for some integer $j$.
    We can then do the following manipulations:
    \begin{align*}
        x &= 4j - 7 \\
        &= 4j - 7 + 8 - 8 \\
        &= 4j - 8 + 1 \\
        &= 4(j - 2) + 1.
    \end{align*}
    Since $j - 2$ is an integer, $x$ is by definition an element of $A$.
    Therefore, $B \subseteq A$.
    Since $A \subseteq B$ and $B \subseteq A$, we have proven that $A = B$.
\end{proof}

\section*{Question 4}
\begin{enumerate}[(a)]
    \item
    \begin{align*}(A \cup \overline{B}) \cap C &= (\{a, b, \{2\}\} \cup \{1, \{2\}, a \}) \cap \{1, \{2\}, c\} \\
        &= \{1, \{2\}, a, b\} \cap \{1, \{2\}, c\} \\
        &= \{1, 2\}
    \end{align*}
    \item
    \begin{align*}
        A \cup (\overline{B} \cap C) &= \{a, b, \{2\}\} \cup (\{1, \{2\}, a\} \cap \{1, \{2\}, c\}) \\
        &= \{a, b, \{2\}\} \cup \{1, \{2\}\} \\
        &= \{1, \{2\}, a, b\}
    \end{align*}
\end{enumerate}

\end{document}