\documentclass{article}
\usepackage{amsmath,amssymb,amsthm}
\usepackage{fancyhdr}
\usepackage{enumerate}

\pagestyle{fancy}
\fancyhf{}
\lhead{3rd Homework - MATH 300 904}
\rhead{Kevin Lei}
\renewcommand{\headrulewidth}{0.4pt}
\renewcommand{\arraystretch}{1.2}

\begin{document}

\section*{Question 1}
\begin{proof}
    Let $n$ be an integer.
    We want to show that if $n^2 - 3n + 5$ is even, then n is odd.
    We will prove this by proving the contrapositive.
    That is, we will show that if $n$ is even, then $n^2 - 3n + 5$ is odd.
    By definition, if $n$ is even, then $$n = 2k$$ for some integer $k$.
    We can substitute this into the expression for $n^2 - 3n + 5$ to get
    \begin{align*}
        n^2 - 3n + 5 &= (2k)^2 - 3(2k) + 5 \\
        &= 4k^2 - 6k + 5 \\
        &= 2(2k^2 - 3k + 2) + 1.
    \end{align*}
    Since $2k^2 - 3k + 2$ is an integer, we have shown that $n^2 - 3n + 5$ is odd.
    Thus, the contrapositive is true, and this also shows that the original statement is true.
\end{proof}

\section*{Question 2}
\begin{proof}
    Let $n$ be an integer.
    We want to show that $n$ is odd if and only if $n + 2$ is odd.
    To prove this, we will need to show that if $n$ is odd, then $n + 2$ is odd, and if $n + 2$ is odd, then $n$ is odd.
    By definition, if $n$ is odd, then $$n = 2k + 1$$ for some integer $k$.
    We can substitute this into the expression for $n + 2$ to get
    \begin{align*}
        n + 2 &= (2k + 1) + 2 \\
        &= 2k + 3 \\
        &= 2(k + 1) + 1.
    \end{align*}
    Since $k + 1$ is an integer, we have shown that $n + 2$ is odd when $n$ is odd.
    Now, we will show that if $n + 2$ is odd, then $n$ is odd using the contrapositive.
    That is, we will show that if $n$ is even, then $n + 2$ is even.
    By definition, if $n$ is even, then $$n = 2k$$ for some integer $k$.
    We can substitute this into the expression for $n + 2$ to get
    \begin{align*}
        n + 2 &= 2k + 2 \\
        &= 2(k + 1).
    \end{align*}
    Since $k + 1$ is an integer, we have shown that $n + 2$ is even when $n$ is even.
    Thus, the contrapositive is true, and this also shows that $n + 2$ is odd when $n$ is odd.
    Now we have proven both directions of the biconditional, so we have shown that $n$ is odd if and only if $n + 2$ is odd.
\end{proof}

\section*{Question 3}
\begin{proof}
    Let $m$ and $n$ be integers.
    We want to show that $mn$ is even if and only if $m$ is even or $n$ is even.
    We will need to prove both ways of the biconditional, so we will first show that 
    if $mn$ is even, then $m$ is even or $n$ is even.
    To do this, we will prove the contrapositive, which is that if $m$ is odd and $n$ is odd, then $mn$ is odd.
    By definition, if $m$ and $n$ are odd, then we can write $m = 2k + 1$ and $n = 2l + 1$ for some integers $k$ and $l$.
    We can substitute these into the expression for $mn$ to get
    \begin{align*}
        mn &= (2k + 1)(2l + 1) \\
        &= 4kl + 2k + 2l + 1 \\
        &= 2(2kl + k + l) + 1.
    \end{align*}
    Since $2kl + k + l$ is an integer, we have shown that $mn$ is odd when $m$ and $n$ are odd,
    and thus the contrapositive is true.
    Since the contrapositive is true, this also shows that if $mn$ is even, then $m$ is even or $n$ is even.
    Now, proving the other direction of the biconditional, we will show that if $m$ is even or $n$ is even, then $mn$ is even.
    By definition, if $m$ is even, then we can write $m = 2k$ for some integer $k$.
    We can substitute this into $mn$ to get $mn = 2kn$, which is even.
    Without loss of generality, we can also show that if $n$ is even, then $mn$ is even.
    Thus, both directions of the biconditional have been proven, and we have shown that $mn$ is even if and only if $m$ is even or $n$ is even.
\end{proof}

\section*{Question 4}
\begin{proof}
    We want to show that if integers $a$ and $b$ are odd, then $4 \nmid (a^2 + b^2)$.
    Seeking a contradiction, suppose that there exist odd integers $a$ and $b$ such that $4 \mid (a^2 + b^2)$.
    Since $a$ and $b$ are odd, we can write $a = 2m + 1$ and $b = 2n + 1$ for some integers $m$ and $n$.
    Additionally, since $4 \mid (a^2 + b^2)$, we can write $a^2 + b^2 = 4k$ for some integer $k$.
    Substituting these expressions into the equation, we get
    \begin{align*}
        (2m + 1)^2 + (2n + 1)^2 &= 4k \\
        4m^2 + 4m + 1 + 4n^2 + 4n + 1 &= 4k \\
        m^2 + m + n^2 + n + \frac{1}{2} &= k.
    \end{align*}
    We know that $k$ is an integer by the closure axioms, but $\frac{1}{2}$ is not reducible to an integer.
    Thus, we have reached a contradiction, and we have shown that if integers $a$ and $b$ are odd, then $4 \nmid (a^2 + b^2)$.
\end{proof}

\newpage
\section*{Question 5}
\begin{proof}
    We want to show that there do not exist integers $m$ and $n$ such that $8m + 26n = 1$.
    Seeking a contradiction, suppose that there exist integers $m$ and $n$ such that $8m + 26n = 1$.
    Then, we can write the following:
    \begin{align*}
        8m + 26n &= 1 \\
        2(4m + 13n) &= 1 \\
        4m + 13n &= \frac{1}{2}.
    \end{align*}
    We know that $4m + 13n$ is an integer by the closure axioms, but $\frac{1}{2}$ is not reducible an integer.
    Thus, we have reached a contradiction, and we have shown that there do not exist integers $m$ and $n$ such that $8m + 26n = 1$.
\end{proof}

\section*{Question 6}
\textbf{Lemma 1.} An integer $n$ is not divisible by 3 if and only if there exists an integer $k$ such that $n = 3k + 1$ or $n = 3k + 2$.
\begin{proof}
    Let $n$ be an integer.
    We want to show that if $3 \mid n^2$, then $3 \mid n$.
    We will prove this by proving the contrapositive, which is that if $3 \nmid n$, then $3 \nmid n^2$.
    By lemma 1, we know that if $3 \nmid n$, then we can write $n = 3k + 1$ or $n = 3k + 2$ for some integer $k$.
    In the case that $n = 3k + 1$, we can write $$n^2 = (3k + 1)^2 = 9k^2 + 6k + 1 = 3(3k^2 + 2k) + 1,$$ which is not divisible by 3.
    In the case that $n = 3k + 2$, we can write $$n^2 = (3k + 2)^2 = 9k^2 + 12k + 4 = 3(3k^2 + 4k + 1) + 1,$$ which is also not divisible by 3.
    Thus, the contrapositive is true, and this also shows that if $3 \mid n^2$, then $3 \mid n$.
\end{proof}

\section*{Question 7}
\begin{proof}
    We want to show that there do not exist integers $m$ and $n$ such that $m^2 = 4n + 3$.
    Seeking a contradiction, suppose that there does exist integers $m$ and $n$ such that $m^2 = 4n + 3$.
    Since $m^2$ can be written as $2(2n + 2) + 1$, we know that $m^2$ is odd.
    By corollary 2.2.5, $m$ is also odd.
    Then, we can write $m = 2k + 1$ for some integer $k$.
    Substituting this into the equation, we get
    \begin{align*}
        (2k + 1)^2 &= 4n + 3 \\
        4k^2 + 4k + 1 &= 4n + 3 \\
        4k^2 + 4k - 2 &= 4n \\
        2k^2 + 2k - \frac{1}{2} &= n.
    \end{align*}
    We know that $n$, $k^2$, and $k$ are all integers by the closure axioms, but $\frac{1}{2}$ is not reducible to an integer.
    Thus, we have reached a contradiction, and we have shown that there do not exist integers $m$ and $n$ such that $m^2 = 4n + 3$.
\end{proof}

\end{document}