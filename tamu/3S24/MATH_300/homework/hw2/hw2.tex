\documentclass{article}
\usepackage{amsmath,amssymb,amsthm}
\usepackage{fancyhdr}
\usepackage{enumerate}

\pagestyle{fancy}
\fancyhf{}
\lhead{2nd Homework - MATH 300 904}
\rhead{Kevin Lei}
\renewcommand{\headrulewidth}{0.4pt}
\renewcommand{\arraystretch}{1.2}

\begin{document}

\section*{Question 1}
\begin{enumerate}[(a)]
    \item
        \begin{tabular}{c|c|c|c}
            $x$ & $P(x)$ & $Q(x)$ & $P(x) \iff Q(x)$ \\
            \hline
            0 & F & F & T \\
            2 & F & F & T \\
            3 & T & F & F \\
            4 & T & T & T \\
            6 & T & T & T \\
        \end{tabular}
    \item The truth value for $\forall x \in \{0, 2, 3, 4, 6\} P(x) \iff Q(x)$ is false.
    \item The truth value for $\exists x \in \{0, 2, 3, 4, 6\} P(x) \iff Q(x)$ is true.
\end{enumerate}


\section*{Question 2}
\begin{enumerate}[(a)]
    \item 20 is a member of the set of the integer multiples of 4.
    \item 3.14 is a member of the set of the rational numbers or $\pi$ is a member of the set of the real numbers.
    \item There exists an integer $n$ in the set of natural numbers such that the square root of $n$ is not a member of the set of real numbers.
\end{enumerate}

\section*{Question 3}
\begin{enumerate}[(a)]
    \item For all integers $n$, $n^2 -4n +3 \geq 0$.
    \item There exists a rational number $x$ such that $x \geq 100$.
\end{enumerate}

\section*{Question 4}
\begin{enumerate}[(a)]
    \item There exists an Aggie who does not follow the Aggie Honor Code.
    \item All students either do not live on campus or is a math major.
    \item There exists an integer $m$ such that $m^2$ is even and $m^3 - 1$ is not divisible by 4.
\end{enumerate}

\section*{Question 5}
Converse: If $f$ is continuous at 0, then $f$ is a linear function. \\
Contrapositive: If $f$ is not continuous at 0, then $f$ is not a linear function. \\
Inverse: If $f$ is not a linear function, then $f$ is not continuous at 0. \\

\section*{Question 6}
\begin{enumerate}[(a)]
    \item For all real numbers $x$, there exists an integer $n$ such that $n$ is less than or equal to $x$ and $x$ is less than $n+1$.
    \item $(\exists x \in \mathbb{R}) (\forall n \in \mathbb{Z}) (n > x \geq n + 1)$
\end{enumerate}

\section*{Question 7}
\begin{enumerate}[(a)]
    \item If $x$ is a multiple of 6, then $x$ is even and is not a multiple of 4. \\
        \textbf{Negation:} There exists a multiple of 6 such that it is not even or it is a multiple of 4.
    \item If $x$ is an even integer, then $x^2$ is divisible by 4. \\
        \textbf{Negation:} There exists an integer $x$ such that $x$ is even and $x^2$ is not divisible by 4.
\end{enumerate}

\section*{Question 8}
\begin{proof}
    Let $n$ be an odd integer.
    By definition of the odd integers, there exists an integer $k$ such that $n = 2k + 1$.
    Then \begin{align*}
        n^2 + 1 &= (2k + 1)^2 + 1 \\
                &= 4k^2 + 4k + 2 \\
                &= 2(2k^2 + 2k + 1).
    \end{align*}
    Since the integers are closed under addition and multiplication, $2k^2 + 2k + 1$ is an integer.
    Therefore, $n^2 + 1 = 2(2k^2 + 2k + 1) = 2l$ for some integer $l$.
    The even integers are defined as integers that can be written as twice an arbitrary integer.
    Since $n^2 + 1$ is twice an arbitrary integer $l$, $n^2 + 1$ is an even integer.
\end{proof}

\end{document}
