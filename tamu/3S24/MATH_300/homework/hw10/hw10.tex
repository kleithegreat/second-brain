\documentclass{article}
\usepackage{amsmath,amssymb,amsthm}
\usepackage{fancyhdr}
\usepackage{enumerate}

\pagestyle{fancy}
\fancyhf{}
\lhead{10th Homework - MATH 300 904}
\rhead{Kevin Lei}
\renewcommand{\headrulewidth}{0.4pt}
\renewcommand{\arraystretch}{1.2}

\begin{document}

\section*{Question 1}
\begin{proof}
Let $A$ be a set. We want to show that $f: \mathcal{P}(A) \to \mathcal{P}(A)$ defined by $f(X) = \overline{X}$ is a bijection. 
First, we must show that $f$ is injective.
Assume that $X_1, X_2 \in \mathcal{P}(A)$ and $f(X_1) = f(X_2)$. 
Then,
\begin{align*}
    f(X_1) &= f(X_2) \\
    \overline{X_1} &= \overline{X_2} \\
    \overline{\overline{X_1}} &= \overline{\overline{X_2}} \\
    X_1 &= X_2.
\end{align*}
Thus, $f$ is injective.
Now, we must show that $f$ is surjective, or in other words, that $\text{Ran}(f) = \mathcal{P}(A)$.
The subset relation $\text{Ran}(f) \subseteq \mathcal{P}(A)$ is trivial.
To show that $\mathcal{P}(A) \subseteq \text{Ran}(f)$, let $Y \in \mathcal{P}(A)$.
Consider the set $\overline{Y}$.
Since $Y \in \mathcal{P}(A)$, $\overline{Y} \in \mathcal{P}(A)$.
Then, $f(\overline{Y}) = \overline{\overline{Y}} = Y$.
Thus, $f$ is surjective.
Since $f$ is both injective and surjective, $f$ is a bijective.
\end{proof}

\section*{Question 2}

\subsection*{Part a}
\begin{proof}
We want to show that the function $f: (-\infty, 1) \to \mathbb{R}$ defined by $f(x) = x^3$ is not bijective.
Seeking a contradiction, assume that $f$ is bijective.
Then, $f$ is surjective.
Consider the element $8 \in \mathbb{R}$.
Since $f$ is surjective, there exists an element $x \in (-\infty, 1)$ such that $f(x) = x^3 = 8$.
Thus,
\begin{align*}
    x^3 &= 8 \\
    x &= 2.
\end{align*}
However, $2 \notin (-\infty, 1)$, a contradiction.
Therefore, $f$ is not bijective.
\end{proof}

\subsection*{Part b}
\begin{proof}
We want to show that the function $D: \mathbb{R}[x] \to \mathbb{R}[x]$ defined by $D(f(x)) = f'(x)$ is not bijective.
Consider the elements $x, x + 1 \in \mathbb{R}[x]$.
Indeed, $\frac{d}{dx}(x) = 1 = \frac{d}{dx}(x + 1)$, but $x \neq x + 1$.
Therefore, $D$ is not injective, and thus not bijective.
\end{proof}

\subsection*{Part c}
\begin{proof}
We want to show that the function $s: \mathbb{N} \times \mathbb{N} \to \mathbb{N}$ defined by $s(m, n) = m + n$ is not bijective.
Consider the elements $(1, 1), (2, 0) \in \mathbb{N} \times \mathbb{N}$.
Indeed, $s(1, 1) = 2 = s(2, 0)$, but $(1, 1) \neq (2, 0)$.
Therefore, $s$ is not injective, and thus not bijective.
\end{proof}

\section*{Question 3}
\begin{proof}
Let $f: X \to Y$ and $g: Y \to Z$ be functions.
We want to show that if $g \circ f$ is injective, then $f$ is injective, but $g$ need not be injective.
Assume that $g \circ f$ is injective.
To show that $f$ is injective, assume that $x_1, x_2 \in X$ and $f(x_1) = f(x_2)$.
Then, $(g \circ f)(x_1) = (g \circ f)(x_2)$.
Since $g \circ f$ is injective, $x_1 = x_2$.
To show that $g$ need not be injective, consider the following counterexample.
Let $X = \{1, 2\}$, $Y = \{3, 4\}$, and $Z = \{5\}$.
Let the functions $f: X \to Y$ and $g: Y \to Z$ be defined by the graphs $G_f = \{(1, 3), (2, 4)\}$ and $G_g = \{(3, 5), (4, 5)\}$.
Then, the graph of $g \circ f$ is $G_{g \circ f} = \{(1, 5), (2, 5)\}$.
\end{proof}

\section*{Question 4}
\begin{enumerate}[(a)]
    \item $f(\{-3, 2, 7\}) = \{10, 5, 50\}$
    \item $f([-1, 3]) = [1, 10]$
    \item $f((-\infty, -2)) = (-3, \infty)$
\end{enumerate}

\section*{Question 5}

\subsection*{Part a}
\begin{proof}
    Let $f: X \to Y$ and $g: Y \to Z$ be invertible functions.
    Since $f$ and $g$ are invertible, they are also bijective.
    Since $f$ and $g$ are bijective, the composition $g \circ f$ is also bijective.
    Since $g \circ f$ is bijective, it is invertible.
\end{proof}

\subsection*{Part b}
\begin{proof}
    Let $f: X \to Y$ and $g: Y \to Z$ be invertible functions.
    Then,
    \begin{align*}
        (f^{-1} \circ g^{-1}) \circ (g \circ f) &= f^{-1} \circ (g^{-1} \circ g) \circ f \\
        &= f^{-1} \circ \text{id}_Y \circ f \\
        &= f^{-1} \circ f \\
        &= \text{id}_X.
    \end{align*}
    Similarly,
    \begin{align*}
        (g \circ f) \circ (f^{-1} \circ g^{-1}) &= g \circ (f \circ f^{-1}) \circ g^{-1} \\
        &= g \circ \text{id}_Y \circ g^{-1} \\
        &= g \circ g^{-1} \\
        &= \text{id}_Z.
    \end{align*}
    Thus, $(g \circ f)^{-1} = f^{-1} \circ g^{-1}$.
\end{proof}

\section*{Question 6}
\begin{proof}

\end{proof}

\section*{Question 7}
\begin{proof}

\end{proof}

\section*{Question 8}
\begin{proof}

\end{proof}

\end{document}