\documentclass{article}
\usepackage{amsmath,amssymb,amsthm}
\usepackage{fancyhdr}
\usepackage{enumerate}

\pagestyle{fancy}
\fancyhf{}
\lhead{11th Homework - MATH 300 904}
\rhead{Kevin Lei}
\renewcommand{\headrulewidth}{0.4pt}
\renewcommand{\arraystretch}{1.2}

\begin{document}

\section*{Question 1}
\begin{proof}
    Let $f: X \to Y$ and $A_1, A_2 \subseteq X$.
    Assume $y \in f[A_1 \cup A_2]$.
    Then,
    \begin{align*}
        y \in f[A_1 \cup A_2] &\iff y = f(x), x \in A_1 \cup A_2 \\
        &\iff y = f(x), x \in A_1 \lor x \in A_2 \\
        &\iff y \in f[A_1] \lor y \in f[A_2] \\
        &\iff y \in f[A_1] \cup f[A_2].
    \end{align*}
    Thus, $f[A_1 \cup A_2] = f[A_1] \cup f[A_2]$.
\end{proof}

\section*{Question 2}
$f^{-1}([-3, 5]) = [1009.5, 1013.5]$

\section*{Question 3}
$D^{-1}(\{4x^3\}) = \{x^4\}$

\section*{Question 4}
\begin{enumerate}
    \item $s^{-1}(\{4\}) = \{(0, 4), (1, 3), (2, 2), (3, 1), (4, 0)\}$
    \item $s^{-1}(\{1\}) = \{(0, 1), (1, 0)\}$
\end{enumerate}

\section*{Question 5}
\begin{proof}
    Let $f: X \to Y$ and $B_1, B_2 \subseteq Y$.
    Assume $x \in f^{-1}[B_1 \cup B_2]$.
    Then,
    \begin{align*}
        x \in f^{-1}[B_1 \cup B_2] &\iff f(x) \in B_1 \cup B_2 \\
        &\iff f(x) \in B_1 \lor f(x) \in B_2 \\
        &\iff x \in f^{-1}[B_1] \lor x \in f^{-1}[B_2] \\
        &\iff x \in f^{-1}[B_1] \cup f^{-1}[B_2].
    \end{align*}
    Thus, $f^{-1}[B_1 \cup B_2] = f^{-1}[B_1] \cup f^{-1}[B_2]$.
\end{proof}

\section*{Question 6}

\subsection*{Part a}
\begin{proof}
    Let $f: X \to Y$ and $A \subseteq X$.
    Assume $x \in A$.
    Then $f(x) \in f[A]$.
    Since $x \in f^{-1}[f[A]]$, $A \subseteq f^{-1}[f[A]]$.
\end{proof}

\subsection*{Part b}
\begin{proof}
    Let $f: X \to Y$ and $A \subseteq X$.
    Assume that $f$ is injective.
    From part a, we know that $A \subseteq f^{-1}[f[A]]$.
    Now let $x \in f^{-1}[f[A]]$.
    Then, $f(x) \in f[A]$.
    Thus, there exists $a \in A$ such that $f(x) = f(a)$.
    Since $f$ is injective, $x = a$.
    Since $x = a$, $x \in A$.
    Therefore, $f^{-1}[f[A]] \subseteq A$, so $f^{-1}[f[A]] = A$.
\end{proof}

\section*{Question 7}

\subsection*{Part a}
\begin{align*}
    1207 &= 569 \cdot 2 + 69 \\
    569 &= 69 \cdot 8 + 17 \\
    69 &= 17 \cdot 4 + 1 \\
\end{align*}
Thus, $\gcd(1207, 569) = 1$.

\subsection*{Part b}
\begin{align*}
    \gcd(1207, 569) &= 1 \\
    &= 69 - 17 \cdot 4 \\
    &= 69 - (569 - 69 \cdot 8) \cdot 4 \\
    &= 69 - 569 \cdot 4 + 69 \cdot 32 \\
    &= 69 \cdot 33 - 569 \cdot 4 \\
    &= (1207 - 569 \cdot 2) \cdot 33 - 569 \cdot 4 \\
    &= 1207 \cdot 33 - 569 \cdot 66 - 569 \cdot 4 \\
    &= 1207 \cdot 33 - 569 \cdot 70.
\end{align*}
Thus, $x = 33$ and $y = -70$.

\end{document}