\documentclass{article}
\usepackage{amsmath,amssymb,amsthm}
\usepackage[hyphens,spaces,obeyspaces]{url}

\title{Math 300 Writing Assignment 1}
\author{Kevin Lei}
\date{February 2, 2020}

\begin{document}

\maketitle

The topic I chose to write about is the famous French mathematician Augustin-Louis Cauchy, 
and I also plan on discussing his contribution to complex analysis.
I decided to write about this because I enjoy learning about the biographical details of imporant figures, 
and I also wanted to do a deep dive into some pure math specialty.
As for why I chose Cauchy of all mathematicians, I wanted to write about someone with a very significant contribution to mathematics,
but not someone who is so well-known that I would have trouble finding a unique angle to write about.
I think for most people, Cauchy isn't the first name that comes to mind when they think of famous mathematicians, 
but he still almost signgle-handedly spearheaded the field of complex analysis.
As for why I chose complex analysis, I felt that it would be the most interesting and new topic (for my background) to write about.
I tried reading a real analysis textbook over winter break (Real Analysis by Jay Cummings), and even though it was written to be on the more accessible side, 
I still found it mind bending with all the epsilons and deltas and theorems (I barely made it past the first chapter). 
I have heard a lot of people describe real analysis as how difficult and intricate math can get, 
while complex analysis is how beautiful and elegant math can get, so I decided to give complex analysis a try.

Augustin-Louis Cauchy was born on August 21, 1789 in Paris, France.
His family was part of the French nobility, and his father was a high-ranking government official.
His family had to move around during his childhood due to the French Revolution, 
and he was educated at home by his father until he was 15. 
From his father's government position, Cauchy's family was close to Pierre-Simon Laplace as well as Joseph-Louis Lagrange. 
Cauchy went on to study at the best schools in France, and graduated from the École Polytechnique in 1807.
Before Cauchy's contributions, complex numbers were mostly just used to solve polynomial equations, 
and Euler had used Maclaurin series to establish his famous identity. 
Cauchy was the first to study functions of complex numbers in depth, 
and he introduced many of the most fundamental concepts in complex analysis that are used widely today. 
Outside of complex analysis, Cauchy's \textit{Cours d'Analyse} introduced rigor to calculus, 
where at the time many concepts in calculus were not well-defined and mostly intuitive, such as the concept of a limit.
By the end of his life, Cauchy had published over 800 papers and 5 books, second in productivity only to Euler.

\newpage

Complex analysis is the study of functions of complex numbers, which are functions of the class $f: \mathbb{C} \to \mathbb{C}$.
Complex functions are often denoted as $f(z)$, where $z$ is a complex number of the form $z = a + bi$.
A complex function that is differentiable on some open disk in the complex domain is called \textit{holomorphic} on that disk.
One of the most important results in complex analysis is Cauchy's integral theorem. 
In equation form, this is written as $$ \oint_C f(z) \; dz = 0, $$ where $C$ is a closed path in the complex domain.
This theorem states that if a function is holomorphic on some open disk, then the integral of that function over any closed path in that disk is zero. 
Intuitively, this theorem can be thought of as a statement on the ``balanced" behavior of holomorphic functions, 
where integrating over a closed path will always have opposite contributions totalling to zero. 
A few more concepts are necessary to discuss another one of Cauchy's most important contributions; residues and their applications (I won't talk about residues here due to word count).
A \textit{pole} of a function is a point where the function is not holomorphic due to the fact that it goes to infinity at that point, 
which is similar to a vertical asymptote for functions $\mathbb{R} \to \mathbb{R}$.
The \textit{Laurent series} is like a Taylor series expansion for complex functions, 
where the series is centered at a pole and has both positive and negative powers of $(z - z_0)$. 
In equation form, the Laurent series is written as $$ f(z) = \sum_{n = -\infty}^{\infty} a_n (z - z_0)^n. $$
Using these concepts, Cauchy defines a crucial concept in complex analysis, the \textit{residue} of a function at a pole, 
and this is used extensively in the \textit{residue theorem}.

\newpage

\section*{References}
\begin{itemize}
    \item Augustin-Louis Cauchy on Wikipedia - \url{https://en.wikipedia.org/wiki/Augustin-Louis_Cauchy}
    \item Holomorphic function on Wikipedia - \url{https://en.wikipedia.org/wiki/Holomorphic_function}
    \item Cauchy's integral theorem on Wikipedia - \url{https://en.wikipedia.org/wiki/Cauchy%27s_integral_theorem}
    \item Residue (complex analysis) on Wikipedia - \url{https://en.wikipedia.org/wiki/Residue_(complex_analysis)}
    \item Laurent series on Wikipedia - \url{https://en.wikipedia.org/wiki/Laurent_series}
    \item ChatGPT 4, Prompts:
    \begin{itemize}
        \item ``what are the most important/fundamental concepts in complex analysis"
        \item ``explain cauchys integral theorem simply; you can use jargon but be sure to explain. assume a vector calculus and linear algebra background."
        \item ``now explain residues. build up your explanation from basics and assume no prior understanding of complex analysis aside from what you just described. still, assume a vector calculus and linear algebra background, and feel free to use new jargon as long as you explain it."
    \end{itemize}
\end{itemize}

\end{document}
