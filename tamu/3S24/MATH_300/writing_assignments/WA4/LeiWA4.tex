\documentclass[12pt,letterpaper]{article}

\usepackage[margin=1in]{geometry}
\usepackage{amsmath}
\usepackage{amssymb}
\usepackage{amsthm}
\usepackage{graphicx}
\usepackage{hyperref}
\usepackage{setspace}

\usepackage[backend=biber, style=numeric]{biblatex}
\DeclareLanguageMapping{english}{english-apa}
\addbibresource{wa4.bib}
\defbibheading{bibliography}[\bibname]{}

\newtheorem{theorem}{Theorem}

\doublespacing

\title{My Dog Left a Residue at Every Pole}
\author{Cauchy and Complex Analysis}
\date{}

\begin{document}

\begin{titlepage}
\maketitle
\thispagestyle{empty}
\begin{center}
    Kevin Lei \\
    MATH 300 904
\end{center}
\begin{abstract}
    Augustin-Louis Cauchy was a prolific French mathematician whose work laid the foundations for the rigorous study of both real and complex analysis. 
    This paper explores Cauchy's life and his contributions to the field of complex analysis. 
    We discuss his background and early life, and trace his education and career as he became one of the preeminent mathematicians of his era. 
    Cauchy introduced new standards of rigor into calculus and analysis, and his work with complex functions opened up rich new areas of study. 
    We examine important results bearing his name, including the Cauchy Integral Theorem, Residue Theorem, and Cauchy-Schwarz Inequality.
\end{abstract}
\end{titlepage}

\newpage
\thispagestyle{empty}
\tableofcontents

\newpage
\setcounter{page}{1}

\section{Introduction}
Augustin-Louis Cauchy (1789-1857) was a French mathematician whose work fundamentally shaped the development of mathematical analysis in the 19th century and beyond. 
Born right before the French Revolution, Cauchy's early life was marked by the chaos and upheaval of this period. 
However, he found a safe space in the precise, logical world of mathematics. 
Cauchy went on to become a prolific mathematician, one of the most published of all time.
He is especially renowned for introducing a new level of mathematical rigor into the fields of real and complex analysis.

In this paper, we explore the life and work of Augustin-Louis Cauchy, with a particular focus on his contributions to complex analysis. 
Section 2 traces Cauchy's life story, from his tumultuous early years during the French Revolution to his career as a professor and researcher. 
In Section 3, we provide an overview of the key concepts in complex analysis that are needed to understand Cauchy's work in this area. 
Section 4 examines some of Cauchy's most celebrated results in complex analysis, including the Cauchy Integral Theorem, the Residue Theorem, and the Cauchy-Schwarz Inequality. 
Finally, we conclude in Section 5 by reflecting on Cauchy's enduring legacy and impact on the field of mathematics.

\section{The Life of Cauchy}
Augustin Louis Cauchy was born on August 21, 1789, in Paris, France.
His parents were Louis François Cauchy, a government official, and Marie-Madeleine Desestre, a housewife.
Cauchy's father was highly ranked among the French nobility at the time.
However, due to the French Revolution (which started only a month before the birth of Cauchy), the family's high and noble status of yesterday was no longer a guarantee of safety.
The revolution marked a hard time for the Cauchy family, as they had to flee Paris to escape the violence of the revolution.
Thus, they moved to Arcueil, a small village near Paris, where Cauchy spent his early years.
The family struggled during this time, as they were no longer able to rely on the wealth and status they once had.
Cauchy's father, Louis François described the situation as follows:
``We never have more than a half pound of bread and sometimes not
even that. This we supplement with the little supply of hard crackers and
rice that we are allotted. Otherwise, we are getting along quite well,
which is the important thing and which goes to show that human beings
can get by with little." \cite{Belhoste1991}
Little is known of the formative years of Cauchy, but it would later become clear that the instablity of his early life would influence his later work.
As a child, Cauchy was described as a ``timid and frail boy" who ``had no liking for sports and games." \cite{Belhoste1991}
He loved purposeful work, and found refuge in thought and quiet study.
The chaos and disorder of the revolution contrasted sharply with the order and precision of mathematics, and Cauchy's later work would reflect this desire for mathematical rigor.

After the death of Robespierre, the Cauchy family returned to Paris.
Cauchy's father regained power in the new government, and worked closely with Laplace and Lagrange, both senators and incredibly influential mathematicians.
This connection would later prove to be beneficial for Cauchy, as he was able to study under Lagrange and Laplace.
Under the suggestion of Laplace, Cauchy was admitted to the École Centrale du Panthéon, one of the best schools in Paris at the time.
Here, he completed his secondary education, and was admitted to the École Polytechnique, a prestigious school for mathematics and science.
Going there meant that Cauchy would no longer be following his family's footsteps in government, as both of his brothers did, since he would instead be studying engineering.
However, Cauchy's family was supportive of his decision, and he went on to be one of the best students at the École Polytechnique.

After graduating from the École Polytechnique in 1807, Cauchy began his career as a military engineer. 
He worked on several projects, including the construction of the Port of Cherbourg. 
However, he soon realized that his true passion was mathematics, and he left his engineering job to pursue a career in academia.
In 1816, Cauchy was appointed as a professor at the École Polytechnique, where he taught for many years. 
During this time, he made significant contributions to various branches of mathematics, including complex analysis.
He was also the first to rigorously define calculus, which is what is known now as real analysis.
For example, consider both the intuitive and rigorous definitions of continuity.
In differential calculus, continuity is typically introduced as a function that can be drawn without lifting the pen.
However, Cauchy defined continuity as follows:
A function $f: \mathbb{R} \to \mathbb{R}$ is continuous at a point $x_0$ if for every $\epsilon > 0$, there exists a $\delta > 0$ such that $|x - x_0| < \delta$ implies $|f(x) - f(x_0)| < \epsilon$.
This definition is much more rigorous than the intuitive definition, and it laid the foundation for modern real analysis.
He published numerous papers and books, which helped establish him as one of the leading mathematicians of his time.

Cauchy's work in complex analysis was particularly groundbreaking.
He introduced the concept of complex functions and developed the theory of complex integration. 
His work laid the foundation for much of modern complex analysis, and many important results in the field are named after him, such as the Cauchy-Riemann equations and the Cauchy integral formula.

In addition to his work in mathematics, Cauchy was also a devout Catholic and a royalist. 
He refused to take an oath of allegiance to the new government after the July Revolution of 1830, which led to him losing his teaching position at the École Polytechnique. 
He went into exile at Turin, Prague, where he continued his mathematical research.
Cauchy returned to France in 1838 and was elected to the French Academy of Sciences. 
He continued to teach and conduct research until his death on May 23, 1857, in Sceaux, France.

\section{Introduction to Complex Analysis}
Complex analysis is the study of functions of complex variables.
Compared to real analysis, complex analysis is often considered to be much more elegant and beautiful, due to the special properties of complex numbers and functions.
First, complex numbers are numbers of the form $z = a + bi$, where $a, b \in \mathbb{R}$ and $i^2 = -1$. \cite{WeissteinComplexNumber}
Thus, complex numbers have both a real and imaginary part, and they can be visualized as points in the complex plane, where the real part is the $x$-coordinate and the imaginary part is the $y$-coordinate.
It is often useful to think of complex numbers as vectors in the complex plane, or tuples $(a, b)$ in $\mathbb{R}^2$.
Complex functions are functions of the form $f: \mathbb{C} \to \mathbb{C}$, where $\mathbb{C}$ is the set of complex numbers.
Much of the elegance of complex analysis stems from the many special properties of these complex functions, which are not present in real functions.
For example, complex differentiability is quite different from functions of real variables.
Recall that complex numers can be thought of as pairs of real numbers.
Complex differentiability is defined as follows:

\begin{theorem}{Complex Differentiability}
    A complex function $f(z) = u(x, y) + iv(x, y)$ has a complex derivative if and only if its real and imaginary parts are continuous and satisfy the Cauchy-Riemann equations:
    \begin{align*}
        \frac{\partial u}{\partial x} &= \frac{\partial v}{\partial y} \\
        \frac{\partial u}{\partial y} &= -\frac{\partial v}{\partial x}
    \end{align*}
\end{theorem}

This is a much stronger condition than real differentiability, which results in many of the nice properties of complex functions.
For example, if a complex function is differentiable in the complex domain, then it is called \textit{holomorphic}, and thus infinitely differentiable. \cite{LangComplexAnalysis1998}
This is in contrast to real functions, where differentiability once does not imply differentiability infinitely many times.

Another important property of complex functions are their \textit{poles} and \textit{Laurent series} representations.
A pole of a complex function is a point where the function is not holomorphic due to it going to infinity.
This is analogous to the concept of a vertical asymptote in real functions.
Recall that real functions can be represented as Taylor series, which are power series centered at a point.
However, this power series representation is not possible when the function has a vertical asymptote.
In complex functions, the Laurent series is a generalization of the Taylor series that allows for poles, and it is defined as:
$$ f(z) = \sum_{n = -\infty}^{\infty} a_n (z - z_0)^n $$
where $z_0$ is the center of the series.
Notice here how the series includes negative powers of $(z - z_0)$, which allows for better understanding of poles in the function. \cite{LangComplexAnalysis1998}

\section{Cauchy's Legacy in Complex Analysis}
Here we will discuss some of the most important results in complex analysis that are named after Cauchy.

\subsection{Cauchy's Integral Theorem}
One of Cauchy's most famous results in complex analysis is the Cauchy's Integral Theorem.
This theorem states that if $f(z)$ is a holomorphic function in a simply connected domain $D$, then the integral of $f(z)$ over any closed curve in $D$ is zero.
In mathematical terms, this can be written as:
$$ \oint_C f(z) dz = 0 $$
where $C$ is a closed curve in $D$.

To break this down, a simply connected domain is a domain in the complex plane that does not contain any holes.
A simply connected domain has no holes in the sense of no ``islands" or ``lakes" in the domain that would prevent forming a single boundary around the domain.
Another way to think about this would be that any domain that can be continuously shrunk to a point without leaving the domain is simply connected.
For example, suppose we have a domain in the complex plane that looks like a circle.
This would be a simply connected domain.
However, if we have a domain in the shape of a donut, this would not be simply connected, as there is a hole in the middle of the domain, and we cannot continuously shrink the domain to a point without leaving the domain.
Next, a closed curve is a curve that starts and ends at the same point.
This is analagous to a closed line integral in vector calculus.

The upshot of Cauchy's Integral Theorem is that it allows us to compute integrals of holomorphic functions over closed curves without having to evaluate the integral directly.
This result is incredibly powerful and has many applications not only in complex analysis but also in fields like physics and engineering.

A simple proof of Cauchy's Integral Theorem can be given using Green's Theorem.
\begin{proof}
    Let $f = u + iv$ be a holomorphic function in a simply connected domain $D$.
    We can break up the differential $dz$ into its real and imaginary parts:
    $$ dz = dx + i dy $$
    Then, the integral of $f(z)$ over a closed curve $C$ can be written as:
    \begin{align*}
        \oint_C f(z) dz &= \oint_C (u + iv)(dx + i dy) \\
        &= \oint_C (u dx - v dy) + i \oint_C (v dx + u dy).
    \end{align*}
    Now, this looks like a direct application of Green's Theorem:
    $$ \oint_C P dx + Q dy = \iint_D \left( \frac{\partial Q}{\partial x} - \frac{\partial P}{\partial y} \right) dA $$
    where $P = u$ and $Q = v$, resulting in:
    $$ \oint_C u dx - v dy = \iint_D \left( - \frac{\partial v}{\partial x} - \frac{\partial u}{\partial y} \right) dx dy $$
    $$ \oint_C v dx + u dy = \iint_D \left( \frac{\partial u}{\partial x} - \frac{\partial v}{\partial y} \right) dx dy .$$
    Since $f$ is holomorphic, it satisfies the Cauchy-Riemann equations, or in other words:
    $$ \frac{\partial u}{\partial x} = \frac{\partial v}{\partial y} \quad \text{and} \quad \frac{\partial u}{\partial y} = -\frac{\partial v}{\partial x}. $$
    Thus, both integrands are zero, so we have:
    $$ \oint_C f(z) dz = 0. $$
\end{proof}

\subsection{Residue Theorem}
In complex analysis, residues are an important concept when discussing complex integration.
Intuitively, one can think of a residue as something that measures how a complex function behaves around a singularity or pole. 
More precisely, the residue of a function $f(z)$ at a pole $z_0$ is the unique complex number $a_{-1}$ such that $f(z)$ can be written as:
$$f(z) = \frac{a_{-1}}{z-z_0} + a_0 + a_1(z-z_0) + a_2(z-z_0)^2 + \cdots$$
near $z_0$. 
The residue is the coefficient $a_{-1}$ of the term $(z-z_0)^{-1}$ in this Laurent series expansion.

The Residue Theorem, due to Cauchy, relates the integral of a meromorphic function (a function holomorphic except at a set of isolated poles) around a closed curve to the sum of the residues at the poles enclosed by the curve. 
In simpler, terms, this just means that instead of doing an annoying integral, we can just add up the residues of the poles inside the curve.
Mathematically, if $f$ is a meromorphic function with poles at $z_1, z_2, \ldots, z_n$ inside a positively oriented simple closed curve $\gamma$, then the Residue Theorem states:
$$\oint_\gamma f(z) , dz = 2\pi i \sum_{k=1}^n \text{Res}(f, z_k)$$
where $\text{Res}(f, z_k)$ denotes the residue of $f$ at the pole $z_k$.

\subsection{Cauchy-Schwarz Inequality (Complex Case)}
The Cauchy-Schwarz inequality is another significant result attributed to Cauchy, although he proved it only for real vector spaces. 
The inequality was later extended to complex vector spaces by Hermann Schwarz. 
In the complex case, the Cauchy-Schwarz inequality states that for any two complex vectors $\mathbf{x} = (x_1, \ldots, x_n)$ and $\mathbf{y} = (y_1, \ldots, y_n)$, we have:
$$\left| \sum_{k=1}^n x_k \overline{y_k} \right|^2 \leq \left(\sum_{k=1}^n |x_k|^2\right) \left(\sum_{k=1}^n |y_k|^2\right)$$
where $\overline{y_k}$ denotes the complex conjugate of $y_k$. \cite{BerkeleyNotes}

The inequality also leads to useful bounds in harmonic analysis, quantum mechanics, and information theory. 
Like many other results initially proven by Cauchy for real numbers, the extension of the Cauchy-Schwarz inequality to the complex domain has greatly expanded its reach and applicability.

\section{Conclusion}
Augustin-Louis Cauchy's life and work had a profound impact on the development of modern mathematics, particularly in the field of complex analysis. 
Despite facing personal and professional challenges, he remained dedicated to his passion for mathematics and made significant contributions to the field of complex analysis today.
Cauchy's approach to analysis set a new standard for mathematical rigor, which helped establish many of the fundamental concepts and theorems in complex analysis. 
As we reflect on Cauchy's life and work, we are reminded of the beauty and elegance of mathematics. 
Cauchy's contributions to complex analysis have indeed been profound and are still very relevant to mathematics today.

\section{References}
\printbibliography

% \appendix
% \section{Appendix}

\end{document}