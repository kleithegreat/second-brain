\documentclass[12pt,letterpaper]{article}

\usepackage[margin=1in]{geometry}
\usepackage{amsmath}
\usepackage{amssymb}
\usepackage{amsthm}
\usepackage{graphicx}
\usepackage{hyperref}
\usepackage{setspace}

\usepackage[backend=biber, style=numeric]{biblatex}
\DeclareLanguageMapping{english}{english-apa}
\addbibresource{wa4.bib}
\defbibheading{bibliography}[\bibname]{}

\newtheorem{theorem}{Theorem}

\doublespacing

\title{My Dog Left a Residue at Every Pole}
\author{Cauchy and Complex Analysis}
\date{}

\begin{document}

\begin{titlepage}
\maketitle
\thispagestyle{empty}
\begin{center}
    Kevin Lei \\
    MATH 300 904
\end{center}
\begin{abstract}

\end{abstract}
\end{titlepage}

\newpage
\thispagestyle{empty}
\tableofcontents

\newpage
\setcounter{page}{1}

\section{Introduction}

\section{The Life of Cauchy}
Augustin Louis Cauchy was born on August 21, 1789, in Paris, France.
His parents were Louis François Cauchy, a government official, and Marie-Madeleine Desestre, a housewife.
Cauchy's father was highly ranked among the French nobility at the time.
However, due to the French Revolution (which started only a month before the birth of Cauchy), the family's high and noble status of yesterday was no longer a guarantee of safety.
The revolution marked a hard time for the Cauchy family, as they had to flee Paris to escape the violence of the revolution.
Thus, they moved to Arcueil, a small village near Paris, where Cauchy spent his early years.
The family struggled during this time, as they were no longer able to rely on the wealth and status they once had.
Cauchy's father, Louis François described the situation as follows:
``We never have more than a half pound of bread and sometimes not
even that. This we supplement with the little supply of hard crackers and
rice that we are allotted. Otherwise, we are getting along quite well,
which is the important thing and which goes to show that human beings
can get by with little." \cite{Belhoste1991}
Little is known of the formative years of Cauchy, but it would later become clear that the instablity of his early life would influence his later work.
As a child, Cauchy was described as a ``timid and frail boy" who ``had no liking for sports and games." \cite{Belhoste1991}
He loved purposeful work, and found refuge in thought and quiet study.
The chaos and disorder of the revolution contrasted sharply with the order and precision of mathematics, and Cauchy's later work would reflect this desire for mathematical rigor.

After the death of Robespierre, the Cauchy family returned to Paris.
Cauchy's father regained power in the new government, and worked closely with Laplace and Lagrange, both senators and incredibly influential mathematicians.
This connection would later prove to be beneficial for Cauchy, as he was able to study under Lagrange and Laplace.
Under the suggestion of Laplace, Cauchy was admitted to the École Centrale du Panthéon, one of the best schools in Paris at the time.
Here, he completed his secondary education, and was admitted to the École Polytechnique, a prestigious school for mathematics and science.
Going there meant that Cauchy would no longer be following his family's footsteps in government, as both of his brothers did, since he would instead be studying engineering.
However, Cauchy's family was supportive of his decision, and he went on to be one of the best students at the École Polytechnique.

After graduating from the École Polytechnique in 1807, Cauchy began his career as a military engineer. 
He worked on several projects, including the construction of the Port of Cherbourg. 
However, he soon realized that his true passion was mathematics, and he left his engineering job to pursue a career in academia.
In 1816, Cauchy was appointed as a professor at the École Polytechnique, where he taught for many years. 
During this time, he made significant contributions to various branches of mathematics, including complex analysis.
He was also the first to rigorously define calculus, which is what is known now as real analysis.
For example, consider both the intuitive and rigorous definitions of continuity.
In differential calculus, continuity is typically introduced as a function that can be drawn without lifting the pen.
However, Cauchy defined continuity as follows:
A function $f: \mathbb{R} \to \mathbb{R}$ is continuous at a point $x_0$ if for every $\epsilon > 0$, there exists a $\delta > 0$ such that $|x - x_0| < \delta$ implies $|f(x) - f(x_0)| < \epsilon$.
This definition is much more rigorous than the intuitive definition, and it laid the foundation for modern real analysis.
He published numerous papers and books, which helped establish him as one of the leading mathematicians of his time.

Cauchy's work in complex analysis was particularly groundbreaking.
He introduced the concept of complex functions and developed the theory of complex integration. 
His work laid the foundation for much of modern complex analysis, and many important results in the field are named after him, such as the Cauchy-Riemann equations and the Cauchy integral formula.

In addition to his work in mathematics, Cauchy was also a devout Catholic and a royalist. 
He refused to take an oath of allegiance to the new government after the July Revolution of 1830, which led to him losing his teaching position at the École Polytechnique. 
He went into exile at Turin, Prague, where he continued his mathematical research.
Cauchy returned to France in 1838 and was elected to the French Academy of Sciences. 
He continued to teach and conduct research until his death on May 23, 1857, in Sceaux, France.

\section{Introduction to Complex Analysis}
Complex analysis is the study of functions of complex variables.
Compared to real analysis, complex analysis is often considered to be much more elegant and beautiful, due to the special properties of complex numbers and functions.
First, complex numbers are numbers of the form $z = a + bi$, where $a, b \in \mathbb{R}$ and $i^2 = -1$.
Thus, complex numbers have both a real and imaginary part, and they can be visualized as points in the complex plane, where the real part is the $x$-coordinate and the imaginary part is the $y$-coordinate.
It is often useful to think of complex numbers as vectors in the complex plane, or tuples $(a, b)$ in $\mathbb{R}^2$.
Complex functions are functions of the form $f: \mathbb{C} \to \mathbb{C}$, where $\mathbb{C}$ is the set of complex numbers.
Much of the elegance of complex analysis stems from the many special properties of these complex functions, which are not present in real functions.
For example, complex differentiability is quite different from functions of real variables.
Recall that complex numers can be thought of as pairs of real numbers.
Complex differentiability is defined as follows:

\begin{theorem}{Complex Differentiability}
    A complex function $f(z) = u(x, y) + iv(x, y)$ has a complex derivative if and only if its real and imaginary parts are continuous and satisfy the Cauchy-Riemann equations:
    \begin{align*}
        \frac{\partial u}{\partial x} &= \frac{\partial v}{\partial y} \\
        \frac{\partial u}{\partial y} &= -\frac{\partial v}{\partial x}
    \end{align*}
\end{theorem}

This is a much stronger condition than real differentiability, which results in many of the nice properties of complex functions.
For example, if a complex function is differentiable on some open disk in the complex domain, then it is called \textit{holomorphic}, and thus infinitely differentiable.

\section{Cauchy's Legacy in Complex Analysis}

\section{Conclusion}

\section{References}
\printbibliography

% \appendix
% \section{Appendix}

\end{document}