\documentclass{article}

\usepackage{graphicx}
\usepackage{amsmath}
\usepackage{hyperref}

\title{PA1: Stacks}
\author{Kevin Lei}
\date{February 6, 2024}

\begin{document}

\maketitle

\section{Introduction}

This programming assignment focuses on three different implementations of the stack data structure. 
The stack consists of an ordered collection of items where the addition and removal of items is done at the same end.
Since the stack is just an abstract data type, we have multiple ways to implement it.
In this programming assignment, the first two implementations are based on dynamically allocated arrays. 
Both start with an initial capacity of 1, but one linearly increases the capacity by 10 each time the array is full, while the other doubles the capacity each time the array is full.
The third implementation is based on a singly linked list.
In this report, we will discuss how these three implementations should behave and perform, and then compare the actual measured performance of these implementations.

\section{Theoretical Analysis}

The main operation we are interested in for this discussion is the push operation. 


\section{Experimental Results}



\end{document}
