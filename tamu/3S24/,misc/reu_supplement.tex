\documentclass{article}
\usepackage[utf8]{inputenc}
\usepackage{amsmath}
\usepackage{amssymb}
\usepackage{amsthm}

\begin{document}

\noindent Prompt: 
Choose one concept in mathematics that interests you. 
What is it? Why do you find it interesting? What work have you done with it? What more would you like to learn about it?

\vspace{5mm}

Althought I have not studied higher level mathematics yet, I find the more fundamental field of linear algebra very interesting.
Not only is it a powerful tool for other fields of math, physics, and engineering, but it also has a lot of interesting concepts and theorems which I find innately elegant.
In my fall semester of sophomore year, I took MATH 304 with Dr. Paouris, and I very much enjoyed building up the theory of linear algebra from the ground up.
Although the beginning of the course mostly consisted of solving linear systems, it was an important foundation for the more abstract concepts that followed.
For example, we introduced the concept of a determinant as a test for invertibility, and later we connected it to matrix rank, and how matricies are maps from \(\mathbb{R}^n\) to \(\mathbb{R}^m\).
By the end of the calss, two of my favorite concepts I learned were the rank-nullity theorem and the fundamental theorem of linear algebra.
Both theorems tie together seemingly disparate but fundametally related concepts, and I found it very satisfying to see how they all fit together.

I had fun with the homework in MATH 304, and I tried to make my answers as rigorous as possible, even though the course was supposed to be more expository and intuitive than proof-based.
I found that doing my homework rigorously helped me understand the material better and ultimately realize these deeper connections that my peers might have missed.
This also lead me to Dr. Paouris's office hours a lot, and I found that I really enjoyed discussing the material with him.
For example, one of our homework problems required showing an inequality for an inner product involving an integral.
I had the intuitive idea of how to do it, but I was having trouble making it rigorous, so Dr. Paouris showed me an epsilon-delta argument that satisfied this.

Beyond MATH 304, I would like to learn more about linear algebra itself as well as its applications, mostly in pure mathematics.
I intend on taking MATH 432, which is a continuation of MATH 304, which goes over Jordan canonical form and goes deeper on eigenvalues.
Although I have read about topics in linear algebra such as the spectral theorem and singular value decomposition, I would like to learn more about them in a more rigorous setting.
Looking forward, I am eager to dive deeper into the world of linear algebra through the NSF REU program. This opportunity is not just a step towards mastering advanced mathematical concepts; it is a gateway to exploring the endless possibilities that linear algebra opens in pure mathematics and beyond. I anticipate that participating in the REU program will not only solidify my understanding of linear algebra but also inspire new directions for my academic and research pursuits.

\end{document}
