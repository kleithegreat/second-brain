\documentclass{article}
\usepackage{amsmath,amssymb,amsthm,latexsym,paralist}
\usepackage{fancyhdr}

\theoremstyle{definition}
\newtheorem{problem}{Problem}
\newtheorem*{solution}{Solution}
\newtheorem*{resources}{Resources}

\newcommand{\name}[2]{\noindent\textbf{Name: #1}\hfill \textbf{UIN: #2}
  \newcommand{\myName}{#1}
  \newcommand{\myUIN}{#2}
}

\newcommand{\honor}{\noindent On my honor, as an Aggie, I have neither
  given nor received any unauthorized aid on any portion of the
  academic work included in this assignment. Furthermore, I have
  disclosed all resources (people, books, web sites, etc.) that have
  been used to answer this homework. \\[2ex]
 \textbf{Electronic signature: \underline{ Kevin Lei } } } % <= type your full name here
 
\newcommand{\checklist}{\noindent\textbf{Checklist:}
\begin{compactitem}[$\Box$] 
\item Did you type in your name and UIN? 
\item Did you disclose all resources that you have used? \\
(This includes all people, books, websites, etc.\ that you have consulted)
\item Did you sign that you followed the Aggie Honor Code? 
\item Did you solve all problems? 
\item Did you submit both the .tex and .pdf files of your homework to each correct link 
on Canvas? 
\end{compactitem}
}

\newcommand{\problemset}[1]{\begin{center}\textbf{Problem Set #1}\end{center}}
\newcommand{\duedate}[1]{\begin{quote}\textbf{Due dates:} Electronic
    submission of \textsl{yourLastName-yourFirstName-hw7.tex} and 
    \textsl{yourLastName-yourFirstName-hw7.pdf} files of this homework is due on
    \textbf{#1} on \texttt{https://canvas.tamu.edu}. You will see two separate links
    to turn in the .tex file and the .pdf file separately. Please do not archive or compress the files.  
    \textbf{If any of the two files are missing, you will receive zero points for this homework.}
    Your files must contain your first and last names and UIN in the given spaces and 
    the electronic signature (your full name) correctly.\end{quote} }

\newcommand{\N}{\mathbf{N}}
\newcommand{\R}{\mathbf{R}}
\newcommand{\Z}{\mathbf{Z}}

\fancyhead[L]{\myName}
\fancyhead[R]{\myUIN}
\pagestyle{fancy}

\begin{document}
\begin{center}
{\large
CSCE 222 Discrete Structures for Computing -- Fall 2023\\[.5ex]
Hyunyoung Lee\\}
\end{center}
\problemset{7}
\duedate{Monday, 11/20/2023 before 11:59 p.m.}
\name{ Kevin Lei }{ 432009232 } % <= type your full name and UIN here
% Omit the parentheses surrounding name and UIN.
% Your name should include your first and last names. 
% Your name and UIN that you type in here are propagated by LaTeX 
% to the header part of each page on the PDF output automatically.

\begin{resources} (All people, books, articles, web pages, etc.\ that
  have been consulted when producing your answers to this homework)
\end{resources}
\honor

\bigskip

\noindent
Total $100+10$ (bonus) points.  \textit{Explanation will be about 90\% of the grade 
for each problem.}

\bigskip

\noindent
The intended formatting is that this first page is a cover page and each 
problem solved on a new page. You only need to fill in your solution between 
the \verb|\begin{solution}| and \verb|\end{solution}| environment.  
Please do not change this overall formatting.

\vfill
\checklist

\newpage
\begin{problem} (20 points) Section 13.1, Exercise 13.4. Explain your reasoning 
carefully, including (but not limited to) why you set up your generating function 
in the way you do.
\end{problem}
\begin{solution}
We know that albert must receive an even amount of cards that is at least 8 and at most 14.
To represent this with a generating function, we can use the following:
$$ A(z) = z^8 + z^{10} + z^{12} + z^{14} $$
Here, the coefficient of $z^k$ represents the number of ways to give Albert $k$ cards.
The same applies for Bella and Clara, but we have different restraints on the number of cards they can receive.
Bella and Clara can both receive an odd number of cards that is at least 3 and at most 9.
Therefore, we have the following:
$$ B(z) = z^3 + z^5 + z^7 + z^9, \quad C(z) = z^3 + z^5 + z^7 + z^9 $$
To find all combinations that satisfy the given conditions, we can multiply the generating functions together,
and then find the coefficient of $z^{20}$, since Grandpa Dell has 20 cards to give out.
\begin{align*}
A(z)B(z)C(z) &= (z^8 + z^{10} + z^{12} + z^{14})(z^3 + z^5 + z^7 + z^9)(z^3 + z^5 + z^7 + z^9) \\
             &= x^{32} + 3x^{30} + 6x^{28} + 10x^{26} + 12x^{24} + 12x^{22} + 10x^{20} + 6x^{18} + 3x^{16} + x^{14} \\
[z^k]A(z)B(z)C(z) &= 10
\end{align*}
Thus, we have 10 different ways to give out the cards.
\end{solution}

\newpage
\begin{problem} (15 points) Section 13.2, Exercise 13.7. Explain.
\end{problem}
\begin{solution}
The sequence $(1, 0, 1, 0, ...)$ can be represented by the following generating function:
$$ A(z) = 1z^0 + 0z^1 + 1z^2 + 0z^3 + ... $$
since the first term is 1 which corresponds with $z^0$, and the second term is 0, and so on.
This can be simplified as:
$$ A(z) = 1 + z^2 + z^4 + z^6 + ... $$
In summation notation, this can be represented as:
$$ A(z) = \sum_{k=0}^{\infty} z^{2k} $$
Since this is a geometric series with common ratio $z^2$ and first term 1, the closed form is:
$$ A(z) = \frac{1}{1 - z^2} $$
The multiplicative inverse of a generating function must multiply with the original generating function to be identically 1.
This means the multiplicative inverse of $A(z)$ is simply just:
$$ B(z) = 1 - z^2 $$
\end{solution}

\newpage
\begin{problem} (15 points) Section 13.3, Exercise 13.12. Explain.
\end{problem}
\begin{solution}
The sequence $(a, a + d, a + 2d, a + 3d, ...)$ can be rewritten as:
$$ (a, a, a, a, ...) + (0, d, 2d, 3d, ...) $$
Doing this, we can use the elementary generating functions for the sequences $(1, 1, 1, 1, ...)$ and $(0, 1, 2, 3, ...)$,
which are $\frac{1}{1 - z}$ and $\frac{z}{(1 - z)^2}$, respectively.
Since we need to multiply $(1, 1, 1, 1, ...)$ by $a$ and $(0, 1, 2, 3, ...)$ by $d$, we need to multiply the generating functions by $a$ and $d$, respectively and add them together.
This gives us the following:
$$ A(z) = a\frac{1}{1 - z} + d\frac{z}{(1 - z)^2} $$
\end{solution}

\newpage
\begin{problem} ($15+15=30$ points) Section 14.2, Exercise 14.10.
\textit{Explain} your steps carefully.\\[.5ex]
For (a), study carefully how the example in Section 14.2 is solved using generating 
functions, and solve it in a similar way. (Be reminded that a ``closed form'' means a 
rational function that does not use a power series.)\\[.5ex]
For (b), do the partial fraction decomposition of $H(z)$ and expand it into a sum of 
two power series and then combine them into a power series to find the coefficient 
for the $z^k$ power term (like we did for the Fibonacci recurrence in the problem 
solving video and in the lecture notes). 
\end{problem}
\begin{solution}
\ \\
(a) Given the recurrence relation $h_0 = 1, \; h_n = 2h_{n-1} + 1$, we can represent this as a generating function
by pluggin in the values of $h_n$ into the ordinary generating function of the sequence $(h_0, h_1, h_2, ...)$:
$$ H(z) = h_0 + \sum_{k=1}^{\infty} (2h_{k-1} + 1) z^k $$
We know that $h_0 = 1$, and we can also distribute $z^k$ to get:
$$ H(z) = 1 + \sum_{k=1}^{\infty} 2h_{k-1}z^k + \sum_{k=1}^{\infty} z^k $$
We know that the series $\sum_{k=1}^{\infty} 2h_{k-1}z^k$ is just $2z$ times the function $H(z)$ itself but without the first term $h_0$.
We also know that $\sum_{k=1}^{\infty} z^k$ is just $\frac{z}{1 - z}$ since it is a geometric series.
Now we have the following:
$$ H(z) = 1 + 2zH(z) + \frac{z}{1 - z} $$
We can solve for $H(z)$ by moving the terms with $H(z)$ to the left side and the constant term to the right side:
\begin{align*}
H(z) - 2zH(z) &= 1 + \frac{z}{1 - z} \\
H(z)(1 - 2z) &= 1 + \frac{z}{1 - z} \\
H(z) &= \frac{1 + \frac{z}{1 - z}}{1 - 2z} \\
H(z) &= \frac{1}{2z^2 - 3z + 1}
\end{align*}
\ \\
(b) First we need to find the partial fraction decomposition of $H(z)$.
\begin{align*}
&H(z) = \frac{1}{2z^2 - 3z + 1} = \frac{1}{(z - 1)(2z - 1)} = \frac{A}{z - 1} + \frac{B}{2z - 1} \\
&1 = A(2z - 1) + B(z - 1) \\
&z = \frac{1}{2}: 1 = B(\frac{1}{2} - 1) \implies B = -2 \\
&z = 1: 1 = A(2 - 1) \implies A = 1 \\
\end{align*}
$$ H(z) = \frac{1}{z - 1} - \frac{2}{2z - 1} $$
Now, we can turn the fractions into geometric series:
\begin{align*}
H(z) &= \frac{1}{z - 1} - \frac{2}{2z - 1} \\
&= \sum_{k=0}^{\infty} z^k - 2 \sum_{k=0}^{\infty} (2z)^k \\
&= \sum_{k=0}^{\infty} z^k - 2 \sum_{k=0}^{\infty} 2^k z^k \\
&= \sum_{k=0}^{\infty} (1 - 2 \cdot 2^k) z^k \\
&= \sum_{k=0}^{\infty} (1 - 2^{k+1}) z^k \\
\end{align*}
Now we have an explicit formula for $h_k$:
$$ h_k = 1 - 2^{k+1} $$

\end{solution}

\newpage
\begin{problem} (15 points) Section 14.7, Exercise 14.30.  Study Example 14.14
in Section 14.7 and solve this exercise problem in a very similar way.  Also, 
\textit{explain} in a similar way as in Example 14.14.
\end{problem}
\begin{solution}
We have a recurrence relation with the initial conditions $g_0 = 2 ,\; g_1 = 1$
given by $g_n - 7g_{n-1} + 12g_{n-2} = 0$ for all $n \geq 2$.
This sequence has the following characteristic function:
$$\chi(z) = z^2 - 7z + 12 = (z - 3)(z - 4)$$
By corollary 14.13 in the textbook, we know the closed form of the coefficients $g_n$ is of the form:
$$ g_n = C_1 3^n + C_2 4^n $$
for some complex numbers $C_1$ and $C_2$.
Using the initial conditions, we can solve for $C_1$ and $C_2$:
\begin{align*}
g_0 = 2 &= C_1 3^0 + C_2 4^0 = C_1 + C_2 \\
g_1 = 1 &= C_1 3^1 + C_2 4^1 = 3C_1 + 4C_2 \\
\end{align*}
Now, we have the system:
$$ C_1 + C_2 = 2, \quad 3C_1 + 4C_2 = 1 $$
\begin{align*} 
  \left[\begin{array}{cc|c} 1 & 1 & 2 \\ 3 & 4 & 1 \\ \end{array}\right] \sim
  \left[\begin{array}{cc|c} 1 & 1 & 2 \\ 0 & 1 & -5 \\ \end{array}\right] \sim
  \left[\begin{array}{cc|c} 1 & 0 & 7 \\ 0 & 1 & -5 \\ \end{array}\right] \rightarrow
  \begin{cases} C_1 = 7 \\ C_2 = -5 \end{cases}
\end{align*}
Therefore, the closed form of the coefficients $g_n$ is:
$$ g_n = 7 \cdot 3^n - 5 \cdot 4^n $$
\end{solution}

\newpage
\begin{problem} (15 points) Section 14.7, Exercise 14.32.  Study Example 14.15
in Section 14.7 and solve this exercise problem in a very similar way.  Also, 
\textit{explain} in a similar way as in Example 14.15.
\end{problem}
\begin{solution}
We have a sequence given by the initial conditions $g_0 = 1,\; g_1 = 1$ and the recurrence relation
$g_n - 10g_{n-1} + 25g_{n-2} = 0$ for all $n \geq 2$.
This sequence has the following characteristic function:
$$ \chi(z) = z^2 - 10z + 25 = (z - 5)^2 $$
By proposition 14.7 in the textbook, we know that the closed form of the coefficients $g_n$ is of the form:
$$ g_n = D_{1, 1} 5^n + D_{1, 2} n 5^n $$
for some complex coefficients $D_{1, 1}$ and $D_{1, 2}$.
Substituting the initial conditions, we get this system of equations:
\begin{align*}
g_0 = 1 &= D_{1, 1} 5^0 + D_{1, 2} 0 5^0 = D_{1, 1} \\
g_1 = 1 &= D_{1, 1} 5^1 + D_{1, 2} 1 5^1 = 5D_{1, 1} + 5D_{1, 2} \\
\end{align*}
We know that $D_{1, 1} = 1$, and we can solve for $D_{1, 2}$:
\begin{align*}
1 &= 5 + 5D_{1, 2} \\
D_{1, 2} &= -\frac{4}{5}
\end{align*}
Now we have the closed form of the recurrence relation:
$$ g_n = 5^n - \frac{4}{5} n 5^n = 5^n - 4n 5^{n-1} $$
\end{solution}

\end{document}
