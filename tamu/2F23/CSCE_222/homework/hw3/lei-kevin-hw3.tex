\documentclass{article}
\usepackage{amsmath,amssymb,amsthm,latexsym,paralist}
\usepackage{fancyhdr}

\theoremstyle{definition}
\newtheorem{problem}{Problem}
\newtheorem*{solution}{Solution}
\newtheorem*{resources}{Resources}

\newcommand{\name}[2]{\noindent\textbf{Name: #1}\hfill \textbf{UIN: #2}
  \newcommand{\myName}{#1}
  \newcommand{\myUIN}{#2}
}
\newcommand{\honor}{\noindent On my honor, as an Aggie, I have neither
  given nor received any unauthorized aid on any portion of the
  academic work included in this assignment. Furthermore, I have
  disclosed all resources (people, books, web sites, etc.) that have
  been used to prepare this homework. \\[2ex]
 \textbf{Electronic signature:} \underline{ \textbf{Kevin Lei} } } % type your full name here
 
\newcommand{\checklist}{\noindent\textbf{Checklist:}
\begin{compactitem}[$\Box$] 
\item Did you type in your name and UIN? 
\item Did you disclose all resources that you have used? \\
(This includes all people, books, websites, etc.\ that you have consulted.)
\item Did you sign that you followed the Aggie Honor Code? 
\item Did you solve all problems? 
\item Did you submit both the .tex and .pdf files of your homework to each correct link on Canvas? 
\end{compactitem}
}

\newcommand{\problemset}[1]{\begin{center}\textbf{Problem Set #1}\end{center}}
\newcommand{\duedate}[1]{\begin{quote}\textbf{Due dates:} Electronic
    submission of \textsl{yourLastName-yourFirstName-hw3.tex} and 
    \textsl{yourLastName-yourFirstName-hw3.pdf} files of this homework is due on
    \textbf{#1} on \texttt{https://canvas.tamu.edu}. You will see two separate links
    to turn in the .tex file and the .pdf file separately. Please do not archive or compress the files.  
    \textbf{If any of the two files are missing, you will receive zero points for this homework.}\end{quote} }

\newcommand{\N}{\mathbf{N}}
\newcommand{\R}{\mathbf{R}}
\newcommand{\Z}{\mathbf{Z}}

\fancyhead[L]{\myName}
\fancyhead[R]{\myUIN}
\pagestyle{fancy}

\begin{document}
\begin{center}
{\large
CSCE 222 Discrete Structures for Computing -- Fall 2023\\[.5ex]
Hyunyoung Lee\\}
\end{center}
\problemset{3}
\duedate{Friday, 9/29/2023 11:59 p.m.}
\name{Kevin Lei}{432009232}  % Type your first and last name and UIN here
% Omit the parentheses surrounding name and UIN.
% Your name should include your first and last names. 
% Your name and UIN that you type in here are propagated by LaTeX 
% to the header part of each page on the PDF output automatically.

\begin{resources} (All people, books, articles, web pages, etc.\ that
  have been consulted when producing your answers to this homework)
\end{resources}
\honor

\bigskip

\noindent
Total $100+10$ (bonus) points.

\bigskip

\noindent
The intended formatting is that this first page is a cover page and each 
problem solved on a new page. You only need to fill in your solution between 
the \verb|\begin{solution}| and \verb|\end{solution}| environment.  
Please do not change this overall formatting.

\vfill
\checklist

\newpage
\begin{problem} (20 points) Section 3.4, Exercise 3.26.
[Hint: Use the definition of set difference, the distributive laws, and de Morgan's
laws involving the set complement. Starting from the right side of the equal sign
may be easier.]
\end{problem}
\begin{solution} 
The set equality $A \cap (B - C) = (A \cap B) - (A \cap C)$ can be proven if the right side can be transformed into the left side.
Starting with the right side, $(A \cap B) - (A \cap C)$ is equal to $(A \cap B) \cap (A \cap C)^c$ by definition of set difference.
Using De Morgan's laws, $(A \cap C)^c$ can be rewritten as $A^c \cup C^c$.
Substituting this back into the right side, we now have $(A \cap B) \cap (A^c \cup C^c)$
Using the distributive law we can rewrite as $(A \cap B \cap A^c) \cup (A \cap B \cap C^c)$.
Since $A \cap A^c$ is the empty set, we can rewrite $(A \cap B \cap A^c) \cup (A \cap B \cap C^c)$ as just $A \cap B \cap C^c$.
Using De Morgan's laws again, we can rewrite $B \cap C^c$ as $B - C$.
Substituting this back, we now have $A \cap (B - C)$, which is equivalent to the original left hand side.
Thus, the set equality is proven.
\end{solution}

\newpage
\begin{problem} (20 points) Section 3.5, Exercise 3.33.
[Hint: To show two sets $S_1$ and $S_2$ are equal ($S_1 = S_2$), you need to 
show that (1) $S_1\subseteq S_2$ \textit{and} (2) $S_2\subseteq S_1$. Here, for each
direction, you need to argue based on the definition of $\subseteq$.] 
\end{problem}
\begin{solution} 
The set equality $(A \cup B) \times C = (A \times C) \cup (B \times C)$ can be proven if both sides are subsets of each other.
First, proving that $(A \cup B) \times C \subseteq (A \times C) \cup (B \times C)$, suppose that the ordered pair $(x, y)$ is some element in $(A \cup B) \times C$.
By definition of the cartesian product, we know that $x \in (A \cup B)$ and $y \in C$.
Given that $x \in (A \cup B)$, we know that $x \in A$ or $x \in B$.
$x \in A$ implies that $(x, y) \in A \times C$, and $x \in B$ implies that $(x, y) \in B \times C$.
Regardless, $(x, y) \in (A \times C) \cup (B \times C)$, so $(A \cup B) \times C \subseteq (A \times C) \cup (B \times C)$.
Now proving that $(A \times C) \cup (B \times C) \subseteq (A \cup B) \times C$, suppose that the ordered pair $(x, y)$ is some arbitrary element in $(A \times C) \cup (B \times C)$.
In the case that $(x, y) \in (A \times C)$, we know that $x \in A$ and $y \in C$.
$x \in A$ implies that $x \in (A \cup B)$, so $(x, y) \in (A \cup B) \times C$.
Similarly, in the case that $(x, y) \in (B \times C)$, we know that $x \in B$ and $y \in C$.
By the same logic, $x \in B$ implies that $x \in (A \cup B)$, so $(x, y) \in (A \cup B) \times C$.
Regardless of whether $(x, y) \in (A \times C)$ or $(x, y) \in (B \times C)$, we know that $(x, y) \in (A \cup B) \times C$, so $(A \times C) \cup (B \times C) \subseteq (A \cup B) \times C$.
Since both sides of the set equality are subsets of each other, the set equality is proven.
\end{solution}

\newpage
\begin{problem} (20 points) Section 3.6, Exercise 3.37. \textit{Justify your answers.}
\end{problem}
\begin{solution} 
(a) The relation "is sibling of" on the set of all people is not reflexive.
A relation is reflexive if every element in the set is related to itself.
However, a person cannot be their own sibling, so the relation is not reflexive.

(b) The relation "is sibling of" on the set of all people is irreflexive.
A relation is irreflexive if no element in the set is related to itself.
Since all people are not their own sibling, no element in the set of all people fulfill the relation, so the relation is irreflexive.

(c) The relation "is sibling of" on the set of all people is asymmetric.
A relation is asymmetric if whenever $x$ is related to $y$, $y$ is not related to $x$.
If a person is a sibling of another person, then the other person is not a sibling of the first person.
Therefore, the relation "is sibling of" is not asymmetric.

(d) The relation "is sibling of" on the set of all people is not antisymmetric.
A relation is antisymmetric if whenever $x$ is related to $y$ and $y$ is related to $x$, then $x = y$.
Since two people can be related to each other as siblings, but not be the same person, the relation "is sibling of" is not antisymmetric.

(e) The relation "is sibling of" on the set of all people is symmetric.
A relation is symmetric if whenever $x$ is related to $y$, then $y$ is related to $x$.
If a person is a sibling of another person, then the other person is also a sibling of the first person.
Therefore, the relation "is sibling of" is symmetric.

(f) The relation "is sibling of" on the set of all people is transitive.
A relation is transitive if whenever $x$ is related to $y$ and $y$ is related to $z$, then $x$ is related to $z$.
If a person is a sibling of another person, and the other person is a sibling of a third person, then the first person is a sibling of the third person.
Therefore, the relation "is sibling of" is transitive.
\end{solution}

\newpage
\begin{problem} (30 points) Section 3.9, Exercise 3.60. Proving your function 
is bijective by showing that it is injective and surjective is required.
[Hint: Define a bijective function $f\colon \N_0\rightarrow \Z$ by
considering the argument being even or odd. Then prove that your 
function is indeed bijective by showing that it is injective and surjective.]
\end{problem}
\begin{solution} 
Two sets $A$ and $B$ are said to have the same cardinality if there exists a bijective function $f\colon A\rightarrow B$.
To prove that $|\N_0| = |\Z|$, we define a function $f\colon \N_0\rightarrow \Z$ where $f(n) = \frac{n}{2}$ if $n$ is even, and $f(n) = -\frac{n+1}{2}$ if $n$ is odd.
To show that this function is bijective, it must be both injective and surjective.
In order for the function to be injective, it must be that $f(a) = f(b)$ implies $a = b$.
In the case that $a$ and $b$ are both even, then $f(a) = \frac{a}{2}$ and $f(b) = \frac{b}{2}$.
If $f(a) = f(b)$, then $\frac{a}{2} = \frac{b}{2}$, then $a = b$ must be true.
In the case that both $a$ and $b$ are odd, then $f(a) = -\frac{a+1}{2}$ and $f(b) = -\frac{b+1}{2}$.
By the same logic, if $f(a) = f(b)$, then $-\frac{a+1}{2} = -\frac{b+1}{2}$, then algebraically, $a = b$ must be true.
In the case that $a$ is even and $b$ is odd or vice versa, then it is impossible for $f(a) = f(b)$ to be true, since $f(a)$ and $f(b)$ will always have different signs by the definition of $f$.
Considering all cases, it is true that $f(a) = f(b)$ implies $a = b$, so the function is injective.
To prove that a function is surjective, we must show that every element in the codomain is mapped to by at least one element in the domain.
The codomain of $f$ is $\Z$, so we must show that every integer is mapped to by at least one natural number which can be represented as $f(n) = z$.
When $z = 0$, $f(0) = 0$, so $z$ is mapped to by $0$.
When $z > 0$, $f(2z) = z$, so $z$ is mapped to by $2z$.
When $z < 0$, $f(-2z-1) = z$, so $z$ is mapped to by $-2z-1$.
Therefore, every element in the codomain is mapped to by at least one element in the domain, so the function is surjective.
Since $f$ is both injective and surjective, it is bijective, so $|\N_0| = |\Z|$.
\end{solution}

\newpage
\begin{problem} (20 points) Section 5.1, Exercise 5.4.
\end{problem}
\begin{solution} 
Proving that $\sim$ is an equivalence relation, it must be reflexive, symmetric, and transitive.
To show that $\sim$ is reflexive, we must show that $x \sim x$ for all $x \in \N_1$.
The relation would give $\frac{x}{x} = 2^{k}$ for some $k \in \Z$.
Since $\frac{x}{x} = 1$, $1 = 2^{0}$, and $0 \in \Z$, $x \sim x$ is true for all $x \in \N_1$, so $\sim$ is reflexive.
To show that $\sim$ is symmetric, we must show that $x \sim y$ implies $y \sim x$ for all $x, y \in \N_1$.
This requirement would mean that $\frac{y}{x}$ must equal $2^{k}$ for some $k \in \Z$.
Since $\frac{y}{x}$ is just the reciprocal of $\frac{x}{y}$, $\frac{y}{x}$ should just be the inverse of $2^{k}$, which is $2^{-k}$.
This shows that $x \sim y$ implies $y \sim x$ for all $x, y \in \N_1$, since $\frac{x}{y}$ and $\frac{y}{x}$ are both powers of $2$.
Therefore, the relation is symmetric.
To show that $\sim$ is transitive, we must show that $x \sim y$ and $y \sim z$ implies $x \sim z$ for all $x, y, z \in \N_1$.
Given that $x \sim y$, we know that $\frac{x}{y} = 2^{k}$ for some $k \in \Z$.
Likewise, given that $y \sim z$, we know that $\frac{y}{z} = 2^{j}$ for some $j \in \Z$.
The relation between $x$ and $z$ would be $\frac{x}{z} = \frac{x}{y} \cdot \frac{y}{z} = 2^{k} \cdot 2^{j} = 2^{k+j}$.
Since $k$ and $j$ are both integers, $k+j$ is also an integer, so the relation holds for $x \sim z$.
Therefore, the relation is transitive.
Since $\sim$ is reflexive, symmetric, and transitive, it is an equivalence relation.
\end{solution}

\end{document}
