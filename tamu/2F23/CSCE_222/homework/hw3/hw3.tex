\documentclass{article}
\usepackage{amsmath,amssymb,amsthm,latexsym,paralist}
\usepackage{fancyhdr}

\theoremstyle{definition}
\newtheorem{problem}{Problem}
\newtheorem*{solution}{Solution}
\newtheorem*{resources}{Resources}

\newcommand{\name}[2]{\noindent\textbf{Name: #1}\hfill \textbf{UIN: #2}
  \newcommand{\myName}{#1}
  \newcommand{\myUIN}{#2}
}
\newcommand{\honor}{\noindent On my honor, as an Aggie, I have neither
  given nor received any unauthorized aid on any portion of the
  academic work included in this assignment. Furthermore, I have
  disclosed all resources (people, books, web sites, etc.) that have
  been used to prepare this homework. \\[2ex]
 \textbf{Electronic signature:} \underline{ \textbf{Kevin Lei} } } % type your full name here
 
\newcommand{\checklist}{\noindent\textbf{Checklist:}
\begin{compactitem}[$\Box$] 
\item Did you type in your name and UIN? 
\item Did you disclose all resources that you have used? \\
(This includes all people, books, websites, etc.\ that you have consulted.)
\item Did you sign that you followed the Aggie Honor Code? 
\item Did you solve all problems? 
\item Did you submit both the .tex and .pdf files of your homework to each correct link on Canvas? 
\end{compactitem}
}

\newcommand{\problemset}[1]{\begin{center}\textbf{Problem Set #1}\end{center}}
\newcommand{\duedate}[1]{\begin{quote}\textbf{Due dates:} Electronic
    submission of \textsl{yourLastName-yourFirstName-hw3.tex} and 
    \textsl{yourLastName-yourFirstName-hw3.pdf} files of this homework is due on
    \textbf{#1} on \texttt{https://canvas.tamu.edu}. You will see two separate links
    to turn in the .tex file and the .pdf file separately. Please do not archive or compress the files.  
    \textbf{If any of the two files are missing, you will receive zero points for this homework.}\end{quote} }

\newcommand{\N}{\mathbf{N}}
\newcommand{\R}{\mathbf{R}}
\newcommand{\Z}{\mathbf{Z}}

\fancyhead[L]{\myName}
\fancyhead[R]{\myUIN}
\pagestyle{fancy}

\begin{document}
\begin{center}
{\large
CSCE 222 Discrete Structures for Computing -- Fall 2023\\[.5ex]
Hyunyoung Lee\\}
\end{center}
\problemset{3}
\duedate{Friday, 9/29/2023 11:59 p.m.}
\name{Kevin Lei}{432009232}  % Type your first and last name and UIN here
% Omit the parentheses surrounding name and UIN.
% Your name should include your first and last names. 
% Your name and UIN that you type in here are propagated by LaTeX 
% to the header part of each page on the PDF output automatically.

\begin{resources} (All people, books, articles, web pages, etc.\ that
  have been consulted when producing your answers to this homework)
\end{resources}
\honor

\bigskip

\noindent
Total $100+10$ (bonus) points.

\bigskip

\noindent
The intended formatting is that this first page is a cover page and each 
problem solved on a new page. You only need to fill in your solution between 
the \verb|\begin{solution}| and \verb|\end{solution}| environment.  
Please do not change this overall formatting.

\vfill
\checklist

\newpage
\begin{problem} (20 points) Section 3.4, Exercise 3.26.
[Hint: Use the definition of set difference, the distributive laws, and de Morgan's
laws involving the set complement. Starting from the right side of the equal sign
may be easier.]
\end{problem}
\begin{solution} 
\end{solution}

\newpage
\begin{problem} (20 points) Section 3.5, Exercise 3.33.
[Hint: To show two sets $S_1$ and $S_2$ are equal ($S_1 = S_2$), you need to 
show that (1) $S_1\subseteq S_2$ \textit{and} (2) $S_2\subseteq S_1$. Here, for each
direction, you need to argue based on the definition of $\subseteq$.] 
\end{problem}
\begin{solution} 
\end{solution}

\newpage
\begin{problem} (20 points) Section 3.6, Exercise 3.37. \textit{Justify your answers.}
\end{problem}
\begin{solution} 
\end{solution}

\newpage
\begin{problem} (30 points) Section 3.9, Exercise 3.60. Proving your function 
is bijective by showing that it is injective and surjective is required.
[Hint: Define a bijective function $f\colon \N_0\rightarrow \Z$ by
considering the argument being even or odd. Then prove that your 
function is indeed bijective by showing that it is injective and surjective.]
\end{problem}
\begin{solution} 
\end{solution}

\newpage
\begin{problem} (20 points) Section 5.1, Exercise 5.4.
\end{problem}
\begin{solution} 
\end{solution}

\end{document}
