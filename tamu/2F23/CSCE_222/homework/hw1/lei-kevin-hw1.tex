\documentclass{article}
\usepackage{amsmath,amssymb,amsthm,latexsym,paralist}
\usepackage{fancyhdr}

\theoremstyle{definition}
\newtheorem{problem}{Problem}
\newtheorem*{solution}{Solution}
\newtheorem*{resources}{Resources}

\newcommand{\name}[2]{\noindent\textbf{Name: #1}\hfill \textbf{UIN: #2}
  \newcommand{\myName}{#1}
  \newcommand{\myUIN}{#2}
}
\newcommand{\honor}{\noindent On my honor, as an Aggie, I have neither
  given nor received any unauthorized aid on any portion of the
  academic work included in this assignment. Furthermore, I have
  disclosed all resources (people, books, web sites, etc.) that have
  been used to prepare this homework. \\[2ex]
 \textbf{Electronic signature:} \underline{ \textbf{Kevin Lei} } } % Type your full name here
 
\newcommand{\checklist}{\noindent\textbf{Checklist:}
\begin{compactitem}[$\Box$] 
\item Did you type in your name and UIN? 
\item Did you disclose all resources that you have used? \\
(This includes all people, books, websites, etc.\ that you have consulted.)
\item Did you sign that you followed the Aggie Honor Code? 
\item Did you solve all problems? 
\item Did you submit both the .tex and .pdf files of your homework to each correct link on Canvas? 
\end{compactitem}
}

\newcommand{\problemset}[1]{\begin{center}\textbf{Problem Set #1}\end{center}}
\newcommand{\duedate}[1]{\begin{quote}\textbf{Due dates:} Electronic
    submission of \textsl{yourLastName-yourFirstName-hw1.tex} and 
    \textsl{yourLastName-yourFirstName-hw1.pdf} files of this homework is due on
    \textbf{#1} on \texttt{https://canvas.tamu.edu}. You will see two separate links
    to turn in the .tex file and the .pdf file separately. Please do not archive or compress the files.  
    \textbf{If any of the two files are missing, you will receive zero points for this homework.}\end{quote} }

\newcommand{\N}{\mathbf{N}}
\newcommand{\R}{\mathbf{R}}
\newcommand{\Z}{\mathbf{Z}}

\fancyhead[L]{\myName}
\fancyhead[R]{\myUIN}
\pagestyle{fancy}

\begin{document}
\begin{center}
{\large
CSCE 222 Discrete Structures for Computing -- Fall 2023\\[.5ex]
Hyunyoung Lee\\}
\end{center}
\problemset{1}
\duedate{Tuesday, 9/5/2023 11:59 p.m.}
\name{ Kevin Lei }{ 432009232 }  % Type your name and UIN here
% Omit the parentheses surrounding name and UIN.
% Your name should include your first and last names. 
% Your name and UIN that you type in here are propagated by LaTeX 
% to the header part of each page on the PDF output automatically.

\begin{resources} (All people, books, articles, web pages, etc. that
  have been consulted when producing your answers to this homework)
\end{resources}
\honor

\bigskip

\noindent
Total $100$ points.

\bigskip

\noindent
The intended formatting is that this first page is a cover page and each 
problem solved on a new page. You only need to fill in your solution between 
the \verb|\begin{solution}| and \verb|\end{solution}| environment.  
Please do not change this overall formatting.

\vfill
\checklist

\newpage
\begin{problem} ($10+10=20$ points) Section 1.1, Exercise 1.3.
For (b), give the knight's graph in a text format by giving all
edges in the graph such that the knight's move from vertex $v_i$ to 
vertex $v_{i+1}$ is given as $(v_i, v_{i+1})$.  Once you have all of the
edges written, you can also give the path in the form of 
$v_i - v_{i+1} - v_{i+2} - \ldots$

Use the common convention of expressing the columns and rows of
a chessboard as a, b, and c, and 1, 2, and 3, respectively.
\end{problem}
\begin{solution}

Part (a). Put simply, there are not enough squares on a 3x3 chessboard that would allow a knight to make a full tour without revisiting at least one square.

Part (b). The knights graph of a 3x3 chessboard is the following: (a1,b3), (a1,c2), (a2,c1), (a2,c3), (a3,b1), (a3,c2), (b1,a3), (b1,c3), (b3,a1), (b3,c1), (c1,a2), (c1,b2), (c2,a1), (c2,a3), (c3,a2), (c3,b2). 
In in the form of $v_i - v_{i+1} - v_{i+2} - \ldots$, the knight's graph is a1-b3-c1-a2-c3-b1-a3-c2-a1. It is clear from this representation that there is no way to visit all 9 squares without revisiting at least one square. This is because the maximum length of any path in this graph is 8 , which is one less than the total number of vertices.

\end{solution}

\newpage
\begin{problem} (2 points $\times$ 5 subproblems = 10 points) Section 2.1, Exercise 2.1
\end{problem}
\begin{solution}
a. Yes, this is a mathematical statement because it can be proven either true or false that $\pi$ is the smallest irrational real number.

b. Yes, this is a mathematical statement because it is a true statement due to the fact that $\pi$ is a well known irrational number.

c. Yes, this is a mathematical statement because it can be determined to be either true or false by solving for its solutions.

d. Yes, this is a mathematical statement because it can be determined to be true or false by comparing each side.

e. No, this is not a mathematical statement because we must know the value of x in order to know whether it can be a true or false statement.
\end{solution}

\newpage
\begin{problem} (3 points $\times$ 5 subproblems = 15 points) Section 2.1, Exercise 2.3
\end{problem}
\begin{solution}
a. True. Evaluating 1/9 in decimal form gives 0.111 repeating.

b. False. 0.121212 is a repeating decimal, which means it can be represented as a fraction of two integers. Therefore, it fits within the definition of a rational number and thus the statement is false.

c. False. The greatest common denominator between 1111 and 11111111 is 1111, which is not 1, and thus the statement is false.

d. True. The product of -1 and -1 is equal to positive 1.

e. True. The set of positive integers is infinite because you can always find a larger positive integer for every positive integer.
\end{solution}

\newpage
\begin{problem} (2 points $\times$ 2 subproblems = 4 points) Section 2.2, Exercise 2.7 (a) and (b)
\end{problem}
\begin{solution}
a. Albert cooks pasta and Emmy is not happy

b. If Albert cooks pasta, then Albert and Emmy are happy
\end{solution}

\newpage
\begin{problem} (3 points $\times$ 2 subproblems = 6 points) Section 2.2, Exercise 2.8 (a) and (d)
\end{problem}
\begin{solution}
a. C $\rightarrow$ $\neg$S

d. S $\leftrightarrow$ $\neg$C
\end{solution}

\newpage
\begin{problem} (15 points) Section 2.2, Exercise 2.18.
Use a truth table to show your reasoning. 

Example \LaTeX\ source for how to draw a truth table is shown 
in the truth-table.tex and truth-table.pdf files.
\end{problem}
\begin{solution}
Inhabitant A saying "I am a knave or B is a knight" can be formalized as A $\leftrightarrow$ ($\neg$A $\lor$ B) where A being true represents inhabitant A is a knight and B being true represents inhabitant B being a knight.
The statement has the following truth table:
\begin{displaymath}
\begin{array}{|c c||c|}
A & B & A \leftrightarrow (\neg A \lor B)\\
\hline
\text{False} & \text{False} & \text{False} \\
\text{False} & \text{True} & \text{False} \\
\text{True} & \text{False} & \text{False} \\
\text{True} & \text{True} & \text{True} \\
\end{array}
\end{displaymath}
As we can see from the truth table, the statement is true if and only if both A and B are true. If A is true, the B must also be true for the statement to be true. If A is false, then there is a contradiction since $(\neg A \lor B)$ evaluates to true. Hence, both A and B must be true in order for the statement to hold, which means both inhabitants A and B are knights.
\end{solution}

\newpage
\begin{problem} (10 points) Section 2.3, Exercise 2.25.
Use a truth table.
\end{problem}
\begin{solution}
\67
$(A \rightarrow B) \land (B \rightarrow A)$ and $A \leftrightarrow B$ are logically equivalent because their logical outcomes are the same for the same values of A and B, as shown by the truth table below.
\begin{displaymath}
\begin{array}{|c c||c|c|}
A & B & (A \rightarrow B) \land (B \rightarrow A) & A \leftrightarrow B \\
\hline
\text{False} & \text{False} & \text{True} & \text{True} \\
\text{False} & \text{True} & \text{False} & \text{False} \\
\text{True} & \text{False} & \text{False} & \text{False} \\
\text{True} & \text{True} & \text{True} & \text{True} \\
\end{array}
\end{displaymath}
    
\end{solution}

\newpage
\begin{problem} (20 points) Section 2.3, Exercise 2.26.
Your answer should consist of a series of logical equivalences 
you learned in the text, and the final step must resolve to $T$.
Do not use a truth table.  Study the proofs of Proposition 2.8 (b) and (c) 
for the expected style of your answer.  Watching the video ``Problem 
Solving Exercise 1" in Module {2.1} will also be helpful.

Example \LaTeX\ source for how to align the steps nicely is shown 
in the truth-table.tex and truth-table.pdf files.
\end{problem}
\begin{solution}
\begin{align*}
(A \land (A \rightarrow B)) \rightarrow B
&\equiv (A \land (\neg A \lor B)) \rightarrow B \quad \mbox{ because } A \rightarrow B \equiv \neg A \lor B\\
&\equiv ((A \land \neg A) \lor (A \land B)) \rightarrow B \quad \mbox{ by distributive law}\\
&\equiv (\text{F} \lor (A \land B)) \rightarrow B \quad \mbox{ because } A \land \neg A \equiv \text{F}\\
&\equiv (A \land B) \rightarrow B \quad \mbox{ by identity law}\\
&\equiv \neg (A \land B) \lor B \quad \mbox{ because } A \rightarrow B \equiv \neg A \lor B\\
&\equiv (\neg A \lor \neg B) \lor B \quad \mbox{ by De Morgan's law}\\
&\equiv \neg A \lor (\neg B \lor B) \quad \mbox{ by associative law}\\
&\equiv \neg A \lor \text{T} \quad \mbox{ since }(\neg B \lor B) \mbox{ must always be true}\\
&\equiv \text{T} \quad \mbox{ by domination law}\\
\end{align*}    
    
\end{solution}

\end{document}
