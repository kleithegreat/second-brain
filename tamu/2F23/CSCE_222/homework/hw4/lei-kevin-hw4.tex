\documentclass{article}
\usepackage{amsmath,amssymb,amsthm,latexsym,paralist}
\usepackage{fancyhdr}

\theoremstyle{definition}
\newtheorem{problem}{Problem}
\newtheorem*{solution}{Solution}
\newtheorem*{resources}{Resources}

\newcommand{\name}[2]{\noindent\textbf{Name: #1}\hfill \textbf{UIN: #2}
  \newcommand{\myName}{#1}
  \newcommand{\myUIN}{#2}
}
\newcommand{\honor}{\noindent On my honor, as an Aggie, I have neither
  given nor received any unauthorized aid on any portion of the
  academic work included in this assignment. Furthermore, I have
  disclosed all resources (people, books, web sites, etc.) that have
  been used to answer this homework. \\[2ex]
 \textbf{Electronic signature: \underline{Kevin Lei} } } % type your full name here
 
\newcommand{\checklist}{\noindent\textbf{Checklist:}
\begin{compactitem}[$\Box$] 
\item Did you type in your name and UIN? 
\item Did you disclose all resources that you have used? \\
(This includes all people, books, websites, etc.\ that you have consulted)
\item Did you sign that you followed the Aggie Honor Code? 
\item Did you solve all problems? 
\item Did you submit both the .tex and .pdf files of your homework to each correct link on Canvas? 
\end{compactitem}
}

\newcommand{\problemset}[1]{\begin{center}\textbf{Problem Set #1}\end{center}}
\newcommand{\duedate}[1]{\begin{quote}\textbf{Due dates:} Electronic
    submission of \textsl{yourLastName-yourFirstName-hw4.tex} and 
    \textsl{yourLastName-yourFirstName-hw4.pdf} files of this homework is due on
    \textbf{#1} on \texttt{https://canvas.tamu.edu}. You will see two separate links
    to turn in the .tex file and the .pdf file separately. Please do not archive or compress the files.  
    \textbf{If any of the two files are missing, you will receive zero points for this homework.}\end{quote} }

\newcommand{\N}{\mathbf{N}}
\newcommand{\R}{\mathbf{R}}
\newcommand{\Z}{\mathbf{Z}}

\fancyhead[L]{\myName}
\fancyhead[R]{\myUIN}
\pagestyle{fancy}

\begin{document}
\begin{center}
{\large
CSCE 222 Discrete Structures for Computing -- Fall 2023\\[.5ex]
Hyunyoung Lee\\}
\end{center}
\problemset{4}
\duedate{Friday, 10/13/2023 before 11:59 p.m.}
\name{Kevin Lei}{432009232} % Type your first and last name and UIN here
% Omit the parentheses surrounding name and UIN.
% Your name should include your first and last names. 
% Your name and UIN that you type in here are propagated by LaTeX 
% to the header part of each page on the PDF output automatically.

\begin{resources} (All people, books, articles, web pages, etc.\ that
  have been consulted when producing your answers to this homework)
\end{resources}
\honor

\bigskip

\noindent
Total $100+5$ (bonus) points.

\bigskip

\noindent
The intended formatting is that this first page is a cover page and each 
problem solved on a new page. You only need to fill in your solution between 
the \verb|\begin{solution}| and \verb|\end{solution}| environment.  
Please do not change this overall formatting.

\bigskip

\noindent
\textbf{Make sure that you strictly follow the structure of induction proof as shown in the 
lecture notes and how I solved in my videos.}

\vfill
\checklist

\newpage
\begin{problem} (15 points) Section 4.1, Exercise 4.3 
\end{problem}
\begin{solution}
We want to prove that the sum of the first $n$ squares is given by \[\sum_{k=1}^{n} k^2 = 1^2 + 2^2 + ... + n^2 = \frac{n(n+1)(2n+1)}{6}\] for all $n \ge 1$.
For the base case, the claim $P(n)$ holds for $n = 1$ since $\sum_{k=1}^{1} k^2 = 1^2 = \frac{1(1 + 1)(2 \times 1 + 1)}{6} = \frac{1\times2\times3}{3} =1$.
In the inductive step, it is our goal to show that $P(n) \rightarrow P(n+1)$ $\forall n \ge 1$, which means we need to prove that \[\sum_{i=1}^{k+1} i^{2} = \frac{(k+1)(k+1+1)(2(k+1)+1)}{6}\]
Using our base case, we can substitute the left side of the equation with $\sum_{i=1}^{k} i^{2} + (k+1)^{2}$.
We can then substitute $\sum_{i=1}^{k} i^{2}$ with $\frac{k(k+1)(2k+1)}{6}$, so our equation then becomes \[\frac{k(k+1)(2k+1)}{6} + (k+1)^{2} = \frac{(k+1)(k+1+1)(2(k+1)+1)}{6}\]
Multiply the equation by 6: \[k(k+1)(2k+1) + 6(k+1)^{2} = (k+1)(k+2)(2k+3)\]
Expand both sides:\[2k^{3}+k^{2}+2k^{2}+k+6k^{2}+12k+6 = 2k^{3}+3k^{2}+6k^{2}+9k+4k+6\]
Collect like terms: \[2k^{3}+9k^{2}+13k+6 = 2k^{3}+9k^{2}+13k+6\]
And thus both sides of the equation are equal, so $P(n) \rightarrow P(n+1)$ $\forall n \ge 1$.
Therefore, we have proven that the sum of the first $n$ squares is given by \[\sum_{k=1}^{n} k^2 = 1^2 + 2^2 + ... + n^2 = \frac{n(n+1)(2n+1)}{6}\] for all $n \ge 1$.
\end{solution}

\newpage
\begin{problem} (15 points) Section 4.1, Exercise 4.4 
\end{problem}
\begin{solution}
We want to prove by induction that the sum of the first $n$ cubes is given by \[\sum_{k=1}^{n} k^3 = 1^3 + 2^3 + ... + n^3 = (1 + 2 + .. + n)^2 = \frac{n^2 (n+1)^2}{4}\] for all $n\ge1$.
To do this we must prove this in two parts, since there are two claims in the equation.
In the first part we will prove that \[\sum_{k=1}^{n} k^3 = 1^3 + 2^3 + ... + n^3 = \frac{n^{2}(n+1)^{2}}{4}\] for all $n\ge1$.
For the base case where $n=1$, the formula holds true since: \[\sum_{k=1}^{1} k^3 = 1^3 = (1)^2 = \frac{1^2 (1+1)^2}{4} = 1\]
In the inductive step, we want to show that $P(n) \rightarrow P(n+1)$ $\forall n \ge 1$, which means we need to prove the following equation: \[\sum_{i=1}^{k+1} i^{3} = \frac{(k+1)^2 (k+2)^2}{4}\]
Using the base case, we can substitute the left side of the equation with $\sum_{i=1}^{k} i^{3} + (k+1)^{3}$, and then substitute $\sum_{i=1}^{k} i^{3}$ with $\frac{k^2 (k+1)^2}{4}$, so our equation then becomes \[\frac{k^2 (k+1)^2}{4} + (k+1)^{3} = \frac{(k+1)^2 (k+2)^2}{4}\]
Multiply the equation by 4: \[k^2 (k+1)^2 + 4(k+1)^{3} = (k+1)^2 (k+2)^2\]
Expand both sides: \[k^4 + 2k^3 + k^2 + 4k^3 + 12k^2 + 12k + 4k^2 + 12k + 12 = k^4 + 4k^3 + 6k^2 + 4k + k^2 + 4k + 4\]
Collect like terms: \[k^4 + 6k^3 + 19k^2 + 28k + 12 = k^4 + 6k^3 + 19k^2 + 28k + 12\]
And thus both sides of the equation are equal, so $P(n) \rightarrow P(n+1)$ $\forall n \ge 1$.
Therefore, we have proven that the sum of the first $n$ cubes is given by \[\sum_{k=1}^{n} k^3 = 1^3 + 2^3 + ... + n^3 = \frac{n^2 (n+1)^2}{4}\] for all $n\ge1$.
In the second part, we want to prove that \[\sum_{k=1}^{n} k^3 = 1^3 + 2^3 + ... + n^3 = (1 + 2 + .. + n)^2\] for all $n\ge1$.
For the base case where $n=1$, the formula holds true since: \[\sum_{k=1}^{1} k^3 = 1^3 = (1)^2 = 1\]
In the inductive step, we need to prove the following equation: \[\sum_{i=1}^{k+1} i^{3} = (1 + 2 + .. + k + (k+1))^2\]
Using the base case, we can substitute the left side of the equation with $\sum_{i=1}^{k} i^{3} + (k+1)^{3}$, and then substitute $\sum_{i=1}^{k} i^{3}$ with $(1 + 2 + .. + k)^2$, so our equation then becomes \[(1 + 2 + .. + k)^2 + (k+1)^{3} = (1 + 2 + .. + k + (k+1))^2\]
Using the formula for the first $n$ natural numbers, we can substitute $(1 + 2 + .. + k)$ with $\frac{k(k+1)}{2}$, so our equation then becomes \[\frac{k(k+1)}{2}^2 + (k+1)^{3} = (\frac{k(k+1)}{2} + (k+1))^2\]
Expanding both sides yields: \[\frac{k^4 + 2k^3 + k^2}{4} + k^3 + 3k^2 + 3k + 1 = \frac{k^4 + 2k^3 + k^2}{4} + k^3 + 3k^2 + 3k + 1\]
And thus both sides of the equation are equal, so we have proven that \[\sum_{k=1}^{n} k^3 = 1^3 + 2^3 + ... + n^3 = (1 + 2 + .. + n)^2\] for all $n\ge1$.
Since we have proven both parts of the equation, we have proven that the sum of the first $n$ cubes is given by \[\sum_{k=1}^{n} k^3 = 1^3 + 2^3 + ... + n^3 = (1 + 2 + .. + n)^2 = \frac{n^2 (n+1)^2}{4}\] for all $n\ge1$.
\end{solution}

\newpage
\begin{problem} (15 points) Section 4.1, Exercise 4.5 
\end{problem}
\begin{solution}
We want to prove by induction that the squares of the first $n$ odd positive integers is given by \[\sum_{k=1}^{n} (2k-1)^2 = 1^2 + 3^2 +5^2 + ... + (2n-1)^2 = \frac{1}{3} (4n^3 - n)\] for all positive integers $n$.
In the base step, we want to show that the equation holds true for $n=1$, which it does since: \[\sum_{k=1}^{1} (2k-1)^2 = 1^2 = \frac{1}{3} (4(1)^3 - 1) = \frac{1}{3} (4 - 1) = \frac{3}{3} = 1\]
In the inductive step, we want to show that $P(n) \rightarrow P(n+1)$ $\forall n \ge 1$, which means we need to prove the following equation: \[\sum_{i=1}^{k+1} (2i-1)^2 = \frac{1}{3} (4(k+1)^3 - (k+1))\]
Starting with the left side, we can substitute the summation with $\sum_{i=1}^{k} (2i-1)^2 + (2(k+1)-1)^2$, and then substitute $\sum_{i=1}^{k} (2i-1)^2$ with $\frac{1}{3} (4k^3 - k)$.
Our equation then becomes \[\frac{1}{3} (4k^3 - k) + (2(k+1)-1)^2 = \frac{1}{3} (4(k+1)^3 - (k+1))\]
Expanding both sides yields: \[\frac{4k^3 - k}{3} + 4k^2 + 4k + 1 = \frac{4k^3 + 12k^2 + 12k + 3}{3}\]
Collect like terms: \[\frac{4k^3 + 12k^2 + 12k + 3}{3} = \frac{4k^3 + 12k^2 + 12k + 3}{3}\]
And thus, both sides of the equation are equal, so we have proven by induction that the squares of the first $n$ odd positive integers is given by \[\sum_{k=1}^{n} (2k-1)^2 = 1^2 + 3^2 +5^2 + ... + (2n-1)^2 = \frac{1}{3} (4n^3 - n)\] for all positive integers $n$.
\end{solution}

\newpage
\begin{problem} (20 points) Section 4.1, Exercise 4.6 
\end{problem}
\begin{solution} 
We want to prove using induction that for all integers $n\ge1$, the integer $2^{2n}-1$ is divisible by 3.
In the base step, we want to show that the equation holds true for $n=1$.
Substituting $n=1$ into the equation yields: $2^{2(1)}-1 = 2^2-1 = 4-1 = 3$, which is divisible by 3, so the base case holds true.
In the inductive step, we want to show that $P(n) \rightarrow P(n+1)$ $\forall n \ge 1$, so we need to prove that $2^{2(n+1)}-1$ is also divisible by 3.
To show this, lets first assume that the statement holds true for some integer $n$, so we can say that $2^{2n}-1 = 3m$ which is divisible by 3.
Algebraically, $2^{2(n+1)}-1 = 2^{2n+2}-1$ can be rewritten as $2^{2n} \times 2^2 - 1$ which is $2^{2n} \times 4 - 1$.
From our assumption, we can say that $2^{2n}=3m+1$, so we can substitute that into our equation to get $(3m+1) \times 4 - 1$.
Simplyfing the equation yields $12m+3$, which is divisible by 3.
Thus, we have shown that $2^{2(n+1)}-1$ is divisible by 3, so $P(n) \rightarrow P(n+1)$ $\forall n \ge 1$.
Therefore, we have proven by induction that for all integers $n\ge1$, the integer $2^{2n}-1$ is divisible by 3.
\end{solution}

\newpage
\begin{problem} (20 points) Section 4.3, Exercise 4.15
\end{problem}
\begin{solution} 
We want to prove by induction that the sum of the first $n$ terms of the fibonacci sequence that have an even index is given by \[\sum_{k=1}^{n} f_{2k} = f_{2} + f_{4} + ... + f_{2n} = f_{2n+1} - 1\]
In the base step, we need to show that the equation holds for $n=1$, which it does since: $\sum_{k=1}^{1} f_{2k} = f_{2} = f_{2(1)+1} - 1 = f_{3} - 1 = 2 - 1 = 1$, and the second fibonacci number is 1.
In the inductive step, we shall assume that the proposition holds true for some arbitray number $n$, so we need to prove that it also holds true for $n+1$.
This means we need to prove the following equality: \[f_{2} + f_{4} + ... + f_{2n} + f_{2(n+1)} = f_{2(n+1)+1} - 1\]
Using the induction hypothesis, we can substitute the left side of the equation with $f_{2n+1} - 1 + f_{2(n+1)}$.
Simplifying the subscripts and adding $1$ to both sides, the equation then becomes \[f_{2n+1}+f_{2n+2} = f_{2n+3}\]
This equation now matches the definition of the fibonacci sequence, since we have $f_{n}+f_{n+1}=f_{n+2}$.
Therefore, we have proven the hypothesis using induction.

\end{solution}

\newpage
\begin{problem} (20 points) Section 4.6, Exercise 4.31
\end{problem}
\begin{solution} 
Let $g_{0}=1, g_{1}=2, g_{2}=3$, and $g_{n}=g_{n-1} + g_{n-2} + g_{n-3}$ for all integers $n\ge3$.
We want to prove using strong induction that $g_{n}\le2^{n}$ for all nonnegative integers $n$.
First we need to show that the proposition holds for all base cases.
For $n=0$, $g_{0}=1$, which is less than or equal to $2^{0}=1$, so the base case where $n=0$ holds true.
For $n=1$, $g_{1}=2$, which is less than or equal to $2^{1}=2$, so the base case when $n=1$ is also true.
Finally, for $n=2$, $g_{2}=3$, which is less than or equal to $2^{2}=4$, so the base case when $n=2$ still holds true.
In the inductive step, ???

\end{solution}

\end{document}
