\documentclass{article}
\usepackage{amsmath,amssymb,amsthm,latexsym,paralist}
\usepackage{fancyhdr}

\theoremstyle{definition}
\newtheorem{problem}{Problem}
\newtheorem*{solution}{Solution}
\newtheorem*{resources}{Resources}

\newcommand{\name}[2]{\noindent\textbf{Name: #1}\hfill \textbf{UIN: #2}
  \newcommand{\myName}{#1}
  \newcommand{\myUIN}{#2}
}
\newcommand{\honor}{\noindent On my honor, as an Aggie, I have neither
  given nor received any unauthorized aid on any portion of the
  academic work included in this assignment. Furthermore, I have
  disclosed all resources (people, books, web sites, etc.) that have
  been used to prepare this homework. \\[2ex]
 \textbf{Electronic signature:} \underline{ \textbf{Kevin Lei} } } % type your full name here
 
\newcommand{\checklist}{\noindent\textbf{Checklist:}
\begin{compactitem}[$\Box$] 
\item Did you type in your name and UIN? 
\item Did you disclose all resources that you have used? \\
(This includes all people, books, websites, etc.\ that you have consulted.)
\item Did you sign that you followed the Aggie Honor Code? 
\item Did you solve all problems? 
\item Did you submit both the .tex and .pdf files of your homework to each correct link on Canvas? 
\end{compactitem}
}

\newcommand{\problemset}[1]{\begin{center}\textbf{Problem Set #1}\end{center}}
\newcommand{\duedate}[1]{\begin{quote}\textbf{Due dates:} Electronic
    submission of \textsl{yourLastName-yourFirstName-hw2.tex} and 
    \textsl{yourLastName-yourFirstName-hw2.pdf} files of this homework is due on
    \textbf{#1} on \texttt{https://canvas.tamu.edu}. You will see two separate links
    to turn in the .tex file and the .pdf file separately. Please do not archive or compress the files.  
    \textbf{If any of the two files are missing, you will receive zero points for this homework.}\end{quote} }

\newcommand{\N}{\mathbf{N}}
\newcommand{\R}{\mathbf{R}}
\newcommand{\Z}{\mathbf{Z}}

\fancyhead[L]{\myName}
\fancyhead[R]{\myUIN}
\pagestyle{fancy}

\begin{document}
\begin{center}
{\large
CSCE 222 Discrete Structures for Computing -- Fall 2023\\[.5ex]
Hyunyoung Lee\\}
\end{center}
\problemset{2}
\duedate{Monday, 9/18/2023 11:59 p.m.}
\name{Kevin Lei}{432009232}  % Type your first and last name and UIN here
% Omit the parentheses surrounding name and UIN.
% Your name should include your first and last names. 
% Your name and UIN that you type in here are propagated by LaTeX 
% to the header part of each page on the PDF output automatically.

\begin{resources} (All people, books, articles, web pages, etc.\ that
  have been consulted when producing your answers to this homework)
\end{resources}
\honor

\bigskip

\noindent
Total $100$ points.

\bigskip

\noindent
The intended formatting is that this first page is a cover page and each 
problem solved on a new page. You only need to fill in your solution between 
the \verb|\begin{solution}| and \verb|\end{solution}| environment.  
Please do not change this overall formatting.

\vfill
\checklist

\newpage
\begin{problem} ($5+5=10$ points) Section 2.6, Exercise 2.53 (a) and (c). Explain.
\end{problem}
\begin{solution} 
Part (a). For the universe U of nonnegative integers, the only values of a, b, c that make predicate C(a, b, c) true are (0, 0, 0). 
This is because Fermat's Last Theorem states that there are no positive integer solutions to $a^{n} + b^{n} = c^{n}$ for values of n greater than 2. 
With positive integer solutions ruled out, the only other possible solution is zero, which is the only solution.

Part (c). For the universe U of \{1, 2, 3, 4, 5\}, there are two values of (a, b, c) that satisfies the predicate S(a, b, c), which is (3, 4, 5) and (4, 3, 5).
Since $a^{2} + b^{2} = c^{2}$ is just the pythagorean theorem, (3, 4, 5) is the only pythagorean triple in this universe, and this can be written in two ways since the tuple is ordered.
\end{solution}

\newpage
\begin{problem} ($5+5=10$ points) Section 2.6, Exercise 2.54 (b) and (c)
\end{problem}
\begin{solution}
Given that the universe is the set of real numbers:

Part (b). $\forall x \exists y (x < y)$ can be translated as for all real numbers $x$, there exists a real number $y$ such that $x$ is less than $y$.

Part (c). $\forall x \forall z \exists y (x < z) \rightarrow ((x < y) \land (y < z))$ can be translated as for all real numbers $x$ and $z$ where $x$ is less than $z$, there exists a real number $y$ such that $y$ is greater than $x$ and less than $z$.

\end{solution}

\newpage
\begin{problem} ($5+5=10$ points) Section 2.7, Exercise 2.58 (a) and (e)
\end{problem}
\begin{solution} 
Part (a). $\neg \forall x \exists y (P(x) \rightarrow Q(y))$ 

$\Leftrightarrow \exists x \forall y \neg (\neg P(x) \lor Q(x))$ 

$\Leftrightarrow \exists x \forall y (P(x) \land \neg Q(x))$

Part(e). $\neg \exists x \exists y (\neg P(x) \land \neg Q(y)) \Leftrightarrow \forall x \forall y (P(x) \lor Q(y))$
\end{solution}

\newpage
\begin{problem} ($5+5=10$ points) Section 2.7, Exercise 2.59 (d) and (e)
\end{problem}
\begin{solution} 
Part (d). Let $P(a, b)$ be the predicate such that $a + b = 1001$ and assume the universe is integers.
The statement can be formalized as $\exists a \forall b P(a, b)$.
Negating this statement, we end up with $\forall a \exists b \neg P(a, b)$.
In english, this would mean that for all integers a there exists an integer b such that a + b is not equal to 1001.

Part (e). Let $P(a, b)$ be the predicate such that $b < a$ where $b$ and $a$ are positive integers. 
The statement can be formalized as $\forall a \exists b P(a, b)$.
The negation of this statement would be $\exists a \forall b \neg Q(a, b)$.
In english, this would mean there exists a positive integer a such that for all positive integers b, the statement a is less than or equal to b is true.
\end{solution}

\newpage
\begin{problem} (15 points) Section 2.9, Exercise 2.73
[Hint: Use the property of ``consecutive integers" and the definition of an ``odd integer".]
\end{problem}
\begin{solution} 
Suppose $m$ and $n$ are consecutive integers. Numbers $m$ and $n$ being consecutive means that one of them must be even and the other must be odd.
By definition, an integer is even if it can be written as $2k$, and an integer is odd if it can be written as $2k + 1$.
If we say that $m$ is equal to $2k$ and n is equal to $2k + 1$, then $m + n$ would be equal to $2k + 2k + 1$.
This simplifies to $4k + 1$, and since it has the $+1$ at the end, following the definition of an odd integer, the sum of $m$ and $n$ must be odd.
\end{solution}

\newpage
\begin{problem} (15 points) Section 2.9, Exercise 2.80 
\end{problem}
\begin{solution}
The statement can be written as $(m + n > 100) \rightarrow (m > 40 \lor n > 60)$ where $m$ and $n$ are integers.
The contrapositive of this would be $(m \leq 40 \land n \leq 60) \rightarrow (m + n \leq 100)$
If we assume that $m \leq 40$ and $n \leq 60$, the the maximum values of $m$ and $n$ are 40 and 60 respectively. 
This would mean that the maximum value of $m + n$ is 100, and the implication holds.
Since the contrapositive is true, the original statement $(m + n > 100) \rightarrow (m > 40 \lor n > 60)$ is also true.
\end{solution}

\newpage
\begin{problem} (15 points) Section 2.9, Exercise 2.84 
\end{problem}
\begin{solution} 
Seeking a contradiction, assume the equation $42m + 70n = 1000$ has integer solutions.
The fundamental theorem of arithmetic states that every integer greater than 1 has a unique prime factorization.
Factoring both sides of the equation, we have $(2 \cdot 3 \cdot 7)m + (5 \cdot 7)n = 2^{3} \cdot 5^{3}$.
Since the $m$ and $n$ terms both have a factor of 7, the equation can be rewritten as $7(2 \cdot 3 \cdot m + 5 \cdot n) = 2^{3} \cdot 5^{3}$
Furthermore, dividing both sides by 7, the equation is now $2 \cdot 3 \cdot m + 5 \cdot n = \frac{2^{3} \cdot 5^{3}}{7}$.
Since $m$ and $n$ are integers, $2 \cdot 3 \cdot m + 5 \cdot n$ must also be an integer.
However, this brings us to a contradiction, since $\frac{2^{3} \cdot 5^{3}}{7}$ is not an integer, but a rational number.
Therefore, we have proved by contradiction that the equation $42m + 70n = 1000$ has no integer solutions.
\end{solution}

\newpage
\begin{problem} (15 points) Section 3.3, Exercise 3.20 
[Hint: Use the definitions of $\subseteq$, $\cup$, and the power set.]
\end{problem}
\begin{solution}
Suppose that arbitrary element $X \in P(A) \cup P(B)$. 
This means that either $X \subseteq A$ or $X \subseteq B$.
If $X \subseteq A$, then $X \subseteq A \cup B$ since $A \subseteq A \cup B$.
Since $X \subseteq A \cup B$, $X \in P(A \cup B)$.
The same reasoning holds if $X \subseteq B$.
If $X \subseteq B$, then $X \subseteq A \cup B$ since $B \subseteq A \cup B$.
Since $X \subseteq A \cup B$, $X \in P(A \cup B)$.
For both cases, $X \in P(A \cup B)$, and we have shown that any element in $P(A) \cup P(B)$ is also in $P(A \cup B)$.
By definition of the subset, $P(A) \cup P(B) \subseteq P(A \cup B)$.
Therefore, the statement is proved.
\end{solution}

\end{document}
