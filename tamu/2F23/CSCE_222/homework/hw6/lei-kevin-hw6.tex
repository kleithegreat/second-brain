\documentclass{article}
\usepackage{amsmath,amssymb,amsthm,latexsym,paralist}
\usepackage{fancyhdr}

\theoremstyle{definition}
\newtheorem{problem}{Problem}
\newtheorem*{solution}{Solution}
\newtheorem*{resources}{Resources}

\newcommand{\name}[2]{\noindent\textbf{Name: #1}\hfill \textbf{UIN: #2}
  \newcommand{\myName}{#1}
  \newcommand{\myUIN}{#2}
}

\newcommand{\honor}{\noindent On my honor, as an Aggie, I have neither
  given nor received any unauthorized aid on any portion of the
  academic work included in this assignment. Furthermore, I have
  disclosed all resources (people, books, web sites, etc.) that have
  been used to answer this homework. \\[2ex]
 \textbf{Electronic signature: \underline{Kevin Lei} } } % <= type your full name here
 
\newcommand{\checklist}{\noindent\textbf{Checklist:}
\begin{compactitem}[$\Box$] 
\item Did you type in your name and UIN? 
\item Did you disclose all resources that you have used? \\
(This includes all people, books, websites, etc.\ that you have consulted)
\item Did you sign that you followed the Aggie Honor Code? 
\item Did you solve all problems? 
\item Did you submit both the .tex and .pdf files of your homework to each correct link on Canvas? 
\end{compactitem}
}

\newcommand{\problemset}[1]{\begin{center}\textbf{Problem Set #1}\end{center}}
\newcommand{\duedate}[1]{\begin{quote}\textbf{Due dates:} Electronic
    submission of \textsl{yourLastName-yourFirstName-hw6.tex} and 
    \textsl{yourLastName-yourFirstName-hw6.pdf} files of this homework is due on
    \textbf{#1} on \texttt{https://canvas.tamu.edu}. You will see two separate links
    to turn in the .tex file and the .pdf file separately. Please do not archive or compress the files.  
    \textbf{If any of the two files are missing, you will receive zero points for this homework.}
    Your files must contain your first and last names and UIN in the given spaces and
    the electronic signature (your full name) correctly.\end{quote} }

\newcommand{\N}{\mathbf{N}}
\newcommand{\R}{\mathbf{R}}
\newcommand{\Z}{\mathbf{Z}}

\fancyhead[L]{\myName}
\fancyhead[R]{\myUIN}
\pagestyle{fancy}

\begin{document}
\begin{center}
{\large
CSCE 222 Discrete Structures for Computing -- Fall 2023\\[.5ex]
Hyunyoung Lee\\}
\end{center}
\problemset{6}
\duedate{Monday, 11/6/2023 before 11:59 p.m.}
\name{Kevin Lei}{432009232} % <= type your full name and UIN here
% Omit the parentheses surrounding name and UIN.
% Your name should include your first and last names. 
% Your name and UIN that you type in here are propagated by LaTeX 
% to the header part of each page on the PDF output automatically.

\begin{resources} (All people, books, articles, web pages, etc.\ that
  have been consulted when producing your answers to this homework)
\end{resources}
\honor

\bigskip

\noindent
Total 100 points.  
For a counting problem, \textit{careful, detailed explanation} will be worth majority (about 80\%) 
of your grade.

\bigskip

\noindent
The intended formatting is that this first page is a cover page and each 
problem solved on a new page. You only need to fill in your solution between 
the \verb|\begin{solution}| and \verb|\end{solution}| environment.  
Please do not change this overall formatting.

\vfill
\checklist

\newpage
\begin{problem} (10 points) Section 12.1, Exercise 12.4.  Specify what counting principle(s)
you are using.  Also explain carefully how you got your final answer.
\end{problem}
\begin{solution}
We can use the multiplication principle here. 
Since there are 26 ways to choose a letter and 10 ways to choose a digit, we can multiply them together to get the total number of license plates.
Thus we have the following:
$$ 26 \times 26 \times 26 \times 10 \times 10 \times 10 \times 10 = 175,760,000 \; \text{license plates} $$
\end{solution}

\newpage
\begin{problem} (10 points) Section 12.1, Exercise 12.5.  Specify what counting principle(s)
you are using.  Also explain carefully how you got your final answer.
\end{problem}
\begin{solution}
For this we need to use the multiplication principle and summation principle.
First we need to find the number of password combinations for password lengths 6, 7, and 8.
Then we can add them together to get the total number of password combinations.
For each character, there are 36 choices since there are 26 lowercase letters and 10 digits.
However, for the first character, there are only 26 choices since the first character must be a digit.
Thus we have the following:
$$ 26 \times 36^5 + 26 \times 36^6 + 26 \times 36^7 = 2,095,636,727,808 \; \text{password combinations} $$
\end{solution}

\newpage
\begin{problem} (15 points) Section 12.1, Exercise 12.9.  Specify what counting principle(s)
you are using.  Also explain carefully how you got your final answer.
\end{problem}
\begin{solution}
Here we can use the subtraction principle by first finding the total number of words of length $n$ over an alphabet of $k$ letters, 
then subtracting all palindromes of length $n$.
The number of words of length $n$ over an alphabet of $k$ letters is given by $k^n$,
since there are $k$ choices for each of the $n$ characters.
This is the multiplication principle.
A word is a palindrome if it is the same forwards and backwards.
This means that the first half of the word must be the same as the second half of the word.
Since the word length can be either odd or even, we need to consider both cases.
For both cases, the number of letters in the first half of the word is $\lceil n/2 \rceil$.
Therefore, the number of palindromes of length $n$ is given by $k^{\lceil n/2 \rceil}$.
Once again, this is the multiplication principle.
Finally, the total number of non-palindrome words of length $n$ over an alphabet of $k$ letters is given by $k^n - k^{\lceil n/2 \rceil}$.
\end{solution}

\newpage
\begin{problem} (10 points) Section 12.2, Exercise 12.19.  Specify what counting principle(s)
you are using.  Also explain carefully how you got your final answer.
\end{problem}
\begin{solution}
Since Tom wants to keep the math books together, they can be treated as a single book in calculating the number of arrangements.
We can use the multiplication principle by first finding the permutations of the math books within the novels, then multiplying by the number of permutations of the novels.
The number of permutations of the novels and math books (math books stick together) is $21!$,
since there are 20 novels, and we are treating the math books as a single book, since they must stick together.
Now, within the math books, there are $7!$ permutations, since there are 7 math books.
Thus, the total number of ways to arrange the bookshelf is $21! \times 7!$.
\end{solution}

\newpage
\begin{problem} ($10+10=20$ points) Section 12.3, Exercise 12.27.
For (a), use the formula involving the factorials.  For (b), use the hint given in the
problem statement, and explain carefully your double counting (combinatorial) 
proof in your own words.
\end{problem}
\begin{solution} 
\end{solution}

\newpage
\begin{problem} (15 points) Section 12.6, Exercise 12.50.  Explain your reasoning
carefully in your own words and show your work step-by-step.
[Hint: Consider the three sets: Set 1 with page numbers that contain a 1 in the least 
significant (1s) digit; set 2 with those that contain a 1 in the middle (10s) digit, and 
set 3 with those that contain a 1 in the most significant (100s) digit.]
\end{problem}
\begin{solution} 
\end{solution}

\newpage
\begin{problem} (20 points) What is the smallest number of ordered pairs of integers 
$(x, y)$ that are needed to guarantee that there are three ordered pairs 
$(x_1, y_1), (x_2, y_2)$, and $(x_3, y_3)$ such that
$x_1\textbf{ mod } 5 = x_2 \textbf{ mod } 5 = x_3 \textbf{ mod } 5$ and 
$y_1 \textbf{ mod } 4 = y_2 \textbf{ mod } 4 = y_3 \textbf{ mod } 4$\,?
Explain your reasoning carefully.
[Hint: This problem is about Pigeonhole Principle (Section 12.7). 
Carefully think what are the pigeonholes and what are the pigeons here.]
\end{problem}
\begin{solution} 
\end{solution}

\end{document}
