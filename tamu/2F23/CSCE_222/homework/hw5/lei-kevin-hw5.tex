\documentclass{article}
\usepackage{amsmath,amssymb,amsthm,latexsym,paralist}
\usepackage{fancyhdr}

\theoremstyle{definition}
\newtheorem{problem}{Problem}
\newtheorem*{solution}{Solution}
\newtheorem*{resources}{Resources}

\newcommand{\name}[2]{\noindent\textbf{Name: #1}\hfill \textbf{UIN: #2}
  \newcommand{\myName}{#1}
  \newcommand{\myUIN}{#2}
}

\newcommand{\honor}{\noindent On my honor, as an Aggie, I have neither
  given nor received any unauthorized aid on any portion of the
  academic work included in this assignment. Furthermore, I have
  disclosed all resources (people, books, web sites, etc.) that have
  been used to answer this homework. \\[2ex]
 \textbf{Electronic signature: \underline{Kevin Lei} } } % type your full name here
 
\newcommand{\checklist}{\noindent\textbf{Checklist:}
\begin{compactitem}[$\Box$] 
\item Did you type in your name and UIN? 
\item Did you disclose all resources that you have used? \\
(This includes all people, books, websites, etc.\ that you have consulted)
\item Did you sign that you followed the Aggie Honor Code? 
\item Did you solve all problems? 
\item Did you submit both the .tex and .pdf files of your homework to each correct link on Canvas? 
\end{compactitem}
}

\newcommand{\problemset}[1]{\begin{center}\textbf{Problem Set #1}\end{center}}
\newcommand{\duedate}[1]{\begin{quote}\textbf{Due dates:} Electronic
    submission of \textsl{yourLastName-yourFirstName-hw5.tex} and 
    \textsl{yourLastName-yourFirstName-hw5.pdf} files of this homework is due on
    \textbf{#1} on \texttt{https://canvas.tamu.edu}. You will see two separate links
    to turn in the .tex file and the .pdf file separately. Please do not archive or compress the files.  
    \textbf{If any of the two files are missing, you will receive zero points for this homework.}\end{quote} }

\newcommand{\N}{\mathbf{N}}
\newcommand{\R}{\mathbf{R}}
\newcommand{\Z}{\mathbf{Z}}

\fancyhead[L]{\myName}
\fancyhead[R]{\myUIN}
\pagestyle{fancy}

\begin{document}
\begin{center}
{\large
CSCE 222 Discrete Structures for Computing -- Fall 2023\\[.5ex]
Hyunyoung Lee\\}
\end{center}
\problemset{5}
\duedate{Monday, 10/23/2023 before 11:59 p.m.}
\name{Kevin Lei}{432009232} % Type your first and last name and UIN here
% Omit the parentheses surrounding name and UIN.
% Your name should include your first and last names. 
% Your name and UIN that you type in here are propagated by LaTeX 
% to the header part of each page on the PDF output automatically.

\begin{resources} (All people, books, articles, web pages, etc.\ that
  have been consulted when producing your answers to this homework)
\end{resources}
\honor

\bigskip

\noindent
Total $100+10$ (bonus) points.

\bigskip

\noindent
The intended formatting is that this first page is a cover page and each 
problem solved on a new page. You only need to fill in your solution between 
the \verb|\begin{solution}| and \verb|\end{solution}| environment.  
Please do not change this overall formatting.

\bigskip

\vfill
\checklist

\newpage
\begin{problem} (10 points) Section 11.1, Exercise 11.3 
\end{problem}
\begin{solution}
i)
\end{solution}

\newpage
\begin{problem} (20 points) Section 11.3, Exercise 11.14.
[Requirement: Study the definition of $\asymp$ involving the inequalities carefully 
and use the definition to answer the questions.] 
\end{problem}
\begin{solution}
\end{solution}

\newpage
\begin{problem} (15 points)
Prove that $3n^2+41 \in O(n^3)$ by giving a direct proof based on the definition of 
big-$O$ involving the inequalities and absolute values, as given in the lecture notes 
Section 11.4.  

To do so, first write out what $3n^2+41 \in O(n^3)$ means according to the definition.  
Then, you need to find a positive real constant $C$ and a positive integer $n_0$ 
that satisfy the definition.
\end{problem}
\begin{solution}
\end{solution}

\newpage
\begin{problem} (15 points) 
Prove that $\frac{1}{2}n^2+5 \in \Omega(n)$ by giving a direct proof based on the 
definition of big-$\Omega$ involving the inequalities and absolute values, as given 
in the lecture notes Section 11.5.

To do so, first write out what $\frac{1}{2}n^2+5 \in \Omega(n)$ means according 
to the definition. Then, you need to find a positive real constant $c$ and a positive 
integer $n_0$ that satisfy the definition.
\end{problem}
\begin{solution}
\end{solution}

\newpage
\begin{problem} ($10+10=20$ points) Read Section 11.6 carefully before attempting 
this problem.

Analyze the running time of the following algorithm using a step count analysis 
as shown in the Horner scheme (Example 11.40).  
\begin{verbatim}
// search a key in an array a[1..n] of length n
search(a, n, key)        cost   times
  for k in (1..n) do      c1    [  ]   
    if a[k]=key then      c2    [  ]
       return k           c3    [  ]
  endfor                  c4    [  ]
  return false            c5    [  ]
\end{verbatim}
(a) Fill in the \verb|[  ]|s in the above code each with a number or an expression involving
\verb|n| that expresses the step count for the line of code.

\medskip
\noindent
(b) Determine the worst-case complexity of this algorithm and give it in the $\Theta$ notation.
Show your work and explain using the definition of $\Theta$ involving the inequalities. 
\end{problem}
\begin{solution} (For part (b))
\end{solution}

\newpage
\begin{problem} ($15+15=30$ points) Read Section 11.6 carefully before attempting this problem.
Analyze the running time of the following algorithm using a step count analysis 
as shown in the Horner scheme (Example 11.40).
\begin{verbatim}
// determine the number of digits of an integer n
binary_digits(n)            cost  times
  int cnt = 1                c1   [  ]
  while (n > 1) do           c2   [  ]
    cnt = cnt + 1            c3   [  ]
    n = floor( n/2.0 )       c4   [  ]
  endwhile                   c5   [  ]
  return cnt                 c6   [  ]
\end{verbatim}
\noindent
(a) Fill in the \verb|[  ]|s in the above code each with a number or an expression involving
\verb|n| that expresses the step count for the line of code.

\medskip
\noindent
(b) Determine the worst-case complexity of this algorithm as a function of $n$
and give it in the $\Theta$ notation.
Show your work and explain using the definition of $\Theta$ involving the inequalities. 
\end{problem}
\begin{solution} (For part (b))
\end{solution}

\end{document}
