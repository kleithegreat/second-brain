\documentclass{article}
\usepackage{amsmath,amssymb,amsthm,latexsym,paralist}
\usepackage{fancyhdr}

\theoremstyle{definition}
\newtheorem{problem}{Problem}
\newtheorem*{solution}{Solution}
\newtheorem*{resources}{Resources}

\newcommand{\name}[2]{\noindent\textbf{Name: #1}\hfill \textbf{UIN: #2}
  \newcommand{\myName}{#1}
  \newcommand{\myUIN}{#2}
}

\newcommand{\honor}{\noindent On my honor, as an Aggie, I have neither
  given nor received any unauthorized aid on any portion of the
  academic work included in this assignment. Furthermore, I have
  disclosed all resources (people, books, web sites, etc.) that have
  been used to answer this homework. \\[2ex]
 \textbf{Electronic signature: \underline{Kevin Lei} } } % type your full name here
 
\newcommand{\checklist}{\noindent\textbf{Checklist:}
\begin{compactitem}[$\Box$] 
\item Did you type in your name and UIN? 
\item Did you disclose all resources that you have used? \\
(This includes all people, books, websites, etc.\ that you have consulted)
\item Did you sign that you followed the Aggie Honor Code? 
\item Did you solve all problems? 
\item Did you submit both the .tex and .pdf files of your homework to each correct link on Canvas? 
\end{compactitem}
}

\newcommand{\problemset}[1]{\begin{center}\textbf{Problem Set #1}\end{center}}
\newcommand{\duedate}[1]{\begin{quote}\textbf{Due dates:} Electronic
    submission of \textsl{yourLastName-yourFirstName-hw5.tex} and 
    \textsl{yourLastName-yourFirstName-hw5.pdf} files of this homework is due on
    \textbf{#1} on \texttt{https://canvas.tamu.edu}. You will see two separate links
    to turn in the .tex file and the .pdf file separately. Please do not archive or compress the files.  
    \textbf{If any of the two files are missing, you will receive zero points for this homework.}\end{quote} }

\newcommand{\N}{\mathbf{N}}
\newcommand{\R}{\mathbf{R}}
\newcommand{\Z}{\mathbf{Z}}

\fancyhead[L]{\myName}
\fancyhead[R]{\myUIN}
\pagestyle{fancy}

\begin{document}
\begin{center}
{\large
CSCE 222 Discrete Structures for Computing -- Fall 2023\\[.5ex]
Hyunyoung Lee\\}
\end{center}
\problemset{5}
\duedate{Monday, 10/23/2023 before 11:59 p.m.}
\name{Kevin Lei}{432009232} % Type your first and last name and UIN here
% Omit the parentheses surrounding name and UIN.
% Your name should include your first and last names. 
% Your name and UIN that you type in here are propagated by LaTeX 
% to the header part of each page on the PDF output automatically.

\begin{resources} (All people, books, articles, web pages, etc.\ that
  have been consulted when producing your answers to this homework)
\end{resources}
\honor

\bigskip

\noindent
Total $100+10$ (bonus) points.

\bigskip

\noindent
The intended formatting is that this first page is a cover page and each 
problem solved on a new page. You only need to fill in your solution between 
the \verb|\begin{solution}| and \verb|\end{solution}| environment.  
Please do not change this overall formatting.

\bigskip

\vfill
\checklist

\newpage
\begin{problem} (10 points) Section 11.1, Exercise 11.3 
\end{problem}
\begin{solution}
To determine who is correct in this situation, we can use the definition of asymptotic equality. 
We have that $f(n) = n^2 + 2n$ and $g(n) = n^2$.
Taking the limit of $\frac{f(n)}{g(n)}$ as $n$ approaches infinity, we have:
$$ \lim_{n \to \infty} \frac{f(n)}{g(n)} = \lim_{n \to \infty} \frac{n^2 + 2n}{n^2} = \lim_{n \to \infty} \frac{n + 2}{n} $$
Using L'Hopital's rule, we have:
$$ \lim_{n \to \infty} \frac{n + 2}{n} = \lim_{n \to \infty} \frac{1}{1} = 1 $$
This means that by definition, $f(n) \sim g(n)$, so Ernie is correct.
Bert says that $f$ and $g$ cannot be asymptotically equal since $f(n) - g(n) \geq 2n$ for all $n \geq 1$.
However, this is the wrong way to look at asymptotic equality.
Asymptotic equality is supposed to describe the behavior of functions for large inputs.
Bert is focusing on the absolute difference between the functions $f$ and $g$.
However, asymptotic equality deals with the relative error $(f(n) - g(n)) / g(n)$.
In the case of $f(n) = n^2 + 2n$ and $g(n) = n^2$, the relative error does vanish for large inputs,
since it is this limit: $\lim_{n \to \infty} \frac{f(n) - g(n)}{g(n)} = \lim_{n \to \infty} \frac{2n}{n^2} = \lim_{n \to \infty} \frac{2}{n} = 0$.
\end{solution}

\newpage
\begin{problem} (20 points) Section 11.3, Exercise 11.14.
[Requirement: Study the definition of $\asymp$ involving the inequalities carefully 
and use the definition to answer the questions.] 
\end{problem}
\begin{solution}
Given that $f: \N_1 \to \R$ and $g: \N_1 \to \R$,
the functions $f$ and $g$ have the same order of growth (represented by $f \asymp g$) 
if and only if there exist positive real constants $c$ and $C$ 
and a positive integer $n_0$ such that $$c|g(n)| \leq |f(n)| \leq C|g(n)|$$ for all $n \geq n_0$.
To show that this is an equivalence relation, we need to show the following:
\begin{enumerate}[(i)]
\item $f \asymp f$
\newline
Here we need to find positive real constants $c$ and $C$ and a positive integer $n_0$ 
such that $c|f(n)| \leq |f(n)| \leq C|f(n)|$ for all $n \geq n_0$.
The obvious choice is $c = C = 1$ and $n_0 = 1$, so the relation is reflexive.
\item $f \asymp g$ if and only if $g \asymp f$.
\newline
Suppose that $f \asymp g$.
This implies that there exist positive real constants $c$ and $C$ and a positive integer $n_0$
such that $c|g(n)| \leq |f(n)| \leq C|g(n)|$ for all $n \geq n_0$.
Splitting up this inequality, we have $c|g(n)| \leq |f(n)|$ and $|f(n)| \leq C|g(n)|$.
Dividing by $c$ and $C$ respectively, we have $|g(n)| \leq \frac{1}{c}|f(n)|$ and $\frac{1}{C}|f(n)| \leq |g(n)|$,
which still holds for all $n \geq n_0$.
Using this, we can write the definition of $g \asymp f$ as an inequality:
$$ \frac{1}{C}|f(n)| \leq |g(n)| \leq \frac{1}{c}|f(n)| $$
Since this holds for all $n \geq n_0$, we have that $g \asymp f$, and $f \asymp g \rightarrow g \asymp f$.
The same logic can be applied to show that given $g \asymp f$, $f \asymp g$ is true.
Therefore, the relation is symmetric.
\item $f \asymp g$ and $g \asymp h$ implies $f \asymp h$.
\newline
Given that $f \asymp g$ and $g \asymp h$, we know that there exist positive real constants $c_1$, $C_1$, $c_2$, and $C_2$
and positive integers $n_1$ and $n_2$ such that $c_1|g(n)| \leq |f(n)| \leq C_1|g(n)|$ for all $n \geq n_1$
and $c_2|h(n)| \leq |g(n)| \leq C_2|h(n)|$ for all $n \geq n_2$.
Using this information, we want to show that there exist positive real constants $c$ and $C$ and a positive integer $n_0$
such that $c|h(n)| \leq |f(n)| \leq C|h(n)|$ for all $n \geq n_0$.
From the definitions, we have that $c_1|g(n)| \leq |f(n)|$ and $c_2|h(n)| \leq |g(n)|$.
Multiplying the second inequality by $c_1$ and putting the two together, we have $c_1c_2|h(n)| \leq |f(n)|$.
Doing the same for the upper bound, we have $|f(n)| \leq C_1C_2|h(n)|$.
Now we have the inequality $c_1c_2|h(n)| \leq |f(n)| \leq C_1C_2|h(n)|$, which holds for all $n \geq max(n_1, n_2)$.
This satisfies the definition of $f \asymp h$, so $f \asymp g$ and $g \asymp h$ implies $f \asymp h$.
\end{enumerate}
Therefore, $\asymp$ is an equivalence relation.
\end{solution}

\newpage
\begin{problem} (15 points)
Prove that $3n^2+41 \in O(n^3)$ by giving a direct proof based on the definition of 
big-$O$ involving the inequalities and absolute values, as given in the lecture notes 
Section 11.4.  

To do so, first write out what $3n^2+41 \in O(n^3)$ means according to the definition.  
Then, you need to find a positive real constant $C$ and a positive integer $n_0$ 
that satisfy the definition.
\end{problem}
\begin{solution}
For functions $f: \N_1 \to \R$ and $g: \N_1 \to \R$, we say that $g$ is an asymptotic upper bound for $f$, 
written as $f \in O(g)$, if and only if there exists a positive real constant $C$ and a positive integer $n_0$
such that $$|f(n)| \leq C|g(n)|$$ for all $n \geq n_0$.
Using the definition of big-$O$, we have that $3n^2 + 41 \in O(n^3)$ if and only if 
there exist a positive real constant $C$ and and a positive integer $n_0$
such that $|3n^2 + 41| \leq C|n^3|$ for all $n \geq n_0$.
Since both sides of the inequality are positive for all $n \geq 1$, we can remove the absolute value signs,
and we have $3n^2 + 41 \leq Cn^3$. This can be rewritten as $C \geq \frac{3n^2 + 41}{n^3}$.
For $n = 1$, we have $C \geq \frac{3 + 41}{1} = 44$.
For larger values of $n$, the fraction $\frac{3n^2 + 41}{n^3}$ approaches zero, so we can choose any value of $C$ greater than 44.
This means we can choose $C = 44$ and $n_0 = 1$ to satisfy the definition of big-$O$.
Therefore, we have directly proven that $3n^2 + 41 \in O(n^3)$.
\end{solution}

\newpage
\begin{problem} (15 points) 
Prove that $\frac{1}{2}n^2+5 \in \Omega(n)$ by giving a direct proof based on the 
definition of big-$\Omega$ involving the inequalities and absolute values, as given 
in the lecture notes Section 11.5.

To do so, first write out what $\frac{1}{2}n^2+5 \in \Omega(n)$ means according 
to the definition. Then, you need to find a positive real constant $c$ and a positive 
integer $n_0$ that satisfy the definition.
\end{problem}
\begin{solution}
For functions $f: \N_1 \to \R$ and $g: \N_1 \to \R$, we say that $g$ is an asymptotic lower bound for $f$,
written as $f \in \Omega(g)$, if and only if there exists a positive real constant $c$ and a positive integer $n_0$
such that $$ c|g(n)| \leq |f(n)| $$ for all $n \geq n_0$.
To prove that $\frac{1}{2}n^2 + 5 \in \Omega(n)$, we need to find a positive real constant $c$ and a positive integer $n_0$
such that $c|n| \leq |\frac{1}{2}n^2 + 5|$ for all $n \geq n_0$.
Since both functions have a domain of $\N_1$, we can remove the absolute value signs, and we have $cn \leq \frac{1}{2}n^2 + 5$.
Now we can rewrite the inequality as $c \leq \frac{n}{2} + \frac{5}{n}$.
Starting with $n = 1$, we have $c \leq \frac{1}{2} + 5 = \frac{11}{2}$.
As $n$ approaches infinity, the fraction $\frac{5}{n}$ approaches zero, so we can choose any value of $c$ less than or equal to $\frac{11}{2}$.
This means we can choose $c = \frac{11}{2}$ and $n_0 = 1$ to satisfy the definition of big-$\Omega$.
Therefore, we have directly proven that $\frac{1}{2}n^2 + 5 \in \Omega(n)$.
\end{solution}

\newpage
\begin{problem} ($10+10=20$ points) Read Section 11.6 carefully before attempting 
this problem.

Analyze the running time of the following algorithm using a step count analysis 
as shown in the Horner scheme (Example 11.40).  
\begin{verbatim}
// search a key in an array a[1..n] of length n
search(a, n, key)        cost   times
  for k in (1..n) do      c1    [ n ]   
    if a[k]=key then      c2    [ n ]
       return k           c3    [ 1 ]
  endfor                  c4    [ n ]
  return false            c5    [ 1 ]
\end{verbatim}
(a) Fill in the \verb|[  ]|s in the above code each with a number or an expression involving
\verb|n| that expresses the step count for the line of code.

\medskip
\noindent
(b) Determine the worst-case complexity of this algorithm and give it in the $\Theta$ notation.
Show your work and explain using the definition of $\Theta$ involving the inequalities. 
\end{problem}
\begin{solution} (For part (b))
The time compexity of this algorithm is given by $T(n) = c_1n + c_2n + c_3 + c_4n + c_5$.
Since the highest order term in this equation is $n$, we can say that $T(n) \in \Theta(n)$.
According to the definition of $\Theta$, this means that there exist positive real constants $c$ and $C$
and a positive integer $n_0$ such that $cn \leq T(n) \leq Cn$ for all $n \geq n_0$.
Let's say that $n_0 = 1$.
Starting with the lower bound, we have $cn \leq c_1n + c_2n + c_3 + c_4n + c_5$.
Since $n$ only gets larger, that means $c$ can be any vaue less than or equal to $c_1 + c_2 + c_4$.
For the upper bound, we have $c_1n + c_2n + c_3 + c_4n + c_5 \leq Cn$.
Following the same logic, we can choose any value for $C$ as long as it is greater than or equal to $c_1 + c_2 + c_4$.
Since there exits positive real constants $c$ and $C$ and a positive integer $n_0$ such that $cn \leq T(n) \leq Cn$ for all $n \geq n_0$,
we have shown that $T(n) \in \Theta(n)$.
\end{solution}

\newpage
\begin{problem} ($15+15=30$ points) Read Section 11.6 carefully before attempting this problem.
Analyze the running time of the following algorithm using a step count analysis 
as shown in the Horner scheme (Example 11.40).
\begin{verbatim}
// determine the number of digits of an integer n
binary_digits(n)            cost  times
  int cnt = 1                c1   [ 1 ]
  while (n > 1) do           c2   [ log(n) ]
    cnt = cnt + 1            c3   [ log(n) ]
    n = floor( n/2.0 )       c4   [ log(n) ]
  endwhile                   c5   [ log(n) ]
  return cnt                 c6   [ 1 ]
\end{verbatim}
\noindent
(a) Fill in the \verb|[  ]|s in the above code each with a number or an expression involving
\verb|n| that expresses the step count for the line of code.

\medskip
\noindent
(b) Determine the worst-case complexity of this algorithm as a function of $n$
and give it in the $\Theta$ notation.
Show your work and explain using the definition of $\Theta$ involving the inequalities. 
\end{problem}
\begin{solution} (For part (b))
The time complexity of this algorithm is given by $T(n) = c_1 + c_2\log(n) + c_3\log(n) + c_4\log(n) + c_5\log(n) + c_6$.
Since the fastest growing term in this equation is $\log(n)$, we can say that $T(n) \in \Theta(\log(n))$.
Using the definition of $\Theta$, this means that there exist constants $c$ and $C$ and a positive integer $n_0$
such that $c\log(n) \leq T(n) \leq C\log(n)$ for all $n \geq n_0$.
Going with $n_0 = 1$, we can analyze the lower and upper bounds separately.
Starting with the lower bound, we need to find a real constant $c$ such that $c\log(n) \leq c_1 + c_2\log(n) + c_3\log(n) + c_4\log(n) + c_5\log(n) + c_6$.
Dividing by $\log(n)$ and taking the limit as $n$ approaches infinity, we know that $c$ must be less than or equal to $c_2 + c_3 + c_4 + c_5$.
Similarly, for the upper bound, there must be a real constant $C$ such that $c_1 + c_2\log(n) + c_3\log(n) + c_4\log(n) + c_5\log(n) + c_6 \leq C\log(n)$.
Doing the same thing, $C$ must be greater than or equal to $c_2 + c_3 + c_4 + c_5$.
Since there exist constants $c$ and $C$ and a positive integer $n_0$ such that $c\log(n) \leq T(n) \leq C\log(n)$ for all $n \geq n_0$,
we have shown that $T(n) \in \Theta(\log(n))$.
\end{solution}

\end{document}
