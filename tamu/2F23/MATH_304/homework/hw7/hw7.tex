\documentclass{article}
\usepackage{amsmath,amssymb,amsthm}
\usepackage{fancyhdr}
\usepackage{enumerate}

\pagestyle{fancy}
\fancyhf{}
\lhead{7th Homework - MATH 304 508}
\rhead{Kevin Lei}
\renewcommand{\headrulewidth}{0.4pt}

\begin{document}

\section*{Question 1}
Determine whether the following transformations are linear: Explain your answer.
\begin{enumerate}[a.]
    \item $ F((x_1, x_2, x_3)^T) = (x_1 - x_2, x_2 - x_1)^T $
    \item $ F((x_1, x_2, x_3)^T) = (1, 2, x_1 + x_2 + x_3)^T $
    \item $ F((x_1)) = (x_1, 2x_1, 3x_1)^T $
    \item $ F((x_1, x_2, x_3, x_4)^T) = (x_1, 0, 0, 0, x_2^2 + x_3^2 + x_4^2)^T $
\end{enumerate}

\noindent\textbf{Solution:} To check for linearity, we need to check for additivity and homogeneity, which implies that the zero vector is preserved.
That means a map is linear if $L(c(u + v)) = cL(u + v) = cL(u) + cL(v)$ for all $u, v$ in the domain and $c \in \mathbb{R}$.
\vspace{0.25cm}
\newline\noindent\textbf{a.} $ F((x_1, x_2, x_3)^T) = (x_1 - x_2, x_2 - x_1)^T $ \\
Additivity:
\begin{align*}
    F((x_1, x_2, x_3)^T + (y_1, y_2, y_3)^T) &= F((x_1 + y_1, x_2 + y_2, x_3 + y_3)^T) \\
    &= (x_1 + y_1 - x_2 - y_2, x_2 + y_2 - x_1 - y_1)^T \\
    &= (x_1 - x_2, x_2 - x_1)^T + (y_1 - y_2, y_2 - y_1)^T \\
    &= F((x_1, x_2, x_3)^T) + F((y_1, y_2, y_3)^T)
\end{align*}
Homogeneity:
\begin{align*}
    F(c(x_1, x_2, x_3)^T) &= F((cx_1, cx_2, cx_3)^T) \\
    &= (cx_1 - cx_2, cx_2 - cx_1)^T \\
    &= c(x_1 - x_2, x_2 - x_1)^T \\
    &= cF((x_1, x_2, x_3)^T)
\end{align*}
Therefore, $F((x_1, x_2, x_3)^T) = (x_1 - x_2, x_2 - x_1)^T$ is linear.

\vspace{0.25cm}
\noindent\textbf{b.} $ F((x_1, x_2, x_3)^T) = (1, 2, x_1 + x_2 + x_3)^T $ \\
Homogeneity:
\begin{align*}
    F((0, 0, 0)^T) &= (1, 2, 0)^T
\end{align*}
Since this transformation does not preserve the zero vector, it is not linear.

\vspace{0.25cm}
\noindent\textbf{c.} $ F((x_1)) = (x_1, 2x_1, 3x_1)^T $ \\
Additivity:
\begin{align*}
    F((x_1) + (y_1)) &= F((x_1 + y_1)) \\
    &= (x_1 + y_1, 2(x_1 + y_1), 3(x_1 + y_1))^T \\
    &= (x_1, 2x_1, 3x_1)^T + (y_1, 2y_1, 3y_1)^T \\
    &= F((x_1)) + F((y_1))
\end{align*}
Homogeneity:
\begin{align*}
    F(c(x_1)) &= F((cx_1)) \\
    &= (cx_1, 2(cx_1), 3(cx_1))^T \\
    &= c(x_1, 2x_1, 3x_1)^T \\
    &= cF((x_1))
\end{align*}
Therefore, $F((x_1)) = (x_1, 2x_1, 3x_1)^T$ is linear.

\vspace{0.25cm}
\noindent\textbf{d.} $ F((x_1, x_2, x_3, x_4)^T) = (x_1, 0, 0, 0, x_2^2 + x_3^2 + x_4^2)^T $\\
Since this transformation includes squared terms, it cannot satisfy additivity, and therefore is not linear.

\section*{Question 2}
Determine whether the following transformations are linear from $C([0,1])$ to $\mathbb{R}$.
\begin{enumerate}[a.]
    \item $ L(f) = f(0), (L := C([0,1]) \to \mathbb{R}) $
    \item $ L(f) = |f(0)|, (L := C([0,1]) \to \mathbb{R}) $
    \item $ L(f) = f'(0) + f(0). \; (L := C^1([0,1]) \to \mathbb{R}) $
    \item $ L(f)(x) = x^2 + f(x), \; (L := C([0,1]) \to C([0,1])) $
\end{enumerate}

\noindent\textbf{Solution:} Linear maps must satisfy additivity and homogeneity.

\vspace{0.25cm}
\noindent\textbf{a.} $ L(f) = f(0), (L := C([0,1]) \to \mathbb{R}) $ \\
Additivity:
\begin{align*}
    L(f + g) &= (f + g)(0) \\
    &= f(0) + g(0) \\
    &= L(f) + L(g)
\end{align*}

\noindent Homogeneity:
\begin{align*}
    L(cf) &= (cf)(0) \\
    &= cf(0) \\
    &= cL(f)
\end{align*}

\noindent The transformation $L(f) = f(0)$ preserves additivity and homogeneity, and therefore is linear.

\vspace{0.25cm}
\noindent\textbf{b.} $ L(f) = |f(0)|, (L := C([0,1]) \to \mathbb{R}) $ \\
Additivity:
\begin{align*}
    L(f + g) &= |(f + g)(0)| \\
    &= |f(0) + g(0)| \\
    &\neq |f(0)| + |g(0)| \\
\end{align*}

Counterexample: $f(x) = x + 1, \; g(x) = x - 1$
\begin{align*}
    L(f + g) &= |(f + g)(0)| \\
    &= |f(0) + g(0)| \\
    &= |1 + (-1)| \\
    &= 0 \\
\end{align*}
\begin{align*}
    L(f) + L(g) &= |f(0)| + |g(0)| \\
    &= |1| + |-1| \\
    &= 2 \\
\end{align*}

Since additivity is not preserved, this transformation is not linear.

\vspace{0.25cm}
\noindent\textbf{c.} $ L(f) = f'(0) + f(0). \; (L := C^1([0,1]) \to \mathbb{R}) $ \\
Additivity:
\begin{align*}
    L(f + g) &= (f + g)'(0) + (f + g)(0) \\
    &= f'(0) + g'(0) + f(0) + g(0) \\
    &= f'(0) + f(0) + g'(0) + g(0) \\
    &= L(f) + L(g)
\end{align*}

\noindent Homogeneity:
\begin{align*}
    L(cf) &= (cf)'(0) + (cf)(0) \\
    &= cf'(0) + cf(0) \\
    &= c(f'(0) + f(0)) \\
    &= cL(f)
\end{align*}

\noindent The transformation $L(f) = f'(0) + f(0)$ preserves additivity and homogeneity, and therefore is linear.

\vspace{0.25cm}
\noindent\textbf{d.} $ L(f)(x) = x^2 + f(x), \; (L := C([0,1]) \to C([0,1])) $ \\
Since $x^2$ is a constant in the domain, that means the zero vector cannot be preserved, and therefore this transformation is not linear.

\newpage
\section*{Question 3}
For each of the following transformations, find a matrix $A$ such that $L(x) = Ax$.
\begin{enumerate}[a.]
    \item $ L((x_1, x_2, x_3)^T) = (x_1 + x_2)^T $
    \item $ L((x_1, x_2, x_3)^T) = (x_1 + x_2, x_2 + x_3, x_1 + x_2 + x_3)^T $
    \item $ L((x_1)) = (x_1, 2x_1, 3x_1)^T $
    \item $ L((x_1, x_2, x_3, x_4)^T) = (x_1 + x_2 + x_3 + 2x_4)^T $
\end{enumerate}

\noindent\textbf{Solution:} To find a matrix A such that $L(x) = Ax$, we need to find the image of the standard basis vectors.

\vspace{0.25cm}
\noindent\textbf{a.} $ L((x_1, x_2, x_3)^T) = (x_1 + x_2)^T $ \\
$$ L((1, 0, 0)^T) = (1, 0)^T, \quad L((0, 1, 0)^T) = (1, 0)^T, \quad L((0, 0, 1)^T) = (0, 0)^T $$
$$ A = \begin{bmatrix} 1 & 1 & 0 \\ 0 & 0 & 0 \end{bmatrix} $$

\vspace{0.25cm}
\noindent\textbf{b.} $ L((x_1, x_2, x_3)^T) = (x_1 + x_2, x_2 + x_3, x_1 + x_2 + x_3)^T $ \\
$$ L((1, 0, 0)^T) = (1, 0, 1)^T, \quad L((0, 1, 0)^T) = (1, 1, 1)^T, \quad L((0, 0, 1)^T) = (0, 1, 1)^T $$
$$ A = \begin{bmatrix} 1 & 1 & 0 \\ 0 & 1 & 1 \\ 1 & 1 & 1 \end{bmatrix} $$

\vspace{0.25cm}
\noindent\textbf{c.} $ L((x_1)) = (x_1, 2x_1, 3x_1)^T $ \\
$$ L((1)) = (1, 2, 3)^T $$
$$ A = \begin{bmatrix} 1 \\ 2 \\ 3 \end{bmatrix} $$

\vspace{0.25cm}
\noindent\textbf{d.} $ L((x_1, x_2, x_3, x_4)^T) = (x_1 + x_2 + x_3 + 2x_4)^T $ \\
$$ L((1, 0, 0, 0)^T) = (1)^T, \quad L((0, 1, 0, 0)^T) = (1)^T $$
$$ L((0, 0, 1, 0)^T) = (1)^T, \quad L((0, 0, 0, 1)^T) = (2)^T $$
$$ A = \begin{bmatrix} 1 & 1 & 1 & 2 \end{bmatrix} $$


\section*{Question 4}
Let $L : \mathbb{R}^3 \to \mathbb{R}^2$ such that
$$ L((x_1, x_2, x_3)^T) = (2x_1, x_1 + x_2). $$
\begin{enumerate}[a.]
    \item Find $A$ that represents $L$ with respect to the standard basis of $\mathbb{R}^3$.
    \item Find $B$ that represents $L$ with respect to the following basis of $\mathbb{R}^3$. \\
    $E := [v_1, v_2, v_3]$, where, $$ v_1 = (1,1,1)^T, \quad v_2 = (1,1,0)^T, \quad v_3 = (1,0,0)^T. $$
\end{enumerate}

\noindent\textbf{Solution:}
\newline\noindent\textbf{a.} To find a matrix A such that $L(x) = Ax$, we need to find the image of the standard basis vectors.
$$ L((1, 0, 0)^T) = (2, 1)^T, \quad L((0, 1, 0)^T) = (0, 1)^T, \quad L((0, 0, 1)^T) = (0, 0)^T $$
$$ A = \begin{bmatrix} 2 & 0 & 0 \\ 1 & 1 & 0 \end{bmatrix} $$
\noindent\textbf{b.} To find a matrix B such that $L(x) = Bx$ where $E$ is a basis of $\mathbb{R}^3$, we need to find the image of $v_1, v_2, v_3$.
$$ L((1, 1, 1)^T) = (2, 2)^T, \quad L((1, 1, 0)^T) = (2, 2)^T, \quad L((1, 0, 0)^T) = (2, 1)^T $$
$$ B = \begin{bmatrix} 2 & 2 & 2 \\ 2 & 2 & 1 \end{bmatrix} $$

\newpage
\section*{Question 5}
In the vector space $C[-\pi, \pi]$ we define inner product
$$ \langle f, g \rangle = \frac{1}{\pi} \int_{-\pi}^{\pi} f(x)g(x)dx. $$
\begin{enumerate}[a.]
    \item Show that the above is indeed an inner product.
    \item Show that $f(x) = \cos(x), \; g(x) = \sin(x)$ are orthogonal and that they have length 1.
\end{enumerate}

\noindent\textbf{Solution:}

\vspace{0.25cm}
\noindent\textbf{a.} To show that the above is an inner product, we need to show that it satisfies the following properties:
\begin{enumerate}[i.]
    \item $\langle av_1 + bv_2, v_3 \rangle = a\langle v_1, v_3 \rangle + b\langle v_2, v_3 \rangle$
    \item $\langle v_1, v_2 \rangle = \langle v_2, v_1 \rangle$
    \item $\langle v_1, v_1 \rangle \geq 0$ and $\langle v_1, v_1 \rangle = 0$ if and only if $v_1 = 0$
\end{enumerate}

\noindent For the first property: \\
\begin{align*}
    \langle af + bg, h \rangle &= \frac{1}{\pi} \int_{-\pi}^{\pi} (af(x) + bg(x))h(x)\;dx \\
    &= \frac{1}{\pi} \int_{-\pi}^{\pi} af(x)h(x) + bg(x)h(x)\;dx \\
    &= \frac{1}{\pi} \int_{-\pi}^{\pi} af(x)h(x)\;dx + \frac{1}{\pi} \int_{-\pi}^{\pi} bg(x)h(x)\;dx \\
    &= a \frac{1}{\pi} \int_{-\pi}^{\pi} f(x)h(x)\;dx + b \frac{1}{\pi} \int_{-\pi}^{\pi} g(x)h(x)\;dx \\
\end{align*}
\begin{align*}
    a \langle f, h \rangle + b \langle g, h \rangle &= a \frac{1}{\pi} \int_{-\pi}^{\pi} f(x)h(x)\;dx + b \frac{1}{\pi} \int_{-\pi}^{\pi} g(x)h(x)\;dx \\
\end{align*}

\noindent For the second property: \\
\begin{align*}
    \langle f, g \rangle &= \frac{1}{\pi} \int_{-\pi}^{\pi} f(x)g(x)\;dx \\
    &= \frac{1}{\pi} \int_{-\pi}^{\pi} g(x)f(x)\;dx \\
    &= \langle g, f \rangle
\end{align*}


\noindent For the third property: \\
\begin{align*}
    \langle f, f \rangle &= \frac{1}{\pi} \int_{-\pi}^{\pi} f(x)f(x)\;dx \\
    &= \frac{1}{\pi} \int_{-\pi}^{\pi} f(x)^2\;dx \\
\end{align*}
We want to show that $\frac{1}{\pi} \int_{-\pi}^{\pi} f(x)^2\;dx$ is positive for all $f \in \mathbb{C}[-\pi, \pi]$ and that it is zero if and only if $f = 0$.
The integrand $f(x)^2$ must be greater than or equal to zero, since it is the square of a real number.
Additionally, we have that $f(x)^2 = 0$ if and only if $f(x) = 0$.
Since the integrand is greater than or equal to zero, and is zero if and only if $f(x) = 0$, then the integral must be greater than or equal to zero, and is zero if and only if $f(x) = 0$.
Therefore, $\langle f, f \rangle \geq 0$ and $\langle f, f \rangle = 0$ if and only if $f = 0$,
and the above is an inner product.

\vspace{0.25cm}
\noindent\textbf{b.} To show that $f(x) = \cos(x), \; g(x) = \sin(x)$ are orthogonal, we need to show that $\langle f, g \rangle = 0$.
The inner product of $f$ and $g$ is:
\begin{align*}
    \langle f, g \rangle &= \frac{1}{\pi} \int_{-\pi}^{\pi} f(x)g(x)\;dx = \frac{1}{\pi} \int_{-\pi}^{\pi} \cos(x)\sin(x)\;dx \\
    &= \frac{1}{\pi} \int_{-\pi}^{\pi} \sin(x) \; dx \int_{-\pi}^{\pi} \cos(x) \; dx \\
    &= \frac{1}{\pi} \left[ -\cos(x) \right]_{-\pi}^{\pi} \left[ \sin(x) \right]_{-\pi}^{\pi} \\
    &= \frac{1}{\pi} (-1 - (-1)) (0 - 0) \\
    &= 0
\end{align*}
Since the inner product of $f$ and $g$ is zero, $f$ and $g$ are orthogonal. \\
To show that $f$ and $g$ have length 1, we can take the euclidian norm of $f$ and $g$, which is just the square root of their inner product with themselves.
\begin{align*}
    \langle f, f \rangle &= \frac{1}{\pi} \int_{-\pi}^{\pi} f(x)f(x)\;dx = \frac{1}{\pi} \int_{-\pi}^{\pi} \cos(x)^2\;dx \\
    &= \frac{1}{2 \pi} \int_{-\pi}^{\pi} 1 + \cos(2x)\;dx \\
    &= \frac{1}{2 \pi} \left[ x + \frac{1}{2} \sin(2x) \right]_{-\pi}^{\pi} = \frac{1}{2 \pi} \left( \pi + \frac{1}{2} \sin(2\pi) - (-\pi + \frac{1}{2} \sin(-2\pi)) \right) \\
    &= \frac{1}{2 \pi} \left( \pi + 0 - (-\pi + 0) \right) = \frac{1}{2 \pi} (2\pi) = 1 \\
    \sqrt{\langle f, f \rangle} &= \sqrt{1} = 1
\end{align*}
The inner product of $f$ and $f$ is 1, so the euclidian norm of $f$ is 1.
\begin{align*}
    \langle g, g \rangle &= \frac{1}{\pi} \int_{-\pi}^{\pi} g(x)g(x)\;dx = \frac{1}{\pi} \int_{-\pi}^{\pi} \sin(x)^2\;dx \\
    &= \frac{1}{2 \pi} \int_{-\pi}^{\pi} 1 - \cos(2x)\;dx \\
    &= \frac{1}{2 \pi} \left[ x - \frac{1}{2} \sin(2x) \right]_{-\pi}^{\pi} = \frac{1}{2 \pi} \left( \pi - \frac{1}{2} \sin(2\pi) - (-\pi - \frac{1}{2} \sin(-2\pi)) \right) \\
    &= \frac{1}{2 \pi} \left( \pi + 0 - (-\pi - 0) \right) = \frac{1}{2 \pi} (2\pi) = 1 \\
    \sqrt{\langle g, g \rangle} &= \sqrt{1} = 1
\end{align*}
The inner product of $g$ and $g$ is 1, so the euclidian norm of $g$ is 1.
Therefore, $f$ and $g$ have length 1.

\end{document}
