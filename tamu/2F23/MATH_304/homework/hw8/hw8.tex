\documentclass{article}
\usepackage{amsmath,amssymb,amsthm}
\usepackage{fancyhdr}
\usepackage{enumerate}

\pagestyle{fancy}
\fancyhf{}
\lhead{8th Homework - MATH 304 508}
\rhead{Kevin Lei}
\renewcommand{\headrulewidth}{0.4pt}
\renewcommand{\arraystretch}{1.2}

\begin{document}

\section*{Question 1}
Let $\{u_1, u_2, u_3\}$ be an orthonormal set of vectors in some vector space with inner product.
Let $$ u:= u_1 + 2u_2 + 3u_3 \text{ and } v:= u_1 - u_3 $$
Compute $\langle u, v \rangle$, $\|u\|$, and $\|v\|$.

\vspace{0.5cm}
\noindent\textbf{Solution:} Since the basis is orthonormal, the inner product of any two vectors in the basis is 0,
and the inner product of a vector in the basis with itself is 1.
\begin{align*}
    \langle u, v \rangle &= \langle u_1 + 2u_2 + 3u_3, u_1 - u_3 \rangle \\
    &= \langle u_1, u_1 - u_3 \rangle + \langle 2u_2, u_1 - u_3 \rangle + \langle 3u_3, u_1 - u_3 \rangle \\
    &= \langle u_1, u_1 \rangle - \langle u_1, u_3 \rangle + \langle 2u_2, u_1 \rangle - \langle 2u_2, u_3 \rangle + \langle 3u_3, u_1 \rangle - \langle 3u_3, u_3 \rangle \\
    &= 1 - 0 + 0 - 0 + 0 - 3 \\
    &= -2 \\
    \\
    \|u\|^2 &= \langle u, u \rangle \\
    &= \langle u_1 + 2u_2 + 3u_3, u_1 + 2u_2 + 3u_3 \rangle \\
    &= \langle u_1, u_1 + 2u_2 + 3u_3 \rangle + \langle 2u_2, u_1 + 2u_2 + 3u_3 \rangle + \langle 3u_3, u_1 + 2u_2 + 3u_3 \rangle \\
    &= \langle u_1, u_1 \rangle + \langle 2u_2, 2u_2 \rangle + \langle 3u_3, 3u_3 \rangle \\
    &= 1 + 4 + 9 \\
    &= 14 \Longrightarrow \|u\| = \sqrt{14} \\
    \\
    \|v\|^2 &= \langle v, v \rangle \\
    &= \langle u_1 - u_3, u_1 - u_3 \rangle \\
    &= \langle u_1, u_1 - u_3 \rangle + \langle -u_3, u_1 - u_3 \rangle \\
    &= \langle u_1, u_1 \rangle - \langle u_1, u_3 \rangle - \langle u_3, u_1 \rangle + \langle u_3, u_3 \rangle \\
    &= 1 - 0 - 0 + 1 \\
    &= 2 \Longrightarrow \|v\| = \sqrt{2} \\
\end{align*}

\newpage
\section*{Question 2}
Consider the vector space $C[-1, 1]$ equipped with the inner product:
$$ \langle f, g \rangle := \int_{-1}^1 f(x)g(x) dx $$
1. Show that $1, x$ are orthogonal.\\
2. Compute the norms $\|1\|$, $\|x\|$.\\

\vspace{0.5cm}
\noindent\textbf{Solution:} Two vectors are orthogonal if their inner product is 0.
\begin{align*}
    \langle 1, x \rangle &= \int_{-1}^1 1 \cdot x \; dx = \int_{-1}^1 x \; dx = \frac{1}{2}x^2 \Big|_{-1}^1 = \frac{1}{2} - \frac{1}{2} = 0 \\
\end{align*}
Thus, $1, x$ are orthogonal.
The norms of $1$ and $x$ are the square root of their inner product with themselves.
\begin{align*}
    \|1\|^2 &= \langle 1, 1 \rangle = \int_{-1}^1 1 \cdot 1 \; dx = \int_{-1}^1 1 \; dx = x \Big|_{-1}^1 = 1 - (-1) = 2 \\
    &\Longrightarrow \|1\| = \sqrt{2} \\
    \|x\|^2 &= \langle x, x \rangle = \int_{-1}^1 x \cdot x \; dx = \int_{-1}^1 x^2 \; dx = \frac{1}{3}x^3 \Big|_{-1}^1 = \frac{1}{3} - \frac{1}{3} = \frac{2}{3} \\
    &\Longrightarrow \|x\| = \sqrt{\frac{2}{3}} \\
\end{align*}

\newpage
\section*{Question 3}
Let $$ u_1 = \left( \frac{1}{3\sqrt{2}}, \frac{1}{3\sqrt{2}}, -\frac{4}{3\sqrt{2}}\right)^T, \; u_2 = \frac{1}{3}(2, 2, 1)^T, \; u_3 = \frac{1}{\sqrt{2}}(1, -1, 0)^T $$
1. Show that $u_1, u_2, u_3$ is an orthonormal basis for $\mathbb{R}^3$.\\
2. Let $x=(1, 2, 2)^T$. Find the projection of $p$ of $x$ onto $S := \text{span}\{u_2, u_3\}$.

\vspace{0.5cm}
\noindent\textbf{Solution:} A set of vectors form an orthonormal basis if they are orthogonal and their norms are 1.
Since we are working in $\mathbb{R}^3$, we can use the dot product to check if the vectors are orthogonal.
\begin{align*}
    \langle u_1, u_2 \rangle &= \frac{1}{3\sqrt{2}} \cdot \frac{2}{3} + \frac{1}{3\sqrt{2}} \cdot \frac{2}{3} + -\frac{4}{3\sqrt{2}} \cdot \frac{1}{3}
    = \frac{2}{9\sqrt{2}} + \frac{2}{9\sqrt{2}} - \frac{4}{9\sqrt{2}} = 0 \\
    \langle u_1, u_3 \rangle &= \frac{1}{3\sqrt{2}} \cdot \frac{1}{\sqrt{2}} + \frac{1}{3\sqrt{2}} \cdot -\frac{1}{\sqrt{2}} + -\frac{4}{3\sqrt{2}} \cdot 0 = \frac{1}{6} - \frac{1}{6} = 0 \\
        \langle u_2, u_3 \rangle &= \frac{2}{3} \cdot \frac{1}{\sqrt{2}} + \frac{2}{3} \cdot -\frac{1}{\sqrt{2}} + \frac{1}{3} \cdot 0 = \frac{2}{3\sqrt{2}} - \frac{2}{3\sqrt{2}} = 0
\end{align*}
Thus, $u_1, u_2, u_3$ are orthogonal.
To check if their norms are 1, we can use the formula $\|u\|^2 = \langle u, u \rangle$.
\begin{align*}
    \|u_1\|^2 &= \langle u_1, u_1 \rangle = \frac{1}{3\sqrt{2}} \cdot \frac{1}{3\sqrt{2}} + \frac{1}{3\sqrt{2}} \cdot \frac{1}{3\sqrt{2}} + -\frac{4}{3\sqrt{2}} \cdot -\frac{4}{3\sqrt{2}} = \frac{1}{18} + \frac{1}{18} + \frac{16}{18} = 1 \\
    \|u_2\|^2 &= \langle u_2, u_2 \rangle = \frac{1}{3} \cdot \frac{1}{3} + \frac{1}{3} \cdot \frac{1}{3} + 1 \cdot 1 = \frac{1}{9} + \frac{1}{9} + 1 = 1 \\
    \|u_3\|^2 &= \langle u_3, u_3 \rangle = \frac{1}{\sqrt{2}} \cdot \frac{1}{\sqrt{2}} + -\frac{1}{\sqrt{2}} \cdot -\frac{1}{\sqrt{2}} + 0 \cdot 0 = \frac{1}{2} + \frac{1}{2} = 1
\end{align*}
Since the squares of norms are 1, the norms are 1. Thus, $u_1, u_2, u_3$ are orthonormal.
To find the projection of $x$ onto $S$, we can use the projection matrix:
$$ p = \text{proj}_S(x) = A (A^T A)^{-1} A^T x $$
We form the matrix A by taking the basis vectors of $S$ and using them as columns.
Since we have that $S$ is spanned by $u_2$ and $u_3$, we have the following:

$$
A = \begin{bmatrix} \frac{2}{3} & \frac{1}{\sqrt{2}} \\ \frac{2}{3} & -\frac{1}{\sqrt{2}} \\ \frac{1}{3} & 0 \end{bmatrix}
, \quad A^T = \begin{bmatrix} \frac{2}{3} & \frac{2}{3} & \frac{1}{3} \\ \frac{1}{\sqrt{2}} & -\frac{1}{\sqrt{2}} & 0 \end{bmatrix}
$$
Now, we need to find $A^T A$ and $(A^T A)^{-1}$.
\begin{align*}
    A^T A =
    \begin{bmatrix} \frac{2}{3} & \frac{2}{3} & \frac{1}{3} \\ \frac{1}{\sqrt{2}} & -\frac{1}{\sqrt{2}} & 0 \end{bmatrix}
    \begin{bmatrix} \frac{2}{3} & \frac{1}{\sqrt{2}} \\ \frac{2}{3} & -\frac{1}{\sqrt{2}} \\ \frac{1}{3} & 0 \end{bmatrix}
    = \begin{bmatrix} \frac{4}{9} + \frac{4}{9} + \frac{1}{9} & \frac{2}{3\sqrt{2}} - \frac{2}{3\sqrt{2}} \\ \frac{2}{3\sqrt{2}} - \frac{2}{3\sqrt{2}} & \frac{1}{2} + \frac{1}{2} \end{bmatrix}
    = \begin{bmatrix} 1 & 0 \\ 0 & 1 \end{bmatrix}
\end{align*}
Since $A^T A$ is just the identity matrix, its inverse is also the identity matrix.

$$ (A^T A)^{-1} = \begin{bmatrix} 1 & 0 \\ 0 & 1 \end{bmatrix} $$
Now, we have:

\begin{align*}
    p &= A (A^T A)^{-1} A^T x = A I A^T x = A A^T x \\
    &= \begin{bmatrix} \frac{2}{3} & \frac{1}{\sqrt{2}} \\ \frac{2}{3} & -\frac{1}{\sqrt{2}} \\ \frac{1}{3} & 0 \end{bmatrix}
        \begin{bmatrix} \frac{2}{3} & \frac{2}{3} & \frac{1}{3} \\ \frac{1}{\sqrt{2}} & -\frac{1}{\sqrt{2}} & 0 \end{bmatrix}
        \begin{bmatrix} 1 \\ 2 \\ 2 \end{bmatrix} \\
    &= \begin{bmatrix} \frac{4}{9} + \frac{1}{2} & \frac{4}{9} - \frac{1}{2} & \frac{2}{9} \\
        \frac{4}{9} - \frac{1}{2} & \frac{4}{9} + \frac{1}{2} & \frac{2}{9} \\
        \frac{2}{9} & \frac{2}{9} & \frac{1}{9} \end{bmatrix}
        \begin{bmatrix} 1 \\ 2 \\ 2 \end{bmatrix}
    =   \begin{bmatrix} \frac{17}{18} & \frac{1}{18} & \frac{2}{9} \\
        \frac{1}{18} & \frac{17}{18} & \frac{2}{9} \\
        \frac{2}{9} & \frac{2}{9} & \frac{1}{9} \end{bmatrix}
    \begin{bmatrix} 1 \\ 2 \\ 2 \end{bmatrix}
    = \frac{1}{18} \begin{bmatrix} 17 & 1 & 4 \\ 1 & 17 & 4 \\ 4 & 4 & 2 \end{bmatrix}
        \begin{bmatrix} 1 \\ 2 \\ 2 \end{bmatrix} \\
    &= \frac{1}{18} \begin{bmatrix} 17 + 2 + 8 \\ 1 + 34 + 8 \\ 4 + 8 + 4 \end{bmatrix}
    = \frac{1}{18} \begin{bmatrix} 27 \\ 43 \\ 16 \end{bmatrix}
\end{align*}
Simplifying further:
\begin{align*}
    p &= \frac{1}{18} \begin{bmatrix} 27 \\ 43 \\ 16 \end{bmatrix}
    =\begin{bmatrix}
        \frac{3}{2} \\
        \frac{43}{18} \\
        \frac{8}{9}
    \end{bmatrix}
\end{align*}

\newpage
\section*{Question 4}
Let $v_1 := (1, 2, 0, -1)^T \; v_2 := (1, -1, 0, 0)^T \; v_3 := (0, 1, 0, -1)^T$.
Find the angle between $v_1, v_2$, $v_2, v_3$, and $v_1, v_3$.
Find the norm of each of these vectors.
Find the projection of $v_1$ onto $v_2$ and onto $v_3$.

\vspace{0.5cm}
\noindent\textbf{Solution:} For two vectors $v_1$ and $v_2$ in a vector space with inner product, the angle $\theta$ between them is given by:
$$ \cos \theta = \frac{\langle v_1, v_2 \rangle}{\|v_1\| \|v_2\|} $$
Finding the angles between $v_1, v_2$, $v_2, v_3$, and $v_1, v_3$:
\begin{align*}
    \cos \theta_{v_1, v_2} &= \frac{\langle v_1, v_2 \rangle}{\|v_1\| \|v_2\|} 
                            = \frac{1 \cdot 1 + 2 \cdot -1 + 0 \cdot 0 + -1 \cdot 0}{\sqrt{1^2 + 2^2 + 0^2 + (-1)^2} \sqrt{1^2 + (-1)^2 + 0^2 + 0^2}}
                            = \frac{1 - 2}{\sqrt{6} \sqrt{2}} = -\frac{1}{\sqrt{12}} \\
    \theta_{v_1, v_2} &= \cos^{-1} \left( -\frac{1}{\sqrt{12}} \right) \\
    \\
    \cos \theta_{v_2, v_3} &= \frac{\langle v_2, v_3 \rangle}{\|v_2\| \|v_3\|}
                            = \frac{1 \cdot 0 + -1 \cdot 1 + 0 \cdot 0 + 0 \cdot -1}{\sqrt{1^2 + (-1)^2 + 0^2 + 0^2} \sqrt{0^2 + 1^2 + 0^2 + (-1)^2}}
                            = \frac{-1}{\sqrt{2} \sqrt{2}} = -\frac{1}{2} \\
    \theta_{v_2, v_3} &= \cos^{-1} \left( -\frac{1}{2} \right) = \frac{2\pi}{3} \\
    \\
    \cos \theta_{v_1, v_3} &= \frac{\langle v_1, v_3 \rangle}{\|v_1\| \|v_3\|}
                            = \frac{1 \cdot 0 + 2 \cdot 1 + 0 \cdot 0 + -1 \cdot -1}{\sqrt{1^2 + 2^2 + 0^2 + (-1)^2} \sqrt{0^2 + 1^2 + 0^2 + (-1)^2}}
                            = \frac{2 + 1}{\sqrt{6} \sqrt{2}} = \frac{3}{\sqrt{12}} = \frac{\sqrt{3}}{2} \\
    \theta_{v_1, v_3} &= \cos^{-1} \left( \frac{\sqrt{3}}{2} \right) = \frac{\pi}{6}
\end{align*}
The norm of these vectors is given by $\|v\| = \sqrt{\langle v, v \rangle}$.
\begin{align*}
    \|v_1\| &= \sqrt{\langle v_1, v_1 \rangle} = \sqrt{1^2 + 2^2 + 0^2 + (-1)^2} = \sqrt{6} \\
    \|v_2\| &= \sqrt{\langle v_2, v_2 \rangle} = \sqrt{1^2 + (-1)^2 + 0^2 + 0^2} = \sqrt{2} \\
    \|v_3\| &= \sqrt{\langle v_3, v_3 \rangle} = \sqrt{0^2 + 1^2 + 0^2 + (-1)^2} = \sqrt{2}
\end{align*}
The projection of one vector onto another is given by:
$$ \text{proj}_u(v) = \frac{\langle u, v \rangle}{\|u\|^2} u $$
Thus we have:
\begin{align*}
    \text{proj}_{v_2}(v_1) &= \frac{\langle v_2, v_1 \rangle}{\|v_2\|^2} v_2 \\
                           &= \frac{1 \cdot 1 + 2 \cdot -1 + 0 \cdot 0 + -1 \cdot 0}{\sqrt{1^2 + (-1)^2 + 0^2 + 0^2}^2} \begin{bmatrix} 1 \\ -1 \\ 0 \\ 0 \end{bmatrix}
                           = \frac{-1}{2} \begin{bmatrix} 1 \\ -1 \\ 0 \\ 0 \end{bmatrix} \\
                            &= \begin{bmatrix} -\frac{1}{2} \\ \frac{1}{2} \\ 0 \\ 0 \end{bmatrix} \\
    \\
    \text{proj}_{v_3}(v_1) &= \frac{\langle v_3, v_1 \rangle}{\|v_3\|^2} v_3 \\
                            &= \frac{1 \cdot 0 + 2 \cdot 1 + 0 \cdot 0 + -1 \cdot -1}{\sqrt{0^2 + 1^2 + 0^2 + (-1)^2}^2} \begin{bmatrix} 0 \\ 1 \\ 0 \\ -1 \end{bmatrix}
                            = \frac{3}{2} \begin{bmatrix} 0 \\ 1 \\ 0 \\ -1 \end{bmatrix} \\
                             &= \begin{bmatrix} 0 \\ \frac{3}{2} \\ 0 \\ -\frac{3}{2} \end{bmatrix}
\end{align*}

\newpage
\section*{Question 5}
Let $A$ be an $m \times n$ matrix. Show that $A^T A$ and $AA^T$ is a symmetric matrix.
Assume that $m \geq n$ and $\text{rank}(A) = n$.
Show that if $P=A(A^T A)^{-1}A^T$ then $$ P^2 = P $$

\vspace{0.5cm}
\noindent\textbf{Solution:} A matrix is symmetric if it is equal to its transpose.
Using the properties of the matrix transpose, we can show that $A^T A$ and $AA^T$ are symmetric.
\begin{align*}
    &\text{Transpose of $A^T A$: } (A^T A)^T = (A^T)^T A^T = A A^T = A^T A \\
    &\text{Transpose of $A A^T$: } (AA^T)^T = A^T (A^T)^T = A^T A = A A^T
\end{align*}
As we can see here, applying the transpose to either matrix results in the same matrix.
Proving that $P^2 = P$ is a bit more involved.
\begin{proof}
    We have that $P = A(A^T A)^{-1}A^T$.
    We need to find $$P^2 = A(A^T A)^{-1}A^T A(A^T A)^{-1}A^T$$ and show that it is equal to $P$.
    Using the associative property of matrix multiplication, we can change our order of multiplication to get
    $$P^2 = A(A^T A)^{-1}(A^T A)(A^T A)^{-1}A^T$$
    Since it is given that rank$(A) = n$, and $A^T A$ must be and $n \times n$ matrix, $A^T A$ is invertible.
    Since $A^T A$ is invertible, we can multiply it by its inverse to get the identity matrix.
    Now, we have
    $$P^2 = A I (A^T A)^{-1}A^T = A (A^T A)^{-1}A^T = P$$
    Thus, $P^2 = P$.
\end{proof}

\end{document}
