\documentclass{article}
\usepackage{amsmath,amssymb,amsthm}
\usepackage{fancyhdr}
\usepackage{enumerate}

\pagestyle{fancy}
\fancyhf{}
\lhead{6th Homework - MATH 304 508}
\rhead{Kevin Lei}
\renewcommand{\headrulewidth}{0.4pt}

\begin{document}

\section*{Question 1}
Let
$$ u_1 := (1, 1, 1)^{T}, u_2 := (1, 2, 2)^{T}, u_3 := (3, 2, 4)^{T} $$
$$ v_1 := (4, 6, 7)^{T}, v_2 := (0, 1, 1)^{T}, v_3 := (0, 1, 2)^{T}$$

\begin{enumerate}[(a)]
    \item
    Find the transition matrix corresponding to the change of basis from ${e_1, e_2, e_3}$ to ${u_1, u_2, u_3}$.
\end{enumerate}

\noindent\textbf{Solution:}
Let $U$ be the transition matrix from the standard basis to the basis ${u_1, u_2, u_3}$.

$$ U = \begin{bmatrix} 1 & 1 & 3 \\ 1 & 2 & 2 \\ 1 & 2 & 4 \end{bmatrix} $$

\noindent
To find the transition matrix from ${e_1, e_2, e_3}$ to ${u_1, u_2, u_3}$, we need to find the inverse of $U$.

$$
\left[\begin{array}{ccc|ccc} 1 & 1 & 3 & 1 & 0 & 0 \\ 1 & 2 & 2 & 0 & 1 & 0 \\ 1 & 2 & 4 & 0 & 0 & 1 \end{array}\right] 
\sim 
\left[\begin{array}{ccc|ccc} 1 & 1 & 3 & 1 & 0 & 0 \\ 0 & 1 & -1 & -1 & 1 & 0 \\ 0 & 1 & 1 & -1 & 0 & 1 \end{array}\right]
\sim
\left[\begin{array}{ccc|ccc} 1 & 0 & 4 & 2 & -1 & 0 \\ 0 & 1 & -1 & -1 & 1 & 0 \\ 0 & 0 & 2 & 0 & -1 & 1 \end{array}\right]
$$
$$
\sim
\left[\begin{array}{ccc|ccc} 1 & 0 & 0 & 2 & 1 & -2 \\ 0 & 1 & 0 & -1 & \frac{1}{2} & \frac{1}{2} \\ 0 & 0 & 1 & 0 & -\frac{1}{2} & \frac{1}{2} \end{array}\right]
$$

\noindent
Thus, we have $U^{-1}$ as:
$$ U^{-1} = \begin{bmatrix} 2 & 1 & -2 \\ -1 & \frac{1}{2} & \frac{1}{2} \\ 0 & -\frac{1}{2} & \frac{1}{2} \end{bmatrix} $$
\noindent
which is the transition matrix from ${e_1, e_2, e_3}$ to ${u_1, u_2, u_3}$.
\newline

\begin{enumerate}[(b)]
    \item
    Find the transition matrix corresponding to the change of basis from ${v_1, v_2, v_3}$ to ${e_1, e_2, e_3}$.
\end{enumerate}

\noindent\textbf{Solution:}
Let $V$ be the transition matrix from the basis ${v_1, v_2, v_3}$ to the standard basis.
$V$ is simply the matrix of the basis vectors ${v_1, v_2, v_3}$.
Thus, we have:
$$ V = \begin{bmatrix} 4 & 0 & 0 \\ 6 & 1 & 1 \\ 7 & 1 & 2 \end{bmatrix} $$

\newpage
\begin{enumerate}[(c)]
    \item
    Find the transition matrix from ${v_1, v_2, v_3}$ to ${u_1, u_2, u_3}$.
\end{enumerate}

\noindent\textbf{Solution:}
We want to find the transition matrix from ${v_1, v_2, v_3}$ to ${u_1, u_2, u_3}$.
This can be done by first finding the transition matrix from ${v_1, v_2, v_3}$ to the standard basis, and then multiplying it by the transition matrix from the standard basis to ${u_1, u_2, u_3}$.
This means the transition matrix from ${v_1, v_2, v_3}$ to ${u_1, u_2, u_3}$ is simply $U^{-1}V$.
Since we have already found $U^{-1}$ and $V$ in previous parts, we have:
$$ U^{-1}V = 
\begin{bmatrix} 2 & 1 & -2 \\ -1 & \frac{1}{2} & \frac{1}{2} \\ 0 & -\frac{1}{2} & \frac{1}{2} \end{bmatrix} 
\begin{bmatrix} 4 & 0 & 0 \\ 6 & 1 & 1 \\ 7 & 1 & 2 \end{bmatrix}
=
\begin{bmatrix} 0 & -1 & -3 \\ \frac{5}{2} & 1 & \frac{3}{2} \\ \frac{1}{2} & 0 & \frac{1}{2} \end{bmatrix}
$$
\newline

\begin{enumerate}[(d)]
    \item
    Let $x = 2v_1 + 3v_2 - 4v_3$. Find the coordinates of $x$ with respect to ${u_1, u_2, u_3}$.
\end{enumerate}

\noindent\textbf{Solution:}
The vector $x$ can be written as:
$$\begin{bmatrix} 2 \\ 3 \\ -4 \end{bmatrix}$$
\noindent
with respect to the basis ${v_1, v_2, v_3}$.
To find the coordinates of $x$ with respect to ${u_1, u_2, u_3}$, we can multiply the vector by the transition matrix from ${v_1, v_2, v_3}$ to ${u_1, u_2, u_3}$.
Thus, we have:
$$
U^{-1}Vx =
\begin{bmatrix} 0 & -1 & -3 \\ \frac{5}{2} & 1 & \frac{3}{2} \\ \frac{1}{2} & 0 & \frac{1}{2} \end{bmatrix}
\begin{bmatrix} 2 \\ 3 \\ -4 \end{bmatrix}
=
\begin{bmatrix} 9 \\ 2 \\ -1 \end{bmatrix}
$$

\noindent
Thus, the coordinates of the vector $x$ with respect to ${u_1, u_2, u_3}$ is $9u_1 + 2u_2 - u_3$.
\newline

\begin{enumerate}[(e)]
    \item
    Verify your answer to previous one, by computing the coordinates in each case with respect to the standard basis.
\end{enumerate}

\noindent
The vector $x$ can be written with respect to the basis $v_1, v_2, v_3$ as:
$$
x = 2\begin{bmatrix} 4 \\ 6 \\ 7 \end{bmatrix} + 3\begin{bmatrix} 0 \\ 1 \\ 1 \end{bmatrix} - 4\begin{bmatrix} 0 \\ 1 \\ 2 \end{bmatrix} 
= \begin{bmatrix} 8 \\ 12 \\ 14 \end{bmatrix} + \begin{bmatrix} 0 \\ 3 \\ 3 \end{bmatrix} - \begin{bmatrix} 0 \\ 4 \\ 8 \end{bmatrix} 
= \begin{bmatrix} 8 \\ 11 \\ 9 \end{bmatrix}
$$
\noindent
Here we have that the coordinates of $x$ with respect to the standard basis is $8e_1 + 11e_2 + 9e_3$.
Writing $x$ with respect to the basis $u_1, u_2, u_3$ gives:
$$
x = 9\begin{bmatrix} 1 \\ 1 \\ 1 \end{bmatrix} + 2\begin{bmatrix} 1 \\ 2 \\ 2 \end{bmatrix} - 1\begin{bmatrix} 3 \\ 2 \\ 4 \end{bmatrix}
= \begin{bmatrix} 9 \\ 9 \\ 9 \end{bmatrix} + \begin{bmatrix} 2 \\ 4 \\ 4 \end{bmatrix} - \begin{bmatrix} 3 \\ 2 \\ 4 \end{bmatrix}
= \begin{bmatrix} 8 \\ 11 \\ 9 \end{bmatrix}
$$
\noindent
Since the two coordinates are the same, we have verified our answer.

\newpage
\section*{Question 2}
Find a basis for the row space, column space and null space of the following matrices.
$$
A = \begin{bmatrix} 1 & 3 & 2 \\ 2 & 1 & 4 \\ 4 & 7 & 8 \end{bmatrix}
B = \begin{bmatrix} -3 & 1 & 3 & 4 \\ 1 & 2 & -1 & -2 \\ -3 & 8 & 4 & 2 \end{bmatrix}
C = \begin{bmatrix} 1 & 3 & -2 & 1 \\ 2 & 1 & 3 & 2 \\ 3 & 4 & 5 & 6 \end{bmatrix}
$$

\noindent\textbf{Solution:} \\
\noindent\textbf{Matrix A} \\
To find the row space, column space, and null space of the matrix, we can use gaussian elimination.
Using row operations, we have the following equivalent matrices:
$$
\begin{bmatrix} 1 & 3 & 2 \\ 2 & 1 & 4 \\ 4 & 7 & 8 \end{bmatrix}
\sim
\begin{bmatrix} 1 & 3 & 2 \\ 0 & -5 & 0 \\ 0 & -5 & 0 \end{bmatrix}
\sim
\begin{bmatrix} 1 & 3 & 2 \\ 0 & -5 & 0 \\ 0 & 0 & 0 \end{bmatrix}
$$
Now we have two linearly independent rows in row echelon form.
Thus, the row space of the matrix is spanned by the vectors:
$$ \text{rowspace}(A) = \text{span}\left\{ \begin{bmatrix} 1 \\ 3 \\ 2 \end{bmatrix}, \begin{bmatrix} 0 \\ -5 \\ 0 \end{bmatrix} \right\} $$

\noindent Identifying the columns of the leading 1's in the row echelon form, we have that the column space is spanned by the vectors:
$$ \text{colspace}(A) = \text{span}\left\{ \begin{bmatrix} 1 \\ 2 \\ 4 \end{bmatrix}, \begin{bmatrix} 3 \\ 1 \\ 7 \end{bmatrix} \right\} $$

\noindent The null space can be found by solving the homogeneous system:
$$
\left[\begin{array}{ccc|c} 1 & 3 & 2 & 0 \\ 0 & -5 & 0 & 0 \end{array}\right]
\sim
\left[\begin{array}{ccc|c} 1 & 0 & 2 & 0 \\ 0 & 1 & 0 & 0 \end{array}\right]
\rightarrow
\begin{cases} x_1 = -2x_3 \\ x_2 = 0 \\ x_3 = x_3 \end{cases}
$$

\noindent Thus, the null space is the following:
$$
N(A) = \{(-2a, 0, a)^T \mid a \in \mathbb{R}\} = \text{span}\left\{ \begin{bmatrix} -2 \\ 0 \\ 1 \end{bmatrix} \right\}
$$

\newpage
\noindent\textbf{Matrix B} \\
First, row reduce the matrix to row echelon form:
$$
\begin{bmatrix} -3 & 1 & 3 & 4 \\ 1 & 2 & -1 & -2 \\ -3 & 8 & 4 & 2 \end{bmatrix}
\sim
\begin{bmatrix} 1 & 2 & -1 & -2 \\ 0 & 7 & 0 & -2 \\ 0 & 14 & 1 & -4 \end{bmatrix}
\sim
\begin{bmatrix} 1 & 2 & -1 & -2 \\ 0 & 7 & 0 & -2 \\ 0 & 0 & 1 & 0 \end{bmatrix}
$$ $$
\sim
\begin{bmatrix} 1 & 2 & 0 & -2 \\ 0 & 1 & 0 & -\frac{2}{7} \\ 0 & 0 & 1 & 0 \end{bmatrix}
\sim
\begin{bmatrix} 1 & 0 & 0 & -\frac{10}{7} \\ 0 & 1 & 0 & -\frac{2}{7} \\ 0 & 0 & 1 & 0 \end{bmatrix}
$$

\noindent From here we can observe the following:
$$
\text{rowspace}(B) = \text{span}\left\{ \begin{bmatrix} -3 \\ 1 \\ 3 \\ 4 \end{bmatrix}, \begin{bmatrix} 1 \\ 2 \\ -1 \\ -2 \end{bmatrix}, \begin{bmatrix} -3 \\ 8 \\ 4 \\ 2 \end{bmatrix} \right\}
,\quad
\text{colspace}(B) = \text{span}\left\{ \begin{bmatrix} -3 \\ 1 \\ -3 \end{bmatrix}, \begin{bmatrix} 1 \\ 2 \\ 8 \end{bmatrix}, \begin{bmatrix} 3 \\ -1 \\ 4 \end{bmatrix} \right\}
$$

\noindent Solving for the null space:
$$
\left[\begin{array}{cccc|c} 1 & 0 & 0 & -\frac{10}{7} & 0 \\ 0 & 1 & 0 & -\frac{2}{7} & 0 \\ 0 & 0 & 1 & 0 & 0 \end{array}\right]
\rightarrow
\begin{cases} x_1 = \frac{10}{7}x_4 \\ x_2 = \frac{2}{7}x_4 \\ x_3 = 0 \\ x_4 = x_4 \end{cases}
$$

$$
N(B) = \left\{ \left(\frac{10}{7}a, \frac{2}{7}a, 0, a\right)^T \mid a \in \mathbb{R} \right\}
= \text{span}\left\{ \begin{bmatrix} 10 \\ 2 \\ 0 \\ 7 \end{bmatrix} \right\}
$$

\newpage
\noindent\textbf{Matrix C} \\
Performing row reduction:
$$
\begin{bmatrix} 1 & 3 & -2 & 1 \\ 2 & 1 & 3 & 2 \\ 3 & 4 & 5 & 6 \end{bmatrix}
\sim
\begin{bmatrix} 1 & 3 & -2 & 1 \\ 0 & -5 & 7 & 0 \\ 0 & -5 & 11 & 3 \end{bmatrix}
\sim
\begin{bmatrix} 1 & 3 & -2 & 1 \\ 0 & -5 & 7 & 0 \\ 0 & 0 & 4 & 3 \end{bmatrix}
\sim
\begin{bmatrix} 1 & 3 & -2 & 1 \\ 0 & -5 & 0 & -\frac{21}{4} \\ 0 & 0 & 1 & \frac{3}{4} \end{bmatrix}
$$ $$ \sim
\begin{bmatrix} 1 & 3 & 0 & \frac{10}{4} \\ 0 & 1 & 0 & \frac{21}{20} \\ 0 & 0 & 1 & \frac{3}{4} \end{bmatrix}
\sim
\begin{bmatrix} 1 & 0 & 0 & -\frac{13}{20} \\ 0 & 1 & 0 & \frac{21}{20} \\ 0 & 0 & 1 & \frac{3}{4} \end{bmatrix}
$$

\noindent The row and column space are spanned by the following basis vectors:
$$
\text{rowspace}(C) = \text{span}\left\{ \begin{bmatrix} 1 \\ 3 \\ -2 \\ 1 \end{bmatrix}, \begin{bmatrix} 2 \\ 1 \\ 3 \\ 2 \end{bmatrix}, \begin{bmatrix} 3 \\ 4 \\ 5 \\ 6 \end{bmatrix} \right\}
,\quad
\text{colspace}(C) = \text{span}\left\{ \begin{bmatrix} 1 \\ 2 \\ 3 \end{bmatrix}, \begin{bmatrix} 3 \\ 1 \\ 4 \end{bmatrix}, \begin{bmatrix} -2 \\ 3 \\ 5 \end{bmatrix} \right\}
$$

\noindent The null space is given by:
$$
\left[\begin{array}{cccc|c} 1 & 0 & 0 & -\frac{13}{20} & 0 \\ 0 & 1 & 0 & \frac{21}{20} & 0 \\ 0 & 0 & 1 & \frac{3}{4} & 0 \end{array}\right]
\rightarrow
\begin{cases} x_1 = \frac{13}{20}x_4 \\ x_2 = -\frac{21}{20}x_4 \\ x_3 = -\frac{3}{4}x_4 \\ x_4 = x_4 \end{cases}
$$
$$
N(C) = \left\{ \left(\frac{13}{20}a, -\frac{21}{20}a, -\frac{3}{4}a, a\right)^T \mid a \in \mathbb{R} \right\}
= \text{span}\left\{ \begin{bmatrix} 13 \\ -21 \\ -15 \\ 20 \end{bmatrix} \right\}
$$

\newpage
\section*{Question 3}
Let $E := [p_{1}(x) = 1, p_{2}(x) = x + 1, p_{3}(x) = x^2 - 1]$ and $F := q_{1}(x) = 1, q_{2}(x) = x, q_{3}(x) = x^2$.
These are two basis of the vector space $P_2$ of all polynomials of degree at least 2. 
Find the transition matrix from E to F and the transition matrix from F to E.
Express the polynomial $$p(x) = 11x^2 - 2x + 5$$ with respect to the basis E.

\vspace{0.5cm}
\noindent\textbf{Solution:} \\
To find the transition matrix from E to F, we can write the basis vectors of E as linear combinations of the basis vectors of F.
$$
\begin{cases} 1 = 1 \cdot 1 + 0 \cdot x + 0 \cdot x^2 \\ 
    x + 1 = 1 \cdot 1 + 1 \cdot x + 0 \cdot x^2 \\ 
    x^2 - 1 = -1 \cdot 1 + 0 \cdot x + 1 \cdot x^2 
\end{cases}
$$
Thus, the transition matrix from E to F is:
$$ \begin{bmatrix} 1 & 1 & -1 \\ 0 & 1 & 0 \\ 0 & 0 & 1 \end{bmatrix} $$

\noindent To find the transition matrix from F to E, we can do the same process.
$$
\begin{cases} 1 = 1 \cdot 1 + 0 \cdot (x + 1) + 0 \cdot (x^2 - 1) \\ 
    x = -1 \cdot 1 + 1 \cdot (x + 1) + 0 \cdot (x^2 - 1) \\ 
    x^2 = 1 \cdot 1 + 0 \cdot (x + 1) + 1 \cdot (x^2 - 1)
\end{cases}
$$
Thus, the transition matrix from F to E is:
$$ \begin{bmatrix} 1 & -1 & 1 \\ 0 & 1 & 0 \\ 0 & 0 & 1 \end{bmatrix} $$

\noindent The polynomial $p(x) = 11x^2 - 2x + 5$ can be written as a vector with respect to the standard basis as:
$$ p(x) = \begin{bmatrix} 5 \\ -2 \\ 11 \end{bmatrix} $$
\noindent To write $p(x)$ with respect to the basis E, we can multiply the vector by the transition matrix from the standard basis to E.
Since the basis F is actually the standard basis, we can use the transition matrix from F to E as the transition matrix from the standard basis to E.
That means we have:
$$ 
\begin{bmatrix} 1 & -1 & 1 \\ 0 & 1 & 0 \\ 0 & 0 & 1 \end{bmatrix} 
\begin{bmatrix} 5 \\ -2 \\ 11 \end{bmatrix} = \begin{bmatrix} 18 \\ -2 \\ 11 \end{bmatrix}
, \quad
p(x) = 18 p_1(x) - 2 p_2(x) + 11 p_3(x)
$$

\end{document}
