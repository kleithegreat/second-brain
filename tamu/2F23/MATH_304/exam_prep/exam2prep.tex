\documentclass{article}
\usepackage{amsmath,amssymb,amsthm}
\usepackage{fancyhdr}
\usepackage{enumerate}

\pagestyle{fancy}
\fancyhf{}
\lhead{MATH 304}
\rhead{Exam 2 Prep}
\renewcommand{\headrulewidth}{0.4pt}

\begin{document}

\section*{Question 1}
Determine if the vectors are linearly independent.
\begin{enumerate}[(a)]
    \item $v_1 = (1, 2, 3)^T, v_2 = (2, 3, 4)^T, v_3 = (3, 4, 5)^T$
    \item $v_2 = (0, 1, 0, 1)^T, v_2 = (1, 0, 1, 0)^T, v_3 = (2, 0 , 2, 0)^T, v_4 = (0, 2, 0, 2)^T$
    \item $v_3 = (-1, 1, -1, 1)^T, v_2 = (1, -1, 1, -1)^T, v_3 = (-1, 1, 1, -1)^T, v_4 = (1, 1, 1, 1)^T$
\end{enumerate}

\noindent\textbf{Solution:}
Use the determinant of the matrix formed by the vectors to determine if they are linearly independent. 
If the determinant is non-zero, then the vectors are linearly independent. If the determinant is zero, then the vectors are linearly dependent.

\noindent\textbf{(a)}
$$ V = \begin{bmatrix} 1 & 2 & 3 \\ 2 & 3 & 4 \\ 3 & 4 & 5 \end{bmatrix} $$
\begin{align*}
    \det(V) &= 1 \times \left| \begin{matrix} 3 & 4 \\ 4 & 5 \end{matrix} \right| - 2 \times \left| \begin{matrix} 2 & 4 \\ 3 & 5 \end{matrix} \right| + 3 \times \left| \begin{matrix} 2 & 3 \\ 3 & 4 \end{matrix} \right| \\
    &= 1 \times (15 - 16) - 2 \times (10 - 12) + 3 \times (8 - 9) \\
    &= 1 \times (-1) - 2 \times (-2) + 3 \times (-1) \\
    &= -1 + 4 - 3 \\
    &= 0
\end{align*}
$ (1, 2, 3)^T, (2, 3, 4)^T, (3, 4, 5)^T $ are linearly dependent.

\vspace{0.25cm}
\noindent\textbf{(b)}
$$ V = \begin{bmatrix} 0 & 1 & 2 & 0 \\ 1 & 0 & 0 & 2 \\ 0 & 1 & 2 & 0 \\ 1 & 0 & 0 & 2 \end{bmatrix} $$
Rows 1 and 3 are identical, so the determinant is zero, and 
\newline $ (0, 1, 0, 1)^T, (1, 0, 1, 0)^T, (2, 0 , 2, 0)^T, (0, 2, 0, 2)^T $ are linearly dependent.

\vspace{0.25cm}
\noindent\textbf{(c)}
$$ V = \begin{bmatrix} -1 & 1 & -1 & 1 \\ 1 & -1 & 1 & 1 \\ -1 & 1 & 1 & 1 \\ 1 & -1 & -1 & 1 \end{bmatrix} $$
Columns 1 and 2 are scalar multiples (by $-1$) of each other, so the determinant is zero, and $ (-1, 1, -1, 1)^T, (1, -1, 1, -1)^T, (-1, 1, 1, -1)^T, (1, 1, 1, 1)^T $ are linearly dependent.

\newpage
\section*{Question 2}
Determine if the following vectors in the vector space of smooth functions in $[0, 1]$ are linearly independent.
\begin{enumerate}[(a)]
    \item $p_1(x) := x^2, \quad p_2(x) := x^3, \quad p_3(x) := x^{99}$
    \item $f_1(x) := e^x, \quad f_2(x) := e^{3x}, \quad f_3(x) := e^{5x}, f_4(x) := e^{7x}$
    \item $f_1(x) := \cos(x), \quad f_2(x) := \sin(x), \quad f_3(x) := x$
\end{enumerate}

\noindent\textbf{Solution:}
We can use the Wronskian to determine if the vectors of functions are linearly independent.

\vspace{0.25cm}
\noindent\textbf{(a)}
Since $x^2, x^3, x^{99}$ cannot be written as a linear combination of each other, they are linearly independent.

\vspace{0.25cm}
\noindent\textbf{(b)}
Since $e^x, e^{3x}, e^{5x}, e^{7x}$ cannot be written as a linear combination of each other, they are linearly independent.

\vspace{0.25cm}
\noindent\textbf{(c)}
\begin{align*}
W(f_1, f_2, f_3) &= \begin{vmatrix} \cos(x) & \sin(x) & x \\ -\sin(x) & \cos(x) & 1 \\ -\cos(x) & -\sin(x) & 0 \end{vmatrix} \\
&= x \times \begin{vmatrix} -\sin(x) & \cos(x) \\ -\cos(x) & -\sin(x) \end{vmatrix} - 1 \times \begin{vmatrix} \cos(x) & \sin(x) \\ -\cos(x) & -\sin(x) \end{vmatrix} \\
&= x \times (\sin^2(x) + \cos^2(x)) - 1 \times (-\sin(x)\cos(x) + \sin(x)\cos(x)) \\
&= x \times 1 - 1 \times 0 \\
&= x
\end{align*}
Since $x$ is not identically zero, $f_1(x) := \cos(x), \quad f_2(x) := \sin(x), \quad f_3(x) := x$ are linearly independent.

\newpage
\section*{Question 3}
Find a basis for the row space, column space, and null space for the following matrices.
$$
A = \begin{bmatrix} 1 & 1 & 0 \\ 3 & 1 & 4 \\ 2 & 3 & 5 \end{bmatrix}, \;
B = \begin{bmatrix} 2 & 0 & 0 & 1 \\ 0 & 4 & -7 & -1 \\ 0 & -7 & 8 & 1 \\ 1 & -1 & 1 & -1 \end{bmatrix}, \;
C = \begin{bmatrix} 1 & 1 \\ -1 & 1 \\ 2 & 0 \end{bmatrix}, \;
D = \begin{bmatrix} 0 & 1 & 2 \\ -1 & 1 & 2 \end{bmatrix}
$$

\noindent\textbf{Solution:} Use Gaussian elimination to find the reduced row echelon form of each matrix.

\vspace{0.25cm}
\noindent\textbf{Matrix A}
$$
\begin{bmatrix} 1 & 1 & 0 \\ 3 & 1 & 4 \\ 2 & 3 & 5 \end{bmatrix} \sim
\begin{bmatrix} 1 & 1 & 0 \\ 0 & -2 & 4 \\ 0 & 1 & 5 \end{bmatrix} \sim
\begin{bmatrix} 1 & 1 & 0 \\ 0 & 1 & 5 \\ 0 & 0 & 14 \end{bmatrix}
$$
Here we can see that all three rows are linearly independent, so the row space is the span of the three rows.
$$\text{rowspace}(A) = \text{span}\left\{ \begin{bmatrix} 1 \\ 1 \\ 0 \end{bmatrix}, \begin{bmatrix} 3 \\ 1 \\ 4 \end{bmatrix}, \begin{bmatrix} 2 \\ 3 \\ 5 \end{bmatrix} \right\}$$
Since dim(rowspace) $=$ dim(colspace), the column space is the span of the three columns.
$$\text{colspace}(A) = \text{span}\left\{ \begin{bmatrix} 1 \\ 3 \\ 2 \end{bmatrix}, \begin{bmatrix} 1 \\ 1 \\ 3 \end{bmatrix}, \begin{bmatrix} 0 \\ 4 \\ 5 \end{bmatrix} \right\}$$
By the Rank-Nullity Theorem:
$$\text{nullspace}(A) = \text{span}\left\{ \begin{bmatrix} 0 \\ 0 \\ 0 \end{bmatrix} \right\}$$

\noindent\textbf{Matrix B}
$$
\begin{bmatrix} 2 & 0 & 0 & 1 \\ 0 & 4 & -7 & -1 \\ 0 & -7 & 8 & 1 \\ 1 & -1 & 1 & -1 \end{bmatrix} \sim
\begin{bmatrix} 0 & 2 & -2 & 3 \\ 0 & -3 & 1 & 0 \\ 0 & -7 & 8 & 1 \\ 1 & -1 & 1 & -1 \end{bmatrix} \sim
\begin{bmatrix} 0 & 2 & -2 & 3 \\ 0 & -1 & -1 & 3 \\ 0 & 7 & -8 & -1 \\ 1 & -1 & 1 & -1 \end{bmatrix} \sim
\begin{bmatrix} 0 & 0 & -4 & 9 \\ 0 & -1 & -1 & 3 \\ 0 & 0 & -15 & 20 \\ 1 & -1 & 1 & -1 \end{bmatrix}
$$ $$ \sim
\begin{bmatrix} 1 & -1 & 1 & -1 \\ 0 & 1 & 1 & -3 \\ 0 & 0 & -4 & 9 \\ 0 & 0 & -15 & 20 \end{bmatrix} \sim
\begin{bmatrix} 1 & -1 & 1 & -1 \\ 0 & 1 & 1 & -3 \\ 0 & 0 & -4 & 9 \\ 0 & 0 & 1 & -16 \end{bmatrix} \sim
\begin{bmatrix} 1 & -1 & 1 & -1 \\ 0 & 1 & 1 & -3 \\ 0 & 0 & 1 & -16 \\ 0 & 0 & 0 & -55 \end{bmatrix}
$$
Since $\text{rank} = 4$:
$$\text{rowspace} = \text{span}\left\{ \begin{bmatrix} 2 \\ 0 \\ 0 \\ 1 \end{bmatrix}, \begin{bmatrix} 0 \\ 4 \\ -7 \\ -1 \end{bmatrix}, \begin{bmatrix} 0 \\ -7 \\ 8 \\ 1 \end{bmatrix}, \begin{bmatrix} 1 \\ -1 \\ 1 \\ -1 \end{bmatrix} \right\}
, \quad
\text{colspace} = \text{span}\left\{ \begin{bmatrix} 2 \\ 0 \\ 0 \\ 1 \end{bmatrix}, \begin{bmatrix} 0 \\ 4 \\ -7 \\ -1 \end{bmatrix}, \begin{bmatrix} 0 \\ -7 \\ 8 \\ 1 \end{bmatrix}, \begin{bmatrix} 1 \\ -1 \\ 1 \\ -1 \end{bmatrix} \right\}
$$ $$
\text{nullspace} = \text{span}\left\{ \begin{bmatrix} 0 \\ 0 \\ 0 \\ 0 \end{bmatrix} \right\}
$$

\noindent\textbf{Matrix C}
$$
\begin{bmatrix} 1 & 1 \\ -1 & 1 \\ 2 & 0 \end{bmatrix} \sim
\begin{bmatrix} 1 & 1 \\ 0 & 2 \\ 0 & -2 \end{bmatrix} \sim
\begin{bmatrix} 1 & 1 \\ 0 & 2 \\ 0 & 0 \end{bmatrix}
$$
Since row 3 can be written as row 1 - row 2, the row space is the span of the first two rows.
$$ \text{rowspace}(C) = \text{span}\left\{ \begin{bmatrix} 1 \\ 1 \end{bmatrix}, \begin{bmatrix} -1 \\ 1 \end{bmatrix} \right\} $$
Since the column space is the span of the vectors in the columns of the original matrix corresponding to the columns with leading ones in the reduced row echelon form, 
the column space is the span of the first two (or both) columns.
$$ \text{colspace}(C) = \text{span}\left\{ \begin{bmatrix} 1 \\ -1 \\ 2 \end{bmatrix}, \begin{bmatrix} 1 \\ 1 \\ 0 \end{bmatrix} \right\} $$
By the Rank-Nullity Theorem:
$$ \text{nullspace}(C) = \text{span}\left\{ \begin{bmatrix} 0 \\ 0 \end{bmatrix} \right\} $$

\noindent\textbf{Matrix D}
$$
\begin{bmatrix} 0 & 1 & 2 \\ -1 & 1 & 2 \end{bmatrix} \sim
\begin{bmatrix} 1 & -1 & -2 \\ 0 & 1 & 2 \end{bmatrix} \sim
\begin{bmatrix} 1 & 0 & 0 \\ 0 & 1 & 2 \end{bmatrix}
$$
Thus, we have:
$$
\text{rowspace}(D) = \text{span}\left\{ \begin{bmatrix} 0 \\ 1 \\ 2 \end{bmatrix}, \begin{bmatrix} -1 \\ 1 \\ 2 \end{bmatrix} \right\}, \quad
\text{colspace}(D) = \text{span}\left\{ \begin{bmatrix} 0 \\ -1 \end{bmatrix}, \begin{bmatrix} 1 \\ 1 \end{bmatrix} \right\}
$$
Solving for the null space:
$$
\begin{cases} x_1 = 0 \\ x_2 + 2x_3 = 0 \end{cases}
\rightarrow
\begin{cases} x_1 = 0 \\ x_2 = -2x_3 \end{cases}
$$
$$
\text{nullspace}(D) = \{(0, -2a, a)^T \mid a \in \mathbb{R} \} = \text{span}\left\{ \begin{bmatrix} 0 \\ -2 \\ 1 \end{bmatrix} \right\}
$$

\newpage
\section*{Question 4}
Let
$$ u_1 := (1, 0 ,2)^T, \; u_2 := (-1, 1, 0)^T, \; u_3 := (1, 0, 1)^T $$
$$ v_1 := (1, 1, -1)^T, \; v_2 := (-1, 0, 0)^T, \; v_3 := (-1, 1, 1)^T $$
\begin{enumerate}[a.]
    \item Find the transition matrix corresponding to the change of basis from $\{e_1, e_2, e_3\}$ to $\{u_1, u_2, u_3\}$.
    \item Find the transition matrix corresponding to the change of basis from $\{v_1, v_2, v_3\}$ to $\{e_1, e_2, e_3\}$.
    \item Find the transition matrix from $\{v_1, v_2, v_3\}$ to $\{u_1, u_2, u_3\}$.
    \item Let $x = 1v_1 + 0v_2 - v_3$. Find the coordinates of $x$ with respect to $\{u_1, u_2, u_3\}$.
    \item Verify your answer to the previous part by computing the coordinates in each case with respect to the standard basis.
\end{enumerate}

\noindent\textbf{Solution:}
\newline\textbf{a.} Let the transition matrix from $\{e_1, e_2, e_3\}$ to $\{u_1, u_2, u_3\}$ be $U^{-1}$.
If $U$ is the transition matrix from $\{u_1, u_2, u_3\}$ to $\{e_1, e_2, e_3\}$, then we just need to find the inverse of $U$.
$$
U = \begin{bmatrix} 1 & -1 & 1 \\ 0 & 1 & 0 \\ 2 & 0 & 1 \end{bmatrix}
, \quad
U^{-1} = \begin{bmatrix} -1 & -1 & 1 \\ 0 & 1 & 0 \\ 2 & 2 & -1 \end{bmatrix}
$$
\textbf{b.} Let the transition matrix from $\{v_1, v_2, v_3\}$ to $\{e_1, e_2, e_3\}$ be $V$.
$$ V = \begin{bmatrix} 1 & -1 & -1 \\ 1 & 0 & 1 \\ -1 & 0 & 1 \end{bmatrix} $$
\textbf{c.} Let the transition matrix from $\{v_1, v_2, v_3\}$ to $\{u_1, u_2, u_3\}$ be $U^{-1}V$.
$$
U^{-1}V = \begin{bmatrix} -1 & -1 & 1 \\ 0 & 1 & 0 \\ 2 & 2 & -1 \end{bmatrix} \begin{bmatrix} 1 & -1 & -1 \\ 1 & 0 & 1 \\ -1 & 0 & 1 \end{bmatrix} 
= \begin{bmatrix} -3 & 1 & 1 \\ 1 & 0 & 1 \\ 5 & -2 & -1 \end{bmatrix}
$$
\textbf{d.} $x$ with respect to $\{u_1, u_2, u_3\}$ is $U^{-1}Vx$.
$$
U^{-1}Vx = \begin{bmatrix} -3 & 1 & 1 \\ 1 & 0 & 1 \\ 5 & -2 & -1 \end{bmatrix} \begin{bmatrix} 1 \\ 0 \\ -1 \end{bmatrix}
= \begin{bmatrix} -4 \\ 0 \\ 6 \end{bmatrix}
$$
\textbf{e.} $x$ with respect to the standard basis:
$$
\begin{bmatrix} 1 & -1 & -1 \\ 1 & 0 & 1 \\ -1 & 0 & 1 \end{bmatrix} \begin{bmatrix} 1 \\ 0 \\ -1 \end{bmatrix}
= \begin{bmatrix} 2 \\ 0 \\ -2 \end{bmatrix}
, \quad
\begin{bmatrix} 1 & -1 & 1 \\ 0 & 1 & 0 \\ 2 & 0 & 1 \end{bmatrix} \begin{bmatrix} -4 \\ 0 \\ 6 \end{bmatrix}
= \begin{bmatrix} 2 \\ 0 \\ -2 \end{bmatrix}
$$

\newpage
\section*{Question 5}
For each of the following choises of $A$, $b$, determine whether $b$ is in the column space of $A$ and state whether the system $Ax = b$ is consistent or not.
\begin{enumerate}[a.]
    \item $A = \begin{bmatrix} 0 & 1 \\ -1 & 1 \end{bmatrix}, \; b = \begin{bmatrix} 1 \\ -1 \end{bmatrix}$
    \item $A = \begin{bmatrix} 0 & 1 \\ 1 & 0 \\ 0 & 1 \end{bmatrix}, \; b = \begin{bmatrix} 2 \\ 5 \\ 2 \end{bmatrix}$
    \item $A = \begin{bmatrix} 1 & 1 & 2 \\ 1 & 1 & 2 \\ 1 & 1 & 2 \end{bmatrix}, \; b = \begin{bmatrix} 1 \\ 2 \\ 3 \end{bmatrix}$
\end{enumerate}

\noindent\textbf{Solution:} The system $Ax = b$ is consistent if and only if $b$ is in the column space of $A$.

\vspace{0.25cm}
\noindent\textbf{a.}
$$
\left[\begin{array}{cc|c} 0 & 1 & 1 \\ -1 & 1 & -1 \end{array}\right] \sim
\left[\begin{array}{cc|c} 1 & -1 & 1 \\ 0 & 1 & 1 \end{array}\right] \sim
\left[\begin{array}{cc|c} 1 & 0 & 2 \\ 0 & 1 & 1 \end{array}\right]
$$
Since the system is consistent, $b \in \text{colspace}(A)$.

\vspace{0.25cm}
\noindent\textbf{b.}
$$
\left[\begin{array}{cc|c} 0 & 1 & 2 \\ 1 & 0 & 5 \\ 0 & 1 & 2 \end{array}\right] \sim
\left[\begin{array}{cc|c} 1 & 0 & 5 \\ 0 & 1 & 2 \\ 0 & 0 & 0 \end{array}\right]
$$
Since the system is consistent, $b \in \text{colspace}(A)$.

\vspace{0.25cm}
\noindent\textbf{c.}
$$ \left[\begin{array}{ccc|c} 1 & 1 & 2 & 1 \\ 1 & 1 & 2 & 2 \\ 1 & 1 & 2 & 3 \end{array}\right] $$
This system is obviously inconsistent, so $b \notin \text{colspace}(A)$.

\newpage
\section*{Question 6}
Let
$$
u_1 := (1, 1, 0)^T, \; u_2 := (1, 0, 1)^T, \; u_3 := (0, 1, 1)^T
$$
be a basis of $\mathbb{R}^3$. Define $: \mathbb{R}^2 \rightarrow \mathbb{R}^3$ as
$$ L(x) := x_1u_1 + x_2u_2 + (x_1 + x_2)u_3 $$
Find a matrix $A$ representing $L$ with respect to the ordered basis $e_1, e_2$ and $u_1, u_2, u_3$.

\vspace{0.25cm}
\noindent\textbf{Solution:}
\newline\noindent Applying $L$ to $e_1$ and $e_2$:
$$ L(e_1) = (u_1, 0, u_3)^T, \quad L(e_2) = (0, u_2, u_3)^T $$
Thus, the matrix $A$ with respect to the ordered basis $e_1, e_2$ and $u_1, u_2, u_3$ is:
$$ A = \begin{bmatrix} 1 & 0 \\ 0 & 1 \\ 1 & 1 \end{bmatrix} $$

\newpage
\section*{Question 7}
For each of the following transformations, find a matrix $A$ such that $L(x) = Ax$.
\begin{enumerate}[a.]
    \item $L((x_1, x_2)^T) = (x_1 + x_2, x_2 + x_1, x_1)^T$
    \item $L((x_1, x_2, x_3)^T) = (x_3, x_2, x_1)^T$
    \item $L((x_1, x_2, x_3)^T) = (x_1)$
    \item $L : P_3 \rightarrow P_3, \; P(a + bx + cx^2) = (c + bx + ax^2)$. (Consider the standard basis).
\end{enumerate}

\noindent\textbf{Solution:} Assuming that the basis of the domain and codomain are the standard basis:
\newline\noindent\textbf{a.}
$ L(e_1) = (1, 1, 1)^T, \; L(e_2) = (1, 1, 0)^T $
$$ A = \begin{bmatrix} 1 & 1 \\ 1 & 1 \\ 1 & 0 \end{bmatrix} $$
\textbf{b.}
$ L(e_1) = (0, 0, 1)^T, \; L(e_2) = (0, 1, 0)^T, \; L(e_3) = (1, 0, 0)^T $
$$ A = \begin{bmatrix} 0 & 0 & 1 \\ 0 & 1 & 0 \\ 1 & 0 & 0 \end{bmatrix} $$
\textbf{c.}
$ L(e_1) = [1], \; L(e_2) = [0], \; L(e_3) = [0] $
$$ A = \begin{bmatrix} 1 & 0 & 0 \end{bmatrix} $$
\textbf{d.}
Assuming that the standard basis for $\mathbb{P}_3$ is $\{1, x, x^2\}$:
$$ L(1) = x^2, \; L(x) = x, \; L(x^2) = 1 $$
$$ A = \begin{bmatrix} 0 & 0 & 1 \\ 0 & 1 & 0 \\ 1 & 0 & 0 \end{bmatrix} $$

\newpage
\section*{Question 8}
Determine which of the following sentences are true or false \\
1. If $\{v_1, ..., v_n\}$ are linearly dependent, then at least one of them can be written as a linear combination of the rest.
    \newline\textbf{True.} By definition of linear dependence. \\
2. If $L:V \rightarrow W$ is a linear map then maps $0_v$ to $0_w$.
    \newline\textbf{True.} \\
3. Let $S$ be a subspace of $V$ and dim($S$) = $n$ = dim($V$). Then $S = V$.
    \newline\textbf{True.} $v$ and $S$ having the same dimension impies that they both span the same space, so $S = V$. \\
4. If $\{v_1, ..., v_n\} \subseteq V$ are linearly independent and dim($V$) = $n + 1$, then one can always find a $v_{n+1}$ such that $\{v_1, ..., v_{n+1}\}$ form a basis for $V$.
    \newline\textbf{True.} Since $V$ requires $n$ linearly independent vectors to form a basis, there must be at least one vector $v_{n+1}$ that is linearly independent from the rest. \\
5. The only linear maps $L: \mathbb{R} \rightarrow \mathbb{R}$ are of the form $f(x) = ax$ for some $a \in \mathbb{R}$.
    \newline\textbf{True.} Adding constants or exponents would make it so that $L$ is not closed under addition or scalar multiplication. \\
6. The map that reflects a point through the origin is a linear map.
    \newline\textbf{True.} It is closed under addition and scalar multiplication. \\
7. The are vector spaces with infinite dimensions.
    \newline\textbf{True.} The vector space of polynomials is an example. \\
8. The every vector space has either only 1 element of infinitely many.
    \newline\textbf{???} \\
9. If 0 is inside a subset $S$ of $V$ then $S$ is a subspace.
    \newline\textbf{False.} $S$ must also be closed under addition and scalar multiplication. \\
10. If $S$ is a subspace then 0 is inside $S$.
    \newline\textbf{True.} $S$ must contain the zero vector in order to be a subspace. \\
11. If $L:V \rightarrow W$ is a linear map then $2L$ is also a linear map.
    \newline\textbf{True.} Linear maps are closed under scalar multiplication. \\
12. If $L:V \rightarrow W$ is a linear map then $L + 2$ is also a linear map.
    \newline\textbf{False.} Adding a scalar to a linear map isn't meaningful. \\
13. If $A$ is a singular $3 \times 3$ matrix then the map $L(x) = A(x)$ is a linear map.
    \newline\textbf{True.} Matrix multiplication is always a linear operation. \\
14. If $Ax = b$ has a solution then $b$ lies inside the row space of $A$.
    \newline\textbf{False.} The theorem is that $b$ lies inside the column space of $A$ if and only if $Ax = b$ is consistent. \\
15. IF $A^T$ is row equivalent with $B^T$, then $A, B$ have the same column space.
    \newline\textbf{True.} Transposes just swap rows and columns, so row equivalence of a transpose is the same as column equivalence. \\

\end{document}
