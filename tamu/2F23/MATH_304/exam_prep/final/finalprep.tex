\documentclass{article}
\usepackage{amsmath,amssymb,amsthm}
\usepackage{fancyhdr}
\usepackage{enumerate}

\pagestyle{fancy}
\fancyhf{}
\lhead{MATH 304}
\rhead{Exam 2 Prep}
\renewcommand{\headrulewidth}{0.4pt}

\begin{document}

\section*{Question 1}
Find the general solution of each of the following systems:
\begin{enumerate}[a.]
    \item $\left\{\begin{aligned}
        y_1 + \phantom{4} y_2 &= y'_1 \\
        -2y_1 + 4y_2 &= y'_2
    \end{aligned}\right.$
    \item $\left\{\begin{aligned}
        y_1 - y_2 &= y'_1 \\
        y_1 + y_2 &= y'_2
    \end{aligned}\right.$
    \item $\left\{\begin{aligned}
        y_1 \phantom{+ 2y_2} + \phantom{6} y_3 &= y'_1 \\
        2y_2 + 6y_3 &= y'_2 \\
        y_2 \phantom{+ 2y_2} + 3y_3 &= y'_3
    \end{aligned}\right.$
\end{enumerate}

\vspace{0.25cm}\noindent\textbf{Solution:}
For systems in the form $y' = Ay$, we can find $y$ as $e^{At}$, where $e^{At} = I + At + \frac{1}{2!}A^2t^2 + \frac{1}{3!}A^3t^3 + \cdots$.
Additionally, when $A$ is diagonalizable, we can write $A = Xe^D X^{-1}$, where $X$ is the matrix of eigenvectors of $A$ and $D$ is the diagonal matrix of eigenvalues of $A$.
Then, $e^{At} = X e^{Dt} X^{-1}$.

\newpage
\noindent\textbf{\textit{System a:}}
\begin{align*}
    A &= \begin{pmatrix} 1 & 1 \\ -2 & 4 \end{pmatrix} \\
    P(\lambda) &= \det(A - \lambda I) = \begin{vmatrix} 1 - \lambda & 1 \\ -2 & 4 - \lambda \end{vmatrix} \\
    &= (1 - \lambda)(4 - \lambda) + 2 = \lambda^2 - 5\lambda + 6 \\
    &= (\lambda - 2)(\lambda - 3) \\
    \\
    &\lambda_1 = 2, \quad \lambda_2 = 3 \\
    \\
    N(A - \lambda_1 I) &= N\left(\begin{pmatrix} -1 & 1 \\ -2 & 2 \end{pmatrix}\right) \\
    &= \left\{ \begin{pmatrix} x_1 \\ x_2 \end{pmatrix} : -x_1 + x_2 = 0 \right\}
    = \text{span}\left\{ \begin{pmatrix} 1 \\ 1 \end{pmatrix} \right\} \\
    N(A - \lambda_2 I) &= N\left(\begin{pmatrix} -2 & 1 \\ -2 & 1 \end{pmatrix}\right) \\
    &= \left\{ \begin{pmatrix} x_1 \\ x_2 \end{pmatrix} : -2x_1 + x_2 = 0 \right\}
    = \text{span}\left\{ \begin{pmatrix} 1 \\ 2 \end{pmatrix} \right\} \\
    \\
    &X = \begin{pmatrix} 1 & 1 \\ 1 & 2 \end{pmatrix} , \quad
    X^{-1} = \begin{pmatrix} 2 & -1 \\ -1 & 1 \end{pmatrix} , \quad
    D = \begin{pmatrix} 2 & 0 \\ 0 & 3 \end{pmatrix} \\
    e^{Dt} &= \begin{pmatrix} e^{2t} & 0 \\ 0 & e^{3t} \end{pmatrix} \\
    \\
    y &= e^{At} = X e^{Dt} X^{-1} \\
    &= \begin{pmatrix} 1 & 1 \\ 1 & 2 \end{pmatrix} \begin{pmatrix} e^{2t} & 0 \\ 0 & e^{3t} \end{pmatrix} \begin{pmatrix} 2 & -1 \\ -1 & 1 \end{pmatrix} \\
    &= \begin{pmatrix} 1 & 1 \\ 1 & 2 \end{pmatrix} \begin{pmatrix} 2e^{2t} & -e^{2t} \\ -e^{3t} & e^{3t} \end{pmatrix} \\
    &= \begin{pmatrix} 2e^{2t} - e^{3t} & -e^{2t} + e^{3t} \\ 2e^{2t} - 2e^{3t} & -e^{2t} + 2e^{3t} \end{pmatrix} \\
\end{align*}

\newpage
\noindent\textbf{\textit{System b:}}
\begin{align*}
    A &= \begin{pmatrix} 1 & -1 \\ 1 & 1 \end{pmatrix} \\
    P(\lambda) &= \det(A - \lambda I) = \begin{vmatrix} 1 - \lambda & -1 \\ 1 & 1 - \lambda \end{vmatrix} \\
    &= (1 - \lambda)^2 + 1 = \lambda^2 - 2\lambda + 2 \\
    \\
    &\lambda_1 = 1 + i, \quad \lambda_2 = 1 - i \\
    \\
    N(A - \lambda_1 I) &= N\left(\begin{pmatrix} -i & -1 \\ 1 & -i \end{pmatrix}\right) \\
    &= \left\{ \begin{pmatrix} x_1 \\ x_2 \end{pmatrix} : -ix_1 - x_2 = 0 \right\}
    = \text{span}\left\{ \begin{pmatrix} 1 \\ -i \end{pmatrix} \right\} \\
    N(A - \lambda_2 I) &= N\left(\begin{pmatrix} i & -1 \\ 1 & i \end{pmatrix}\right) \\
    &= \left\{ \begin{pmatrix} x_1 \\ x_2 \end{pmatrix} : ix_1 - x_2 = 0 \right\}
    = \text{span}\left\{ \begin{pmatrix} 1 \\ i \end{pmatrix} \right\} \\
    \\
    &X = \begin{pmatrix} 1 & 1 \\ -i & i \end{pmatrix} , \quad
    X^{-1} = \frac{1}{2i} \begin{pmatrix} i & -1 \\ i & 1 \end{pmatrix} , \quad
    D = \begin{pmatrix} 1 + i & 0 \\ 0 & 1 - i \end{pmatrix} \\
    &\text{Using the fact that } e^{a + bi} = e^a(\cos b + i \sin b) \\
    e^{Dt} &= \begin{pmatrix} e^t(\cos t + i \sin t) & 0 \\ 0 & e^t(\cos t - i \sin t) \end{pmatrix} \\
    \\
    y &= e^{At} = X e^{Dt} X^{-1} \\
    &= \begin{pmatrix} 1 & 1 \\ -i & i \end{pmatrix} \begin{pmatrix} e^t(\cos t + i \sin t) & 0 \\ 0 & e^t(\cos t - i \sin t) \end{pmatrix} \frac{1}{2i} \begin{pmatrix} i & -1 \\ i & 1 \end{pmatrix} \\
    &= \frac{1}{2i} \begin{pmatrix} e^t (\cos t + i \sin t) & e^t (\cos t - i \sin t) \\ -ie^t (\cos t + i \sin t) & ie^t (\cos t - i \sin t) \end{pmatrix} \begin{pmatrix} i & -1 \\ i & 1 \end{pmatrix} \\
    &= \frac{1}{2i} \begin{pmatrix} ie^t (\cos t + i \sin t) + ie^t (\cos t - i \sin t) & -e^t (\cos t + i \sin t) + e^t (\cos t - i \sin t) \\ e^t (\cos t + i \sin t) - e^t (\cos t - i \sin t) & ie^t (\cos t + i \sin t) + ie^t (\cos t - i \sin t) \end{pmatrix} \\
    &= \frac{1}{2i} \begin{pmatrix} 2ie^t \cos t & -2ie^t \sin t \\ 2ie^t \sin t & 2ie^t \cos t \end{pmatrix} \\
    &= \begin{pmatrix} e^t \cos t & -e^t \sin t \\ e^t \sin t & e^t \cos t \end{pmatrix} \\
\end{align*}

\newpage
\noindent\textbf{\textit{System c:}}
\begin{align*}
    A &= \begin{pmatrix} 1 & 0 & 1 \\ 0 & 2 & 6 \\ 0 & 1 & 3 \end{pmatrix} \\
    P(\lambda) &= \det(A - \lambda I) = \begin{vmatrix} 1 - \lambda & 0 & 1 \\ 0 & 2 - \lambda & 6 \\ 0 & 1 & 3 - \lambda \end{vmatrix}
    = (1 - \lambda) \begin{vmatrix} 2 - \lambda & 6 \\ 1 & 3 - \lambda \end{vmatrix} \\
    &= (1 - \lambda)((2 - \lambda)(3 - \lambda) - 6) = (1 - \lambda)(\lambda^2 - 5\lambda) = -\lambda^3 + 6\lambda^2 - 5\lambda \\
    &= -\lambda(\lambda^2 - 6\lambda + 5) = -\lambda(\lambda - 1)(\lambda - 5) \\
    \\
    &\lambda_1 = 0, \quad \lambda_2 = 1, \quad \lambda_3 = 5 \\
    \\
    N(A - \lambda_1 I) &= N\left(\begin{pmatrix} 1 & 0 & 1 \\ 0 & 2 & 6 \\ 0 & 1 & 3 \end{pmatrix}\right) = N\left(\begin{pmatrix} 1 & 0 & 1 \\ 0 & 1 & 3 \\ 0 & 0 & 0 \end{pmatrix}\right) \\
    &= \left\{ \begin{pmatrix} x_1 \\ x_2 \\ x_3 \end{pmatrix} : x_1 + x_3 = 0, \; x_2 + 3x_3 = 0 \right\}
    = \text{span}\left\{ \begin{pmatrix} 1 \\ 3 \\ -1 \end{pmatrix} \right\} \\
    N(A - \lambda_2 I) &= N\left(\begin{pmatrix} 0 & 0 & 1 \\ 0 & 1 & 6 \\ 0 & 1 & 2 \end{pmatrix}\right) = \text{span}\left\{ \begin{pmatrix} 1 \\ 0 \\ 0 \end{pmatrix} \right\} \\
    N(A - \lambda_3 I) &= N\left(\begin{pmatrix} -4 & 0 & 1 \\ 0 & -3 & 6 \\ 0 & 1 & -2 \end{pmatrix}\right) = N\left(\begin{pmatrix} 4 & 0 & -1 \\ 0 & 1 & -2 \\ 0 & 0 & 0 \end{pmatrix}\right) \\
    &= \left\{ \begin{pmatrix} x_1 \\ x_2 \\ x_3 \end{pmatrix} : 4x_1 - x_3 = 0, \; x_2 - 2x_3 = 0 \right\}
    = \text{span}\left\{ \begin{pmatrix} 1 \\ 8 \\ 4 \end{pmatrix} \right\} \\
    \\
    &X = \begin{pmatrix} 1 & 1 & 1 \\ 3 & 0 & 8 \\ -1 & 0 & 4 \end{pmatrix} , \quad
    X^{-1} = \frac{1}{20} \begin{pmatrix} 0 & 4 & -8 \\ 20 & -5 & 5 \\ 0 & 1 & 3 \end{pmatrix} , \quad
    D = \begin{pmatrix} 0 & 0 & 0 \\ 0 & 1 & 0 \\ 0 & 0 & 5 \end{pmatrix} \\
    e^{Dt} &= \begin{pmatrix} 1 & 0 & 0 \\ 0 & e^t & 0 \\ 0 & 0 & e^{5t} \end{pmatrix} \\
    \\
    y &= e^{At} = X e^{Dt} X^{-1} \\
    &= \begin{pmatrix} 1 & 1 & 1 \\ 3 & 0 & 8 \\ -1 & 0 & 4 \end{pmatrix} \begin{pmatrix} 1 & 0 & 0 \\ 0 & e^t & 0 \\ 0 & 0 & e^{5t} \end{pmatrix} \frac{1}{20} \begin{pmatrix} 0 & 4 & -8 \\ 20 & -5 & 5 \\ 0 & 1 & 3 \end{pmatrix} \\
    &= \frac{1}{20} \begin{pmatrix}20e^t & e^{5t} - 5e^t + 4 & 3e^{5t} + 5e^t - 8 \\ 0 & 8e^{5t} + 12 & 24e^{5t} - 24 \\ 0 & 4e^{5t} - 4 & 12e^{5t} + 8 \end{pmatrix}
\end{align*}

\newpage
\section*{Question 2}
Solve the following initial value problems:
\begin{enumerate}[a.]
    \item $\left\{\begin{aligned}
        -y_1 + 2y_2 &= y'_1 \\
        \phantom{-}2y_1 - \phantom{2}y_2 &= y'_2
    \end{aligned}\right.$, \; $y_1(0) = 3$, \; $y_2(0) = 1$.
    \item $\left\{\begin{aligned}
        \phantom{2}y_1 - 2y_2 &= y'_1 \\
        2y_1 + \phantom{2}y_2 &= y'_2
    \end{aligned}\right.$, \; $y_1(0) = 1$, \; $y_2(0) = -2$.
\end{enumerate}

\vspace{0.25cm}\noindent\textbf{Solution:}
Again, we use the fact that $y = e^{At}$, where $A$ is the matrix of coefficients of the system.
To solve the initial conditions, we use the fact that $y = e^{At}c$, where $c$ is a vector of constants.
Then, we can solve for $c$ using the initial conditions.

\newpage
\noindent\textbf{\textit{System a:}}
\begin{align*}
    A &= \begin{pmatrix} -1 & 2 \\ 2 & -1 \end{pmatrix} \\
    P(\lambda) &= \det(A - \lambda I) = \begin{vmatrix} -1 - \lambda & 2 \\ 2 & -1 - \lambda \end{vmatrix} \\
    &= (-1 - \lambda)^2 - 4 = \lambda^2 + 2\lambda - 3 \\
    &= (\lambda + 3)(\lambda - 1) \\
    \\
    &\lambda_1 = -3, \quad \lambda_2 = 1 \\
    \\
    N(A - \lambda_1 I) &= N\left(\begin{pmatrix} 2 & 2 \\ 2 & 2 \end{pmatrix}\right) \\
    &= \left\{ \begin{pmatrix} x_1 \\ x_2 \end{pmatrix} : x_1 + x_2 = 0 \right\}
    = \text{span}\left\{ \begin{pmatrix} 1 \\ -1 \end{pmatrix} \right\} \\
    N(A - \lambda_2 I) &= N\left(\begin{pmatrix} -2 & 2 \\ 2 & -2 \end{pmatrix}\right) \\
    &= \left\{ \begin{pmatrix} x_1 \\ x_2 \end{pmatrix} : -x_1 + x_2 = 0 \right\}
    = \text{span}\left\{ \begin{pmatrix} 1 \\ 1 \end{pmatrix} \right\} \\
    \\
    &X = \begin{pmatrix} 1 & 1 \\ -1 & 1 \end{pmatrix} , \quad
    X^{-1} = \frac{1}{2} \begin{pmatrix} 1 & -1 \\ 1 & 1 \end{pmatrix} , \quad
    D = \begin{pmatrix} -3 & 0 \\ 0 & 1 \end{pmatrix} \\
    e^{Dt} &= \begin{pmatrix} e^{-3t} & 0 \\ 0 & e^t \end{pmatrix} \\
    \\
    y &= e^{At} = X e^{Dt} X^{-1} c \\
    &= \begin{pmatrix} 1 & 1 \\ -1 & 1 \end{pmatrix} \begin{pmatrix} e^{-3t} & 0 \\ 0 & e^t \end{pmatrix} \frac{1}{2} \begin{pmatrix} 1 & -1 \\ 1 & 1 \end{pmatrix} c \\
    &= \frac{1}{2} \begin{pmatrix} e^{-3t} & e^t \\ -e^{-3t} & e^t \end{pmatrix} \begin{pmatrix} 1 & -1 \\ 1 & 1 \end{pmatrix} c \\
    &= \frac{1}{2} \begin{pmatrix} e^{-3t} + e^t & -e^{-3t} + e^t \\ -e^{-3t} + e^t & e^{-3t} + e^t \end{pmatrix} c \\
    \\
    y(0) &= \begin{pmatrix} 3 \\ 1 \end{pmatrix} = \frac{1}{2} \begin{pmatrix} 2 & 0 \\ 0 & 2 \end{pmatrix} c
    = \begin{pmatrix} 1 & 0 \\ 0 & 1 \end{pmatrix} c \\
    &\implies c = \begin{pmatrix} 3 \\ 1 \end{pmatrix} \\
    \\
    y(t) &= \frac{1}{2} \begin{pmatrix} e^{-3t} + e^t & -e^{-3t} + e^t \\ -e^{-3t} + e^t & e^{-3t} + e^t \end{pmatrix} \begin{pmatrix} 3 \\ 1 \end{pmatrix} \\
    &= \begin{pmatrix} e^{-3t} + 2e^t \\ -e^{-3t} + 2e^t \end{pmatrix}
\end{align*}

\newpage
\noindent\textbf{\textit{System b:}}
\begin{align*}
    A &= \begin{pmatrix} 1 & -2 \\ 2 & 1 \end{pmatrix} \\
    P(\lambda) &= \det(A - \lambda I) = \begin{vmatrix} 1 - \lambda & -2 \\ 2 & 1 - \lambda \end{vmatrix} \\
    &= (1 - \lambda)^2 + 4 = \lambda^2 - 2\lambda + 5 \\
    \\
    &\lambda_1,\; \lambda_2 = \frac{2 \pm \sqrt{4 - 20}}{2} \rightarrow \lambda_1 = 1 + 2i, \quad \lambda_2 = 1 - 2i \\
    \\
    N(A - \lambda_1 I) &= N\left(\begin{pmatrix} -2i & -2 \\ 2 & -2i \end{pmatrix}\right) \\
    &= \left\{ \begin{pmatrix} x_1 \\ x_2 \end{pmatrix} : -ix_1 - x_2 = 0 \right\}
    = \text{span}\left\{ \begin{pmatrix} 1 \\ -i \end{pmatrix} \right\} \\
    N(A - \lambda_2 I) &= N\left(\begin{pmatrix} 2i & -2 \\ 2 & 2i \end{pmatrix}\right) \\
    &= \left\{ \begin{pmatrix} x_1 \\ x_2 \end{pmatrix} : ix_1 - x_2 = 0 \right\}
    = \text{span}\left\{ \begin{pmatrix} 1 \\ i \end{pmatrix} \right\} \\
    \\
    &X = \begin{pmatrix} 1 & 1 \\ -i & i \end{pmatrix} , \quad
    X^{-1} = \frac{1}{2i} \begin{pmatrix} i & -1 \\ i & 1 \end{pmatrix} , \quad
    D = \begin{pmatrix} 1 + 2i & 0 \\ 0 & 1 - 2i \end{pmatrix} \\
    &\text{Using the fact that } e^{a + bi} = e^a(\cos b + i \sin b) \\
    e^{Dt} &= \begin{pmatrix} e^t(\cos 2t + i \sin 2t) & 0 \\ 0 & e^t(\cos 2t - i \sin 2t) \end{pmatrix} \\
    &\text{Let } \alpha = \cos 2t + i \sin 2t, \; \beta = \cos 2t - i \sin 2t \\
    y &= e^{At} = X e^{Dt} X^{-1} \\
    &= \begin{pmatrix} 1 & 1 \\ -i & i \end{pmatrix} \begin{pmatrix} e^t \alpha & 0 \\ 0 & e^t \beta \end{pmatrix} \frac{1}{2i} \begin{pmatrix} i & -1 \\ i & 1 \end{pmatrix} \\
    &= \frac{1}{2i} \begin{pmatrix} e^t \alpha & e^t \beta \\ -ie^t \alpha & ie^t \beta \end{pmatrix} \begin{pmatrix} i & -1 \\ i & 1 \end{pmatrix} \\
    &= \frac{1}{2i} \begin{pmatrix} ie^t \alpha + ie^t \beta & -e^t \alpha + e^t \beta \\ e^t \alpha - e^t \beta & ie^t \alpha + ie^t \beta \end{pmatrix} \\
    &= \frac{1}{2i} \begin{pmatrix} ie^t (\cos 2t + i \sin 2t) + ie^t (\cos 2t - i \sin 2t) & -e^t (\cos 2t + i \sin 2t) + e^t (\cos 2t - i \sin 2t) \\ e^t (\cos 2t + i \sin 2t) - e^t (\cos 2t - i \sin 2t) & ie^t (\cos 2t + i \sin 2t) + ie^t (\cos 2t - i \sin 2t) \end{pmatrix} \\
    &= \frac{1}{2i} \begin{pmatrix} 2i e^{t} \cos(2t) & -2i e^{t} \sin(2t) \\ 2i e^{t} \sin(2t) &  2i e^{t} \cos(2t) \end{pmatrix} \\
    &= e^t \begin{pmatrix} \cos(2t) & -\sin(2t) \\ \sin(2t) & \cos(2t) \end{pmatrix} \\
    \\
    y(0) &= \begin{pmatrix} 1 \\ -2 \end{pmatrix} = e^0 \begin{pmatrix} \cos(0) & -\sin(0) \\ \sin(0) & \cos(0) \end{pmatrix} c
    = \begin{pmatrix} 1 & 0 \\ 0 & 1 \end{pmatrix} c \\
    &\implies c = \begin{pmatrix} 1 \\ -2 \end{pmatrix} \\
    \\
    y(t) &= e^t \begin{pmatrix} \cos(2t) & -\sin(2t) \\ \sin(2t) & \cos(2t) \end{pmatrix} \begin{pmatrix} 1 \\ -2 \end{pmatrix} \\
    &= e^t \begin{pmatrix} \cos(2t) + 2\sin(2t) \\ \sin(2t) - 2\cos(2t) \end{pmatrix}
\end{align*}

\newpage
\section*{Question 3}
In each of the following, "diagonalize" the matrix $X$ and use it to compute $A^{-1}$, $A^4$, $e^A$.
\begin{align*}
    A = \begin{pmatrix} 0 & 1 \\ 1 & 0 \end{pmatrix}, \quad
    A = \begin{pmatrix} 2 & 2 & 1 \\ 0 & 1 & 2 \\ 0 & 0 & -1 \end{pmatrix}, \quad
    A = \begin{pmatrix} 1 & 2 & -1 \\ 2 & 4 & -2 \\ 3 & 6 & -3 \end{pmatrix}
\end{align*}

\vspace{0.25cm}\noindent\textbf{Solution:}
Use the fact that $A = XDX^{-1}$, where $X$ is the matrix of eigenvectors of $A$ and $D$ is the diagonal matrix of eigenvalues of $A$.
Then, $A^{-1} = XD^{-1}X^{-1}$, $A^4 = XD^4X^{-1}$, and $e^A = Xe^DX^{-1}$.

\newpage
\noindent\textbf{\textit{First matrix:}}
\begin{align*}
    A &= \begin{pmatrix} 0 & 1 \\ 1 & 0 \end{pmatrix} \\
    P(\lambda) &= \det(A - \lambda I) = \begin{vmatrix} -\lambda & 1 \\ 1 & -\lambda \end{vmatrix} \\
    &= \lambda^2 - 1 = (\lambda - 1)(\lambda + 1) \\
    \\
    &\lambda_1 = 1, \quad \lambda_2 = -1 \\
    \\
    N(A - \lambda_1 I) &= N\left(\begin{pmatrix} -1 & 1 \\ 1 & -1 \end{pmatrix}\right)
    = \left\{ \begin{pmatrix} x_1 \\ x_2 \end{pmatrix} : -x_1 + x_2 = 0 \right\}
    = \text{span}\left\{ \begin{pmatrix} 1 \\ 1 \end{pmatrix} \right\} \\
    N(A - \lambda_2 I) &= N\left(\begin{pmatrix} 1 & 1 \\ 1 & 1 \end{pmatrix}\right)
    = \left\{ \begin{pmatrix} x_1 \\ x_2 \end{pmatrix} : x_1 + x_2 = 0 \right\}
    = \text{span}\left\{ \begin{pmatrix} 1 \\ -1 \end{pmatrix} \right\} \\
    \\
    &X = \begin{pmatrix} 1 & 1 \\ 1 & -1 \end{pmatrix} , \quad
    X^{-1} = -\frac{1}{2} \begin{pmatrix} -1 & -1 \\ -1 & 1 \end{pmatrix} = \frac{1}{2} \begin{pmatrix} 1 & 1 \\ 1 & -1 \end{pmatrix} , \quad
    D = \begin{pmatrix} 1 & 0 \\ 0 & -1 \end{pmatrix} \\
    \\
    A^{-1} &= XD^{-1}X^{-1} = \begin{pmatrix} 1 & 1 \\ 1 & -1 \end{pmatrix} \begin{pmatrix} 1 & 0 \\ 0 & -1 \end{pmatrix} \frac{1}{2} \begin{pmatrix} 1 & 1 \\ 1 & -1 \end{pmatrix} \\
    &= \frac{1}{2} \begin{pmatrix} 1 & -1 \\ 1 & 1 \end{pmatrix} \begin{pmatrix} 1 & 1 \\ 1 & -1 \end{pmatrix} \\
    &= \frac{1}{2} \begin{pmatrix} 0 & 2 \\ 2 & 0 \end{pmatrix} \\
    &= \begin{pmatrix} 0 & 1 \\ 1 & 0 \end{pmatrix} \\
    \\
    A^4 &= XD^4X^{-1} = \begin{pmatrix} 1 & 1 \\ 1 & -1 \end{pmatrix} \begin{pmatrix} 1 & 0 \\ 0 & 1 \end{pmatrix} \frac{1}{2} \begin{pmatrix} 1 & 1 \\ 1 & -1 \end{pmatrix} \\
    &= \frac{1}{2} \begin{pmatrix} 1 & 1 \\ 1 & -1 \end{pmatrix} \begin{pmatrix} 1 & 1 \\ 1 & -1 \end{pmatrix} \\
    &= \frac{1}{2} \begin{pmatrix} 2 & 0 \\ 0 & 2 \end{pmatrix} \\
    &= \begin{pmatrix} 1 & 0 \\ 0 & 1 \end{pmatrix} \\
    \\
    e^A &= Xe^DX^{-1} = \begin{pmatrix} 1 & 1 \\ 1 & -1 \end{pmatrix} \begin{pmatrix} e & 0 \\ 0 & e^{-1} \end{pmatrix} \frac{1}{2} \begin{pmatrix} 1 & 1 \\ 1 & -1 \end{pmatrix} \\
    &= \frac{1}{2} \begin{pmatrix} e & e^{-1} \\ e & -e^{-1} \end{pmatrix} \begin{pmatrix} 1 & 1 \\ 1 & -1 \end{pmatrix} \\
    &= \frac{1}{2} \begin{pmatrix} e + e^{-1} & e - e^{-1} \\ e - e^{-1} & e + e^{-1} \end{pmatrix} \\
\end{align*}

\newpage
\noindent\textbf{\textit{Second matrix:}}
\begin{align*}
    A &= \begin{pmatrix} 2 & 2 & 1 \\ 0 & 1 & 2 \\ 0 & 0 & -1 \end{pmatrix} \\
    P(\lambda) &= \det(A - \lambda I) = \begin{vmatrix} 2 - \lambda & 2 & 1 \\ 0 & 1 - \lambda & 2 \\ 0 & 0 & -1 - \lambda \end{vmatrix} \\
    &= (-1 - \lambda)(2 - \lambda)(1 - \lambda)
    \\
    &\lambda_1 = -1, \quad \lambda_2 = 2, \quad \lambda_3 = 1 \\
    \\
    N(A - \lambda_1 I) &= N\left(\begin{pmatrix} 3 & 2 & 1 \\ 0 & 2 & 2 \\ 0 & 0 & 0 \end{pmatrix}\right) = N\left(\begin{pmatrix} 3 & 1 & 0 \\ 0 & 1 & 1 \\ 0 & 0 & 0 \end{pmatrix}\right) \\
    &= \left\{ \begin{pmatrix} x_1 \\ x_2 \\ x_3 \end{pmatrix} : 3x_1 + x_2 = 0, \; x_2 + x_3 = 0 \right\}
    = \text{span}\left\{ \begin{pmatrix} 1 \\ -3 \\ 3 \end{pmatrix} \right\} \\
    N(A - \lambda_2 I) &= N\left(\begin{pmatrix} 0 & 2 & 1 \\ 0 & -1 & 2 \\ 0 & 0 & -3 \end{pmatrix}\right) = N\left(\begin{pmatrix} 0 & 1 & 0 \\ 0 & 0 & 1 \\ 0 & 0 & 0 \end{pmatrix}\right)
    = \text{span}\left\{ \begin{pmatrix} 1 \\ 0 \\ 0 \end{pmatrix} \right\} \\
    N(A - \lambda_3 I) &= N\left(\begin{pmatrix} 1 & 2 & 1 \\ 0 & 0 & 2 \\ 0 & 0 & -2 \end{pmatrix}\right) = N\left(\begin{pmatrix} 1 & 2 & 0 \\ 0 & 0 & 1 \\ 0 & 0 & 0 \end{pmatrix}\right)
    = \text{span}\left\{ \begin{pmatrix} 2 \\ -1 \\ 0 \end{pmatrix} \right\} \\
    \\
    &X = \begin{pmatrix} 1 & 1 & 2 \\ -3 & 0 & -1 \\ 3 & 0 & 0 \end{pmatrix} , \quad
    X^{-1} = \frac{1}{3} \begin{pmatrix} 0 & 0 & 1 \\ 3 & 6 & 5 \\ 0 & -3 & -3 \end{pmatrix} , \quad
    D = \begin{pmatrix} -1 & 0 & 0 \\ 0 & 2 & 0 \\ 0 & 0 & 1 \end{pmatrix} \\
    \\
    A^{-1} &= XD^{-1}X^{-1} = \begin{pmatrix} 1 & 1 & 2 \\ -3 & 0 & -1 \\ 3 & 0 & 0 \end{pmatrix} \begin{pmatrix} -1 & 0 & 0 \\ 0 & \frac{1}{2} & 0 \\ 0 & 0 & 1 \end{pmatrix} \frac{1}{3} \begin{pmatrix} 0 & 0 & 1 \\ 3 & 6 & 5 \\ 0 & -3 & -3 \end{pmatrix} \\
    \\
    A^4 &= XD^4X^{-1} = \begin{pmatrix} 1 & 1 & 2 \\ -3 & 0 & -1 \\ 3 & 0 & 0 \end{pmatrix} \begin{pmatrix} 1 & 0 & 0 \\ 0 & 16 & 0 \\ 0 & 0 & 1 \end{pmatrix} \frac{1}{3} \begin{pmatrix} 0 & 0 & 1 \\ 3 & 6 & 5 \\ 0 & -3 & -3 \end{pmatrix} \\
    \\
    e^A &= Xe^DX^{-1} = \begin{pmatrix} 1 & 1 & 2 \\ -3 & 0 & -1 \\ 3 & 0 & 0 \end{pmatrix} \begin{pmatrix} e^{-1} & 0 & 0 \\ 0 & e^2 & 0 \\ 0 & 0 & e \end{pmatrix} \frac{1}{3} \begin{pmatrix} 0 & 0 & 1 \\ 3 & 6 & 5 \\ 0 & -3 & -3 \end{pmatrix} \\
\end{align*}

\newpage
\noindent\textbf{\textit{Third matrix:}}
\begin{align*}
    A &= \begin{pmatrix} 1 & 2 & -1 \\ 2 & 4 & -2 \\ 3 & 6 & -3 \end{pmatrix} \\
    P(\lambda) &= \det(A - \lambda I) = \begin{vmatrix} 1 - \lambda & 2 & -1 \\ 2 & 4 - \lambda & -2 \\ 3 & 6 & -3 - \lambda \end{vmatrix} \\
\end{align*}

\newpage
\section*{Question 4}
Let 
$$ A = \begin{pmatrix} 2 & 1 \\ 1 & 1 \\ 2 & 1 \end{pmatrix}, \quad
   b = \begin{pmatrix} 12 \\ 6 \\ 18 \end{pmatrix} $$
\begin{enumerate}[a.]
    \item Use the Gram-Schmidt process to find an orthonormal basis for the column space of $A$.
    \item Factor $A$ into $QR$.
    \item Use the above to solve the system $Ax = b$.
\end{enumerate}

\vspace{0.25cm}\noindent\textbf{Solution:}

\section*{Question 5}
Let $\{x_1, x_2, x_3\} = \{(0, 1, 0), (2, 1, 2), (0, 0, 1)\}$, be a basis of $\mathbb{R}^3$.
\begin{enumerate}[a.]
    \item Use the Gram-Schmidt process to obtain an orthonormal basis.
    \item Let $ b := (1, 1, 1) $. Compute the projection of $ b $ onto $\text{span}\{x_1, x_2\}$ and to $\text{span}\{x_3, x_2\}$.
\end{enumerate}

\vspace{0.25cm}\noindent\textbf{Solution:}

\section*{Question 6}
Consider the vector space \(C[0,1]\) with the inner product
$$ \langle f, g \rangle = \int_{0}^{1} f(x)g(x)dx. $$
\begin{enumerate}[a.]
    \item Find an orthonormal basis of the subspace \(E\) spanned by \(1, x, x^2\).
    \item Compute the length of \(2x^2 + 3\).
    \item Compute the projection of \(e^x\) onto \(E\)
\end{enumerate}

\vspace{0.25cm}\noindent\textbf{Solution:}

\section*{Question 7}
Find the orthogonal complement of the subspace of \(\mathbb{R}^4\) spanned by \((1, 1, 1, 1), (1, -1, 1, -1)\).

\vspace{0.25cm}\noindent\textbf{Solution:}

\section*{Question 8}
For each of the following systems \(Ax = b\) find all least squares solutions:
\begin{align*}
    A = \begin{pmatrix} 0 & 1 \\ 1 & 1 \\ 2 & 1 \end{pmatrix}, \quad 
    b = \begin{pmatrix} 6 \\ 0 \\ 0 \end{pmatrix}, \quad \text{and} \quad
    A = \begin{pmatrix} 1 & 1 & 0 \\ 0 & 0 & 1 \\ 0 & 1 & 2 \\ 2 & 2 & 1 \end{pmatrix}, \quad 
    b = \begin{pmatrix} 2 \\ 1 \\ 1 \\ 0 \end{pmatrix}
\end{align*}

\vspace{0.25cm}\noindent\textbf{Solution:}

\end{document}
