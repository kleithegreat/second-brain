\documentclass{article}
\usepackage{amsmath,amssymb,amsthm}
\usepackage{fancyhdr}
\usepackage{enumerate}
\usepackage{algorithm2e}

\title{Project of Algorithms on Condition Satisfiability}
\author{Kevin Lei}
\date{July 22, 2024}

\begin{document}

\maketitle

\section{Main Idea}

In the ``Condition Satisfiability Problem", we are given the following:
\begin{enumerate}
    \item A number of $n$ boolean variables $x_1, x_2, \ldots, x_n$, where $x_i$ must be either true or false.
    \item A set of $P$ ``lead-to'' conditions $L = \{T_1, T_2, \ldots, T_P\}$, where each $T_i$ has $k_i \geq 0$ boolean variables to the left of its $\Rightarrow$ symbol and always one boolean variable to the right.
    A ``lead-to" condition takes the form of $(x_{i_1} \land x_{i_2} \land \ldots \land x_{i_k}) \Rightarrow x_j$.
    The $k_i + 1$ variables for each $T_i \in L$ are unique, and the degenerate case with $k=0$ simply means that $x_j$ is true.
    \item A set of $Q$ ``false-must-exist" conditions $F = \{M_1, M_2, \ldots, M_Q\}$, where each $M_i$ is a cumulative logical OR of $m_i > 0$ negated boolean variables.
    A ``false-must-exist" condition takes the form of $(\neg x_{i_1} \lor \neg x_{i_2} \lor \ldots \lor \neg x_{m})$.
    The $m_i$ variables for each $M_i \in F$ are unique.
\end{enumerate}

We want to find the truth values of $x_1, x_2, \ldots, x_n$ such that the $P$ conditions in $L$ and the $Q$ conditions in $F$ will all evaluate to true.
Such an assignment of truth values is called a ``satisfying solution".
If there is no such assignment, then we want to output ``\textit{No satisfying solution exists}".
\vspace{1em}
\newline\noindent
The main idea behind this algorithm is to initialize a set of $n$ boolean variables to be all false.
Then, we iterate through all of the conditions until no more changes are being made.
We check the ``lead-to" conditions first, and if any has a true left-hand side and a false right-hand side, then we set the variable on the right-hand side to true.
We then check the ``false-must-exist" conditions, and if any has all of its variables to be true, then we set the first variable in the statement to false.
Once no more changes can be made according to our rules, we do a final check of all the conditions.
If any of them fail, then we output ``\textit{No satisfying solution exists}".
After we check all of the conditions and they all pass, then we output the satisfying solution.

\section{Pseudocode}

\RestyleAlgo{ruled}
\begin{algorithm}[H]
\caption{Condition Satisfiability}
\KwIn{$n$ boolean variables, a set $L = \{T_1, T_2, \ldots, T_P\}$ of $P$ ``lead-to" conditions, and a set $F = \{M_1, M_2, \ldots, M_Q\}$ of $Q$ ``false-must-exist" conditions.}
\KwOut{A satisfying solution or ``No satisfying solution exists".}

$x_1, x_2, \ldots, x_n$ = false\;
changed = true\;

\While{changed}{
    changed = false\;
    
    \For{each $T_i \in L$}{
        \If{$(x_{i_1} \land x_{i_2} \land \ldots \land x_{i_k}) \land \neg x_j$}{
            $x_j$ = true\;
            changed = true\;
        }
    }
    
    \For{each $M_i \in F$}{
        \If{$\neg M_i$}{
            $x_{i_1}$ = false\;
            changed = true\;
        }
    }
}

\For{each $T_i \in L$}{
    \If{$(x_{i_1} \land x_{i_2} \land \ldots \land x_{i_k}) \land \neg x_j$}{
        \Return ``No satisfying solution exists"\;
    }
}

\For{each $M_i \in F$}{
    \If{$\neg M_i$}{
        \Return ``No satisfying solution exists"\;
    }
}

\Return $x_1, x_2, \ldots, x_n$\;
\end{algorithm}

\section{Proof of Correctness}

\section{Runtime Complexity Analysis}

\end{document}