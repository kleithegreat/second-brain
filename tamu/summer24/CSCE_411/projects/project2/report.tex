\documentclass{article}
\usepackage{amsmath,amssymb,amsthm}
\usepackage{fancyhdr}
\usepackage{enumerate}
\usepackage[ruled]{algorithm2e}

\title{Project of Algorithms on Node Labeling}
\author{Kevin Lei}
\date{August 2, 2024}

\newtheorem{theorem}{Theorem}

\begin{document}

\maketitle

\section{Introduction}

In this project we discuss the ``Node Labeling Problem'', in which we attempt to label the nodes of a graph with unique labels from a set of labels.
We have the following definitions:
\begin{itemize}
    \item Let $G = (V, E)$ be an undirected graph.
    \item Let $d(u, v)$ be the distance between nodes $u$ and $v$.
    \item For all nodes $v \in V$, let $N(v, h) \subseteq V$ be the set of nodes that are at most $h$ hops away from $v$.
    \item Let $K = \{0, 1, \ldots, k-1\}$ be the set of $k$ integers, where $k \leq |V|$.
    \item For all $v \in V$, let $c(v) \in K$ be the label of node $v$, where different nodes may have the same label.
    \item Let $C(v, h)$ be the set of labels of nodes in $N(v, h)$.
    \item A labeling of the nodes is \textit{valid} if every label in $K$ is used at least once.
    \item Let $r(v)$ be the smallest integer such that the node $v$ has all the labels in $K$ in $N(v, r(v))$.
    \item Let $m(v)$ be the smallest integer such that the node $v$ has at least $k$ nodes in $N(v, m(v))$.
\end{itemize}

Formally, our relevant sets and values can be defined as follows:
\begin{align*}
    N(v, h) &\triangleq \{u \in V \mid d(u, v) \leq h\} \\
    C(v, h) &\triangleq \{c(u) \mid u \in N(v, h)\} \\
    r(v) &\triangleq \min\{h \mid |C(v, h)| = k\} \\
    m(v) &\triangleq \min\{h \mid |N(v, h)| \geq k\}.
\end{align*}

Note that in general, we have $|C(v, h)| \leq |N(v, h)|$, since the labels of nodes in $N(v, h)$ are not necessarily distinct, 
and $r(v) \geq m(v)$, since there must be at least one node per label but not necessarily one label per node.

\newpage\noindent
The Node-Labeling Decision Problem is defined as follows:

\vspace{1em}\noindent
Given:
\begin{itemize}
    \item An undirected graph $G = (V, E)$
    \item A set of $k \leq |V|$ labels $K = \{0, 1, \ldots, k-1\}$
    \item A nonnegative integer $R$,
\end{itemize}
does there exist a labeling $c(v)$ for all $v \in V$ such that $|C(v, R)| = k$ for all $v \in V$?

\vspace{1em}\noindent
Now consider this as an optimization problem.
The Node-Labeling Optimization Problem is defined as follows:

\vspace{1em}\noindent
Given:
\begin{itemize}
    \item An undirected graph $G = (V, E)$
    \item A set of $k \leq |V|$ labels $K = \{0, 1, \ldots, k-1\}$,
\end{itemize}
find a valid labeling for all the nodes such that $\max_{v \in V} \frac{r(v)}{m(v)}$ is minimized.

\vspace{1em}

In the case of the optimization problem, if an algorithm that solves it has $\max_{v \in V} \frac{r(v)}{m(v)} \leq \rho$ for all possible instances, 
then we say that the algorithm has a \textit{proximity ratio} of $\rho$, and the algorithm is a $\rho$-proximity algorithm.

\vspace{1em}

First, we will prove that the Node-Labeling Decision Problem is NP-Complete.
Then, we will present a polynomial-time algorithm for the Node-Labeling Optimization Problem where the graph is a tree,
analyze the proximity ratio of the algorithm, and finally analyze the runtime complexity of the algorithm.

\section{NP-Completeness Proof}

\begin{theorem}
    The Node-Labeling Decision Problem is NP-Complete.
\end{theorem}

\begin{proof}
    A problem is NP-Complete if it is in NP and every problem in NP can be reduced to it in polynomial time.
    We will show the former by presenting a polynomial time algorithm to verify a solution to the Node-Labeling Decision Problem,
    and the latter by reducing HAM-CYCLE to the Node-Labeling Decision Problem.
    
    First, consider the following algorithm to verify a solution to the Node-Labeling Decision Problem:

    \begin{algorithm}[H]
        \caption{Verify a Solution to the Node-Labeling Decision Problem}
        \KwIn{An undirected graph $G = (V, E)$, a set of $k \leq |V|$ labels $K = \{0, 1, \ldots, k-1\}$, a nonnegative integer $R$, and a labeling $c: V \to K$}
        \KwOut{True if the labeling is valid and $|C(v, R)| = k$ for all $v \in V$, False otherwise}
        \BlankLine
        
        \For{$v \in V$}{
            $l = \emptyset$
            
            \If{$\neg$ BFS($v$, $0$, $l$)}{
                \Return False
            }
        }
        \Return True
    \end{algorithm}
    
    \begin{algorithm}[H]
        \caption{BFS}
        \KwIn{A node $v$, current depth $d$, set of labels seen $l$}
        \KwOut{True if all $k$ labels are seen within depth $R$, False otherwise}
        \BlankLine
        
        \If{$d > R$}{
            \Return False
        }
        
        $l = l \cup \{c(v)\}$
        
        \If{$|l| = k$}{
            \Return True
        }
        
        \For{each neighbor $u$ of $v$}{
            \If{BFS($u$, $d + 1$, $l$)}{
                \Return True
            }
        }
        
        \Return False
    \end{algorithm}
    
    This algorithm works by performing a breadth-first search from each node $v$ in the graph,
    and checking if all $k$ labels are seen within depth $R$.
    If a depth of $R$ is reached without seeing all $k$ labels, the algorithm returns False.
    Otherwise, the algorithm returns True.
    The algorithm runs in $O(|V| \cdot (|V| + |E|))$ time, which is polynomial in the size of the input.
    Thus, the Node-Labeling Decision Problem is in NP.

    Now we perform a reduction from HAM-CYCLE to the Node-Labeling Decision Problem.
    Given an instance of HAM-CYCLE with the graph $G = (V, E)$, we construct an instance of the Node-Labeling Decision Problem as follows:
    \begin{itemize}
        \item The graph is the same: $G = (V, E)$.
        \item $k = |V|$.
        \item $R = |V| - 1$.
    \end{itemize}
    We claim that there exists a Hamiltonian cycle in $G$ if and only if there exists a valid labeling for the corresponding instance of the Node-Labeling Decision Problem.
    \paragraph{($\Rightarrow$)} Assume there exists a Hamiltonian cycle in $G$.
    Label the nodes in the Hamiltonian cycle from $0$ to $|V| - 1$ in order.
    Thus, for all $v \in V$, all other nodes are at least $|V| - 1$ hops away.
    This means that $N(v, R)$ contains all nodes in $V$, so $|C(v, R)| = |V| = k$.
    Thus, the labeling is valid.
    \paragraph{($\Leftarrow$)} Assume that there exists a labeling $c(V)$ for all $v \in V$ such that $|C(v, R)| = k$ for all $v \in V$ where $k = |V|$ and $R = |V| - 1$.
    This means that for any node $v$, it can see all other nodes within $|V| - 1$ hops.
    By the way the reduction is constructed, this is only possible if there is a path starting from $v$ that visits all other nodes exactly once.
    In other words, there exists a Hamiltonian cycle in $G$.
    
    \vspace{1em}
    This reduction can be done in polynomial time, since the graph is the same, and if we really need to, we can count $|V|$ in $O(|V|)$ time.
    Since the Node-Labeling Decision Problem is in NP and HAM-CYCLE can be reduced to it in polynomial time, the Node-Labeling Decision Problem is NP-Complete.
\end{proof}

\section{Approximation Algorithm}

Here we discuss an algorithm to solve the Node-Labeling \textit{Optimization} Problem when the input graph is a tree.


\section{Pseudocode}

\begin{algorithm}[H]
\caption{}
\KwIn{}
\KwOut{}
\BlankLine

\end{algorithm}

\section{Proximity Ratio Analysis}



\section{Runtime Complexity Analysis}



\end{document}
