\documentclass{article}
\usepackage{amsmath,amssymb,amsthm}
\usepackage{fancyhdr}
\usepackage{enumerate}
\usepackage[ruled]{algorithm2e}

\title{Project of Algorithms on Node Labeling}
\author{Kevin Lei}
\date{August 2, 2024}

\newtheorem{theorem}{Theorem}

\begin{document}

\maketitle

\section{Introduction}

In this project we discuss the ``Node Labeling Problem'', in which we attempt to label the nodes of a graph with unique labels from a set of labels.
We have the following definitions:
\begin{itemize}
    \item Let $G = (V, E)$ be an undirected graph.
    \item Let $d(u, v)$ be the distance between nodes $u$ and $v$.
    \item For all nodes $v \in V$, let $N(v, h) \subseteq V$ be the set of nodes that are at most $h$ hops away from $v$.
    \item Let $K = \{0, 1, \ldots, k-1\}$ be the set of $k$ integers, where $k \leq |V|$.
    \item For all $v \in V$, let $c(v) \in K$ be the label of node $v$, where different nodes may have the same label.
    \item Let $C(v, h)$ be the set of labels of nodes in $N(v, h)$.
    \item A labelling of the nodes is \textit{valid} if every label in $K$ is used at least once.
    \item Let $r(v)$ be the smallest integer such that the node $v$ has all the labels in $K$ in $N(v, r(v))$.
    \item Let $m(v)$ be the smallest integer such that the node $v$ has at least $k$ nodes in $N(v, m(v))$.
\end{itemize}

Formally, our relevant sets and values can be defined as follows:
\begin{align*}
    N(v, h) &\triangleq \{u \in V \mid d(u, v) \leq h\} \\
    C(v, h) &\triangleq \{c(u) \mid u \in N(v, h)\} \\
    r(v) &\triangleq \min\{h \mid |C(v, h)| = k\} \\
    m(v) &\triangleq \min\{h \mid |N(v, h)| \geq k\}.
\end{align*}

Note that in general, we have $|C(v, h)| \leq |N(v, h)|$, since the labels of nodes in $N(v, h)$ are not necessarily distinct, 
and $r(v) \geq m(v)$, since there must be at least one label per node.

\newpage\noindent
The Node-Labelling Decision Problem is defined as follows:

\vspace{1em}\noindent
Given:
\begin{itemize}
    \item An undirected graph $G = (V, E)$
    \item A set of $k \leq |V|$ labels $K = \{0, 1, \ldots, k-1\}$
    \item A nonnegative integer $R$,
\end{itemize}
does there exist a labeling $c(v)$ for all $v \in V$ such that $|C(v, R)| = k$ for all $v \in V$?

\vspace{1em}\noindent
Now consider this as an optimization problem.
The Node-Labelling Optimization Problem is defined as follows:

\vspace{1em}\noindent
Given:
\begin{itemize}
    \item An undirected graph $G = (V, E)$
    \item A set of $k \leq |V|$ labels $K = \{0, 1, \ldots, k-1\}$,
\end{itemize}
find a valid labeling for all the nodes such that $\max_{v \in V} \frac{r(v)}{m(v)}$ is minimized.

\vspace{1em}

In the case of the optimization problem, if an algorithm that solves it has $\max_{v \in V} \frac{r(v)}{m(v)} \leq \rho$ for all possible instances, 
then we say that the algorithm has a \textit{proximity ratio} of $\rho$, and the algorithm is a $\rho$-proximity algorithm.

\vspace{1em}

First, we will prove that the Node-Labelling Decision Problem is NP-Complete.
Then, we will present a polynomial-time algorithm for the Node-Labelling Optimization Problem where the graph is a tree,
analyze the proximity ratio of the algorithm, and finally analyze the runtime complexity of the algorithm.

\section{NP-Completeness Proof}

\begin{theorem}
    The Node-Labelling Decision Problem is NP-Complete.
\end{theorem}

\begin{proof}
    A problem is NP-Complete if it is in NP and every problem in NP can be reduced to it in polynomial time.
    We will show the former by presenting a polynomial time algorithm to verify a solution to the Node-Labelling Decision Problem,
    and the latter by reducing ... to the Node-Labelling Decision Problem.
\end{proof}

\section{Main Idea of the Algorithm}



\section{Pseudocode}

\begin{algorithm}[H]
\caption{}
\KwIn{}
\KwOut{}
\BlankLine

\end{algorithm}

\section{Proximity Ratio Analysis}



\section{Runtime Complexity Analysis}



\end{document}
