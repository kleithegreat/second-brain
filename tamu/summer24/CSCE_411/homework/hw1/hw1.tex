\documentclass{article}
\usepackage{amsmath,amssymb,amsthm}
\usepackage{fancyhdr}
\usepackage{enumerate}
\usepackage{algorithm2e}

\pagestyle{fancy}
\fancyhf{}
\lhead{1st Homework - CSCE 411 700}
\rhead{Kevin Lei}
\renewcommand{\headrulewidth}{0.4pt}
\renewcommand{\arraystretch}{1.2}

\begin{document}

\section{Main Idea}

In the "Maximum Sum Contiguous Subsequence" problem, we are given a sequence $S$ of $n$ numbers, and asked to find the contiguous subsequence with the largest sum.
Formally, we want to find $$\arg\max_{i, j} \sum_{k=i}^{j} S[k]$$ where $1 \leq i \leq j \leq n$.
To solve this, we can apply dynamic programming, since this problem exhibits optimal substructure and overlapping subproblems.
The main idea of this algorithm is keeping track of positive contributions to the sum.
First, we initialize variables to keep track of the maximum sum, the current sum, and the indices of the maximum sum subsequence.
Then, we loop through the sequence for each ending index $j \in \{1, 2, \ldots, n\}$.
On each iteration, we add the current number to the current sum.
If the current sum is greater than the maximum sum, we update the maximum sum and the indices of the maximum sum subsequence.
If the current sum becomes negative, we reset the current sum to 0 and update the starting index of the subsequence to the next index.
At the end of the loop, we use the saved indices to return the maximum sum subsequence.

\section{Pseudocode}

\RestyleAlgo{ruled}
\begin{algorithm}[H]
\caption{Maximum Sum Contiguous Subsequence}
\KwIn{A sequence $S$ of $n$ numbers}
\KwOut{A contiguous subsequence of $S$ with the largest sum}
$maxSum = -\infty$\;
$currentSum = 0$\;
$i_{max}$, $j_{max}$, $i = 1$\;
\For{$j = 1$ \KwTo $n$}{
    $currentSum$ += $S[j]$\;
    \If{$currentSum > maxSum$}{
        $maxSum = currentSum$\;
        $i_{max} = i$\;
        $j_{max} = j$\;
    }
    \If{$currentSum < 0$}{
        $currentSum = 0$\;
        $i = j$\;
    }
}
\Return{$S[i_{max}: j_{max}]$}
\end{algorithm}

\section{Proof of Correctness}

We will prove the correctness of the algorithm by induction.

\begin{proof}

\end{proof}

\section{Runtime Analysis}

\end{document}