\documentclass{article}
\usepackage{amsmath,amssymb,amsthm}
\usepackage{fancyhdr}
\usepackage{enumerate}
\usepackage[ruled]{algorithm2e}

\pagestyle{fancy}
\fancyhf{}
\lhead{6th Homework - CSCE 411 700}
\rhead{Kevin Lei}
\renewcommand{\headrulewidth}{0.4pt}
\renewcommand{\arraystretch}{1.2}

\newtheorem{theorem}{Theorem}

\begin{document}

\noindent
In this exercise, we want to show that the subgraph isomorphism problem is NP-complete.
The subgraph isomorphism problem takes two undirected graphs $G_1$ and $G_2$ as input and asks whether $G_1$ is isomorphic to a subgraph of $G_2$.

\begin{theorem}
The subgraph isomorphism problem is NP-complete.
\end{theorem}

\begin{proof}
    A problem is NP-complete if and only if it is in NP and it is NP-hard.
    First, we show that the subgraph isomorphism problem is in NP, i.e. we can verify a proposed solution in polynomial time.
    Given graphs $G_1$ and $G_2$ and a mapping $f: V_{G_1} \to V_{G_2}$, we must determine the following in polynomial time:
    \begin{enumerate}
        \item Whether $f$ is an injection.
        \item Whether for all edges $(u, v) \in E_{G_1}$, $(f(u), f(v)) \in E_{G_2}$.
    \end{enumerate}
    The function $f$ is an injection if and only if for all $u, v \in V_{G_1}$, $f(u) = f(v)$ implies $u = v$.
    A simple algorithm to verify this is as follows:

    \begin{algorithm}[H]
        \caption{Verify Injection}
        \KwIn{Graphs $G_1$ and $G_2$, mapping $f: V_{G_1} \to V_{G_2}$}
        \KwOut{Whether $f$ is an injection}
        Initialize hash map $H$\;
        \For{$u \in V_{G_1}$}{
            \If{$H[f(u)]$ is defined}{
                \Return false\;
            }
            $H[f(u)] = u$\;
        }
        \Return true\;
    \end{algorithm}

    The runtime of this algorithm is bounded solely by the number of vertices in $G_1$, i.e. $O(|V_{G_1}|)$, so it runs in polynomial time.
    To verify the second condition, we simply iterate over all edges in $G_1$ and check whether the corresponding edges exist in $G_2$.
    A simple algorithm to verify this is as follows:

    \begin{algorithm}[H]
        \caption{Verify Edge Mapping}
        \KwIn{Graphs $G_1$ and $G_2$, mapping $f: V_{G_1} \to V_{G_2}$}
        \KwOut{Whether $f$ maps edges of $G_1$ to edges of $G_2$}
        Initialize hash set $S$ from $E_{G_2}$\;
        \For{$(u, v) \in E_{G_1}$}{
            \If{$(f(u), f(v)) \notin S$}{
                \Return false\;
            }
        }
        \Return true\;
    \end{algorithm}

    The runtime of this algorithm is bounded by the number of edges in $G_1$, i.e. $O(|E_{G_1}|)$, so it runs in polynomial time.
    Therefore, the subgraph isomorphism problem is in NP.

    \noindent Next, to show that the subgraph isomorphism problem is NP-hard, we reduce the clique problem to the subgraph isomorphism problem.
    The clique problem takes an undirected graph $G$ and an integer $k$ as input and asks whether $G$ contains a clique of size $k$, where a clique is a complete subgraph.
    Given an instance of the clique problem $(G, k)$, we construct the following instance of the subgraph isomorphism problem $(G_1, G_2)$:

    \begin{algorithm}
        \caption{Construct Subgraph Isomorphism Instance}
        \KwIn{Graph $G$, integer $k$}
        \KwOut{Graphs $G_1$, $G_2$}
        Initialize graph $G_1$\;
        \For{$i = 1$ to $k$}{
            Add vertex $v_i$ to $G_1$\;
        }
        \For{$i = 1$ to $k$}{
            \For{$j = 1$ to $k$}{
                \If{$i \neq j$}{
                    Add edge $(v_i, v_j)$ to $G_1$\;
                }
            }
        }
        Initialize graph $G_2$ as a copy of $G$\;
        \Return $(G_1, G_2)$\;
    \end{algorithm}

    This construction algorithm runs in polynomial time due to the following factors:
    
    \begin{enumerate}
        \item Creating $G_1$ (a complete graph with $k$ vertices) takes $O(k^2)$ time.
        \item $G_2$ is just a copy of the original graph $G$, so copying it takes $O(|V_G| + |E_G|)$ time.
    \end{enumerate}

    Now, we need to show that this reduction is correct, i.e., $G$ has a clique of size $k$ if and only if $G_1$ is isomorphic to a subgraph of $G_2$.
    First we show the forward direction:
    \paragraph{($\Rightarrow$)}
    Assume that $G$ contains a clique of size $k$.
    That means there exists some subgraph of $G$ that is a complete graph on $k$ vertices.
    By construction, $G_1$ is a complete graph on $k$ vertices, so $G_1$ is isomorphic to this subgraph of $G$.
    Since $G_2$ is a copy of $G$, $G_1$ is isomorphic to a subgraph of $G_2$.
    This proves that this reduction preserves the answer in the forward direction.
    
    \vspace{1em}
    \noindent
    And now the backward direction:
    \paragraph{($\Leftarrow$)}
    Now assume that $G_1$ is isomorphic to a subgraph of $G_2$.
    By construction, $G_1$ is a complete graph on $k$ vertices.
    By our initial assumption, there then exists a clique of size $k$ in $G_2$.
    Since $G_2$ is a copy of $G$, there exists a clique of size $k$ in $G$,
    and thus the reduction preserves the answer in both directions.

    \vspace{1em}
    \noindent
    At this point we have proven the following:
    \begin{enumerate}
        \item There exists a polynomial-time checking algorithm for the subgraph isomorphism problem.
        \item The reduction from the clique problem to the subgraph isomorphism problem is correct.
        \item The reduction runs in polynomial time.
    \end{enumerate}
    Therefore, the subgraph isomorphism problem is in NP and is NP-hard, so it is NP-complete.
\end{proof}

\end{document}
