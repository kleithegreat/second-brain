\documentclass{article}
\usepackage{amsmath,amssymb,amsthm}
\usepackage{fancyhdr}
\usepackage{enumerate}
\usepackage[ruled]{algorithm2e}

\pagestyle{fancy}
\fancyhf{}
\lhead{6th Homework - CSCE 411 700}
\rhead{Kevin Lei}
\renewcommand{\headrulewidth}{0.4pt}
\renewcommand{\arraystretch}{1.2}

\newtheorem{theorem}{Theorem}

\begin{document}

\noindent In this exercise, we want to show that the subgraph isomorphism problem is NP-complete.
The subgraph isomorphism problem takes two undirected graphs $G_1$ and $G_2$ as input and asks whether $G_1$ is isomorphic to a subgraph of $G_2$.

\begin{theorem}
The subgraph isomorphism problem is NP-complete.
\end{theorem}

\begin{proof}
    A problem is NP-complete if it is in NP and every problem in NP is reducible to it in polynomial time.
    First, we show that the subgraph isomorphism problem is in NP, i.e. we can verify a proposed solution in polynomial time.
    Given graphs $G_1$ and $G_2$ and a mapping $f: V_{G_1} \to V_{G_2}$, we must determine the following in polynomial time:
    \begin{enumerate}
        \item Whether $f$ is an injection.
        \item Whether for all edges $(u, v) \in E_{G_1}$, $(f(u), f(v)) \in E_{G_2}$.
    \end{enumerate}
    The function $f$ is an injection if and only if for all $u, v \in V_{G_1}$, $f(u) = f(v)$ implies $u = v$.
    A simple algorithm to verify this is as follows:

    \begin{algorithm}[H]
        \caption{Verify Injection}
        \KwIn{Graphs $G_1$ and $G_2$, mapping $f: V_{G_1} \to V_{G_2}$}
        \KwOut{Whether $f$ is an injection}
        Initialize hash map $H$\;
        \For{$u \in V_{G_1}$}{
            \If{$H[f(u)]$ is defined}{
                \Return false\;
            }
            $H[f(u)] = u$\;
        }
        \Return true\;
    \end{algorithm}

    The runtime of this algorithm is bounded solely by the number of vertices in $G_1$, i.e. $O(|V_{G_1}|)$, so it runs in polynomial time.
    To verify the second condition, we simply iterate over all edges in $G_1$ and check whether the corresponding edges exist in $G_2$.
    A simple algorithm to verify this is as follows:

    \begin{algorithm}[H]
        \caption{Verify Edge Mapping}
        \KwIn{Graphs $G_1$ and $G_2$, mapping $f: V_{G_1} \to V_{G_2}$}
        \KwOut{Whether $f$ maps edges of $G_1$ to edges of $G_2$}
        Initialize hash set $S$ from $E_{G_2}$\;
        \For{$(u, v) \in E_{G_1}$}{
            \If{$(f(u), f(v)) \notin S$}{
                \Return false\;
            }
        }
        \Return true\;
    \end{algorithm}

    The runtime of this algorithm is bounded by the number of edges in $G_1$, i.e. $O(|E_{G_1}|)$, so it runs in polynomial time.
    Therefore, the subgraph isomorphism problem is in NP.

    Next, we want to show the following:
    \begin{enumerate}
        \item The NP-complete problem 3-SAT can be reduced to the subgraph isomorphism problem.
        \item The reduction from 3-SAT to the subgraph isomorphism problem preserves answers.
        \item The reduction from 3-SAT to the subgraph isomorphism problem can be done in polynomial time.
    \end{enumerate}

    Given a 3-SAT formula $\varphi$ with $n$ boolean variables $x_1, x_2, \ldots, x_n$ and $m$ clauses $C_1, C_2, \ldots, C_m$, we can construct the graphs $G_1$ and $G_2$ as follows:
\end{proof}

\end{document}
