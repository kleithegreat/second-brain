% Options for packages loaded elsewhere
\PassOptionsToPackage{unicode}{hyperref}
\PassOptionsToPackage{hyphens}{url}
%
\documentclass[
]{article}
\usepackage{amsmath,amssymb}
\usepackage{iftex}
\ifPDFTeX
  \usepackage[T1]{fontenc}
  \usepackage[utf8]{inputenc}
  \usepackage{textcomp} % provide euro and other symbols
\else % if luatex or xetex
  \usepackage{unicode-math} % this also loads fontspec
  \defaultfontfeatures{Scale=MatchLowercase}
  \defaultfontfeatures[\rmfamily]{Ligatures=TeX,Scale=1}
\fi
\usepackage{lmodern}
\ifPDFTeX\else
  % xetex/luatex font selection
\fi
% Use upquote if available, for straight quotes in verbatim environments
\IfFileExists{upquote.sty}{\usepackage{upquote}}{}
\IfFileExists{microtype.sty}{% use microtype if available
  \usepackage[]{microtype}
  \UseMicrotypeSet[protrusion]{basicmath} % disable protrusion for tt fonts
}{}
\makeatletter
\@ifundefined{KOMAClassName}{% if non-KOMA class
  \IfFileExists{parskip.sty}{%
    \usepackage{parskip}
  }{% else
    \setlength{\parindent}{0pt}
    \setlength{\parskip}{6pt plus 2pt minus 1pt}}
}{% if KOMA class
  \KOMAoptions{parskip=half}}
\makeatother
\usepackage{xcolor}
\usepackage{color}
\usepackage{fancyvrb}
\newcommand{\VerbBar}{|}
\newcommand{\VERB}{\Verb[commandchars=\\\{\}]}
\DefineVerbatimEnvironment{Highlighting}{Verbatim}{commandchars=\\\{\}}
% Add ',fontsize=\small' for more characters per line
\newenvironment{Shaded}{}{}
\newcommand{\AlertTok}[1]{\textcolor[rgb]{1.00,0.00,0.00}{\textbf{#1}}}
\newcommand{\AnnotationTok}[1]{\textcolor[rgb]{0.38,0.63,0.69}{\textbf{\textit{#1}}}}
\newcommand{\AttributeTok}[1]{\textcolor[rgb]{0.49,0.56,0.16}{#1}}
\newcommand{\BaseNTok}[1]{\textcolor[rgb]{0.25,0.63,0.44}{#1}}
\newcommand{\BuiltInTok}[1]{\textcolor[rgb]{0.00,0.50,0.00}{#1}}
\newcommand{\CharTok}[1]{\textcolor[rgb]{0.25,0.44,0.63}{#1}}
\newcommand{\CommentTok}[1]{\textcolor[rgb]{0.38,0.63,0.69}{\textit{#1}}}
\newcommand{\CommentVarTok}[1]{\textcolor[rgb]{0.38,0.63,0.69}{\textbf{\textit{#1}}}}
\newcommand{\ConstantTok}[1]{\textcolor[rgb]{0.53,0.00,0.00}{#1}}
\newcommand{\ControlFlowTok}[1]{\textcolor[rgb]{0.00,0.44,0.13}{\textbf{#1}}}
\newcommand{\DataTypeTok}[1]{\textcolor[rgb]{0.56,0.13,0.00}{#1}}
\newcommand{\DecValTok}[1]{\textcolor[rgb]{0.25,0.63,0.44}{#1}}
\newcommand{\DocumentationTok}[1]{\textcolor[rgb]{0.73,0.13,0.13}{\textit{#1}}}
\newcommand{\ErrorTok}[1]{\textcolor[rgb]{1.00,0.00,0.00}{\textbf{#1}}}
\newcommand{\ExtensionTok}[1]{#1}
\newcommand{\FloatTok}[1]{\textcolor[rgb]{0.25,0.63,0.44}{#1}}
\newcommand{\FunctionTok}[1]{\textcolor[rgb]{0.02,0.16,0.49}{#1}}
\newcommand{\ImportTok}[1]{\textcolor[rgb]{0.00,0.50,0.00}{\textbf{#1}}}
\newcommand{\InformationTok}[1]{\textcolor[rgb]{0.38,0.63,0.69}{\textbf{\textit{#1}}}}
\newcommand{\KeywordTok}[1]{\textcolor[rgb]{0.00,0.44,0.13}{\textbf{#1}}}
\newcommand{\NormalTok}[1]{#1}
\newcommand{\OperatorTok}[1]{\textcolor[rgb]{0.40,0.40,0.40}{#1}}
\newcommand{\OtherTok}[1]{\textcolor[rgb]{0.00,0.44,0.13}{#1}}
\newcommand{\PreprocessorTok}[1]{\textcolor[rgb]{0.74,0.48,0.00}{#1}}
\newcommand{\RegionMarkerTok}[1]{#1}
\newcommand{\SpecialCharTok}[1]{\textcolor[rgb]{0.25,0.44,0.63}{#1}}
\newcommand{\SpecialStringTok}[1]{\textcolor[rgb]{0.73,0.40,0.53}{#1}}
\newcommand{\StringTok}[1]{\textcolor[rgb]{0.25,0.44,0.63}{#1}}
\newcommand{\VariableTok}[1]{\textcolor[rgb]{0.10,0.09,0.49}{#1}}
\newcommand{\VerbatimStringTok}[1]{\textcolor[rgb]{0.25,0.44,0.63}{#1}}
\newcommand{\WarningTok}[1]{\textcolor[rgb]{0.38,0.63,0.69}{\textbf{\textit{#1}}}}
\setlength{\emergencystretch}{3em} % prevent overfull lines
\providecommand{\tightlist}{%
  \setlength{\itemsep}{0pt}\setlength{\parskip}{0pt}}
\setcounter{secnumdepth}{-\maxdimen} % remove section numbering
\ifLuaTeX
  \usepackage{selnolig}  % disable illegal ligatures
\fi
\IfFileExists{bookmark.sty}{\usepackage{bookmark}}{\usepackage{hyperref}}
\IfFileExists{xurl.sty}{\usepackage{xurl}}{} % add URL line breaks if available
\urlstyle{same}
\hypersetup{
  hidelinks,
  pdfcreator={LaTeX via pandoc}}

\author{}
\date{}

\begin{document}

\hypertarget{hw-1-descriptive-statistics-and-measurement-error}{%
\subparagraph{HW 1: Descriptive Statistics and Measurement Error}\label{hw-1-descriptive-statistics-and-measurement-error}}

\hypertarget{january-2023}{%
\subparagraph{31 January 2023}\label{january-2023}}

\hypertarget{kevin-lei}{%
\subparagraph{Kevin Lei}\label{kevin-lei}}

\hypertarget{section}{%
\subparagraph{432009232}\label{section}}

\hypertarget{engr-216-445}{%
\subparagraph{ENGR 216-445}\label{engr-216-445}}

\hypertarget{page-1-of-x}{%
\subparagraph{Page 1 of x}\label{page-1-of-x}}

\hypertarget{given}{%
\subsubsection{Given:}\label{given}}

An excel spreadsheet of proportions of concrete ingredients and its compressive strength

\hypertarget{find}{%
\subsubsection{Find:}\label{find}}

\begin{enumerate}
\def\labelenumi{\alph{enumi})}
\tightlist
\item
  The mean, median, and mode of the water component\\
\item
  The range, variance, and standard deviation of the fine aggregate component\\
\item
  The mean and standard error of the cement component\\
\item
  The mean and standard error of the concrete compressive strength
\end{enumerate}

\hypertarget{diagram}{%
\subsubsection{Diagram:}\label{diagram}}

Not applicable

\hypertarget{theory}{%
\subsubsection{Theory:}\label{theory}}

The mean of a set of data is the sum of all the terms divided by the number of terms.\\
The median of a set of data is the middle value when the data is sorted. If there is an even amount of values, then the median is the mean of the middle two values.\\
The mode is the value that appears most frequently in a set of data.
The range of a set of data is the distance between its mininum and maximum values.\\
The variance of a set of data is the sum of the squares of the difference between each value and the mean all divided by the number of terms.\\
The standard deviation is the average distance from the mean for each value in a set of data. It can be calculated by taking the square root of the variance.\\
The standard error measures the accuracy of a sample and is calculated by dividing the standard deviation of the sample by the square root of the number of terms.

\hypertarget{assumptions}{%
\subsubsection{Assumptions:}\label{assumptions}}

None

\hypertarget{solution}{%
\subsubsection{Solution:}\label{solution}}

\begin{Shaded}
\begin{Highlighting}[]
\ImportTok{import}\NormalTok{ pandas }\ImportTok{as}\NormalTok{ pd}

\NormalTok{data }\OperatorTok{=}\NormalTok{ pd.read\_csv(}\StringTok{"Concrete\_Data.csv"}\NormalTok{)}

\CommentTok{\#a) The mean, median, and mode of the water component (COLUMN INDEX 3)}
\NormalTok{meanWater }\OperatorTok{=}\NormalTok{ data[}\StringTok{"Water  (component 4)(kg in a m\^{}3 mixture)"}\NormalTok{].mean()}
\NormalTok{medianWater }\OperatorTok{=}\NormalTok{ data[}\StringTok{"Water  (component 4)(kg in a m\^{}3 mixture)"}\NormalTok{].median()}
\NormalTok{modeWater }\OperatorTok{=}\NormalTok{ data[}\StringTok{"Water  (component 4)(kg in a m\^{}3 mixture)"}\NormalTok{].mode()}

\CommentTok{\#b) The range, variance, and standard deviation of the fine aggregate component}
\NormalTok{rangeFineAgg }\OperatorTok{=}\NormalTok{ data[}\StringTok{"Fine Aggregate (component 7)(kg in a m\^{}3 mixture)"}\NormalTok{].}\BuiltInTok{max}\NormalTok{() }\OperatorTok{{-}}\NormalTok{ data[}\StringTok{"Fine Aggregate (component 7)(kg in a m\^{}3 mixture)"}\NormalTok{].}\BuiltInTok{min}\NormalTok{()}
\NormalTok{varianceFineAgg }\OperatorTok{=}\NormalTok{ data[}\StringTok{"Fine Aggregate (component 7)(kg in a m\^{}3 mixture)"}\NormalTok{].var()}
\NormalTok{stdevFineAgg }\OperatorTok{=}\NormalTok{ data[}\StringTok{"Fine Aggregate (component 7)(kg in a m\^{}3 mixture)"}\NormalTok{].std()}

\CommentTok{\#c) The mean and standard error of the cement component}
\NormalTok{meanCement }\OperatorTok{=}\NormalTok{ data[}\StringTok{"Cement (component 1)(kg in a m\^{}3 mixture)"}\NormalTok{].mean()}
\NormalTok{errorCement }\OperatorTok{=}\NormalTok{ data[}\StringTok{"Cement (component 1)(kg in a m\^{}3 mixture)"}\NormalTok{].sem()}

\CommentTok{\#d) The mean and standard error of the concrete compressive strength}
\NormalTok{meanCompress }\OperatorTok{=}\NormalTok{ data[}\StringTok{"Concrete compressive strength(MPa, megapascals)"}\NormalTok{].mean()}
\NormalTok{errorCompress }\OperatorTok{=}\NormalTok{ data[}\StringTok{"Concrete compressive strength(MPa, megapascals)"}\NormalTok{].sem()}
\end{Highlighting}
\end{Shaded}

\hypertarget{results}{%
\paragraph{\texorpdfstring{\textbf{\emph{Results:}}}{Results:}}\label{results}}

\textbf{a)}\\
- mean of water component: 181.56728155339806 kilograms per cubic meter\\
- median of water component: 185.0 kilograms per cubic meter\\
- mode of water component: 192.0 kilograms per cubic meter

\textbf{b)}\\
- range of fine aggregate component: 398.6 kilograms per cubic meter\\
- variance of fine aggregate component: 6428.18779179522 kg\textsuperscript{2/m}6\\
- standard deviation of fine aggregate component: 80.17598014240437 kilograms per cubic meter

\textbf{c)}\\
- mean of cement component: 281.16786407766995 kilograms per cubic meter\\
- standard error of cement component: 3.2562978889642458 kilograms per cubic meter

\textbf{d)}\\
- mean of concrete compressive strength: 35.817961165048544 megapascals\\
- standard error of concrete compressive strength: 0.520531668545947 megapascals

\end{document}
